% !TeX root = spherical_harmonics.tex
% !TeX spellcheck = de_DE
\begin{sheet}
	\begin{problem}[title={Basiswechsel zwischen verschiedenen Gelfand-Zetlin-Basen}]
			Man kann zwischen den für die vier Fälle aus \ref{cq:fortsetzung} gefundenen Basen wechseln. Gib hierfür explizite Formeln an. Was sagt dies über die relative Lage der Basisvektoren zueinander aus?
	\end{problem}

	\begin{problem}[title={Gelfand-Zetlin-Wald?}]
		Wo ein Baum ist, gibt es meist mehrere und wenn hinreichend viele aufeinander hocken, hat man einen Wald. Da ein Gelfand-Zetlin-Baum mit einer irreduziblen Darstellung anfängt, liegt die Idee nahe, einen Gelfand-Zetlin-Wald für halbeinfache Darstellungen zu definieren.
		
		Hierbei gilt zu klären: Wie geht man mit mehrfach auftretenden isomorphen oder gleichen Unterdarstellungen um? Ist es sinnvoll, diese alle einzeln als Knoten aufzuzählen?
		
		Finde eine sinnvolle Definition des Gelfand-Zetlin-Waldes.
		
		Hinweis: Man kann einen Verzweigungsgraph definieren. Dies ist ein Gelfand-Zetlin-Baum, der isomorphe Unterräume als einen Knoten miteinander identifiziert. Er enthält genauso viel Information, wie ein Gelfand-Zetlin-Baum, lässt sich aber nicht mit einem Algorithmus konstruieren.
	\end{problem}


\end{sheet}
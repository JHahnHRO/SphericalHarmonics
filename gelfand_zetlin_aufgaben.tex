% !TeX root = spherical_harmonics.tex
% !TeX spellcheck = de_DE
\begin{sheet}
	\begin{problem}[title={Basiswechsel zwischen verschiedenen Gelfand-Zetlin-Basen}]
			Man kann zwischen den für die vier Fälle aus \ref{cq:fortsetzung} gefundenen Basen wechseln. Gib hierfür explizite Formeln an. Was sagt dies über die relative Lage der Basisvektoren zueinander aus?
	\end{problem}

	\begin{problem}[title={Gelfand-Zetlin-Wald}]
		Wo ein Baum ist, gibt es meist mehrere und wenn hinreichend viele aufeinander hocken, hat man einen Wald. Da ein Gelfand-Zetlin-Baum mit einer irreduziblen Darstellung anfängt, liegt die Idee nahe, einen Gelfand-Zetlin-Wald für halbeinfache Darstellungen zu definieren.
		
		Hierbei gilt zu klären: Wie geht man mit mehrfach auftretenden isomorphen oder gleichen irreduziblen Unterdarstellungen um? Ist es sinnvoll, diese alle einzeln als Knoten aufzuzählen?
		
		Finde eine sinnvolle Definition des Gelfand-Zetlin-Waldes.
		
		Hinweise:
		\begin{itemize}
		 \item Man kann einen \emph{Verzweigungsgraph} definieren. Dies ist ein Gelfand-Zetlin-Baum, der isomorphe Unterräume als einen Knoten miteinander identifiziert. Er enthält genauso viel Information, wie ein Gelfand-Zetlin-Baum.
		\item Kann man kann es sich durch geschickte Einschränkung auf bestimmte halbeinfache Darstellungen relativ einfach machen?
		\item Nach welchen Kriterien kann man die irreduziblen Unterdarstellungen voneinander trennen?
		\end{itemize}
	\end{problem}


\begin{problem}\label{gelfand_zetlin:ex:Delta_vs_rQuadrat}
Betrachte einen $n$-dimensionalen $\IR$-Vektorraum $V$ mit Skalarprodukt.

Es sei $r^2$ das Polynom $x_1^2+x_2^2+\ldots+x_n^2$ und $\Delta=\partial_1^2 + \partial_2^2 +\ldots+\partial_n^2$ der Laplace-Operator für $V$.

\begin{subproblem}
Zeige: $\rho:=p\mapsto r^2\cdot p$ und $\Delta$ sind $O_n$-lineare Abbildungen $\IK[x_1,\ldots,x_n]\to\IK[x_1,\ldots,x_n]$.
\end{subproblem}

\begin{subproblem}
Berechne $(\rho\circ\Delta - \Delta\circ\rho)(p)$ für ein homogenes Polynom $p$ vom Grad $l$.
\end{subproblem}

\begin{subproblem}
In unserer expliziten Formel für die (komplexen) spherical harmonics kamen Ausdrücke der Form $(\partial_x^2+\partial_y^2)^a\left((x^2+y^2)^b (x\pm iy)^m\right)$ vor. Beweise, dass sich $(x+iy)^m$ aus diesem Ausdruck ausklammern lässt.
\end{subproblem}

\end{problem}
\end{sheet}
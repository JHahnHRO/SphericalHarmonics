% !TeX spellcheck = de_DE
\documentclass[fontsize=11pt,fleqn,a4paper]{scrartcl}

\author{Andrea Hanke \and Johannes Hahn}
\title{Warum spherical harmonics?}
%\subtitle{}
\date{08.08.21 -- 15.08.21}

%%% General math packages

% AMS
\usepackage{amsmath,  
            amssymb,  % Symbols
            amsthm,   % provides theorem environments
            amsfonts  % fonts like \mathbb and \mathfrak
            }

% useful things for math typesetting like \smash, \psmallmatrix, ...
\usepackage{mathtools}

% For \Set and \set. Automatic resizing of the curly braces and the middle vertical line
\usepackage{braket}   

% For even more extensible arrows
\usepackage{extpfeil} 

% Blockmatrices
%\usepackage{multirow}

% and brakets around matrix rows
%\usepackage{bigdelim}


%%% Own symbols and operators
\newcommand{\IN}{\mathbb{N}}
\newcommand{\IZ}{\mathbb{Z}}
\newcommand{\IQ}{\mathbb{Q}}
\newcommand{\IR}{\mathbb{R}}
\newcommand{\IC}{\mathbb{C}}
\newcommand{\IK}{\mathbb{K}}

\DeclarePairedDelimiter{\abs}{\lvert}{\rvert}
\DeclarePairedDelimiter{\norm}{\lVert}{\rVert}
\DeclarePairedDelimiter{\ceil}{\lceil}{\rceil}
\DeclarePairedDelimiter{\floor}{\lfloor}{\rfloor}

\newcommand{\isomorphic}{\cong}

\renewcommand{\Im}{\operatorname{\mathfrak{Im}}}
\renewcommand{\Re}{\operatorname{\mathfrak{Re}}}

\newcommand{\dd}{\,\textrm{d}}

\DeclareMathOperator{\id}{id}
\DeclareMathOperator{\tr}{tr}
\DeclareMathOperator{\casimir}{\mathbf{\Omega}}
\DeclareMathOperator{\Hom}{Hom}
\DeclareMathOperator{\End}{End}
\DeclareMathOperator{\Aut}{Aut}
\DeclareMathOperator{\Sym}{Sym}
\DeclareMathOperator{\STF}{STF}

\DeclareMathOperator{\sgn}{sgn}
\DeclareMathOperator{\ord}{ord}

\DeclareMathOperator{\im}{im}


\newcommand{\vc}[1]{\ensuremath{\mathbf{#1}}}

\newcommand{\mapsfrom}{\mathrel{\reflectbox{\ensuremath{\mapsto}}}}

%%% Theorem environments

% Define name strings
\newcommand{\captionstringtheorem}{Theorem}
\newcommand{\captionstringcentralquestion}{Central Question}
\newcommand{\captionstringmaintheorem}{Main Theorem}
\newcommand{\captionstringlemma}{Lemma}
\newcommand{\captionstringcorollary}{Corollary}
\newcommand{\captionstringlemmadef}{Lemma and definition}
\newcommand{\captionstringtheoremdef}{Theorem and definition}
\newcommand{\captionstringdefinition}{Definition}
\newcommand{\captionstringproposition}{Proposition}
\newcommand{\captionstringexample}{Example}
\newcommand{\captionstringconjecture}{Conjecture}
\newcommand{\captionstringconvention}{Convention}
\newcommand{\captionstringremark}{Remark}


% Definition of two similar styles
\newtheoremstyle{dotless} % Name
			{\bigskipamount}    % Space above
			{\bigskipamount}    % Space below
			{\nopagebreak}      % Body font, also suppress pagebreak between "Theorem 3.14:" and text
			{}                  % Indent amount
			{\bfseries}         % Theorem head font
			{:}                 % Punctuation after theorem head
			{\newline}          % Space after theorem head
			{}                  % Theorem head spec (can be left empty, meaning 'normal')
\newtheoremstyle{dotless2} % Name
			{\bigskipamount}    % Space above
			{0.0em}             % Space below
			{}                  % Body font
			{}                  % Indent amount
			{\bfseries}         % Theorem head font
			{:}                 % Punctuation after theorem head
			{0.5em}             % Space after theorem head
			{}                  % Theorem head spec (can be left empty, meaning 'normal')
% "3.14 Theorem" instead of "Theorem 3.14"
\swapnumbers

\newcounter{theoremnumber}
\numberwithin{theoremnumber}{section}

\theoremstyle{dotless}
\newtheorem{theorem}[theoremnumber]{\captionstringtheorem}
\newtheorem{maintheorem}[theoremnumber]{\captionstringmaintheorem}
\newtheorem{centralquestion}[theoremnumber]{\captionstringcentralquestion}
\newtheorem{theoremdef}[theoremnumber]{\captionstringtheoremdef}
\newtheorem{lemma}[theoremnumber]{\captionstringlemma}
\newtheorem{lemmadef}[theoremnumber]{\captionstringlemmadef}
\newtheorem{corollary}[theoremnumber]{\captionstringcorollary}
\newtheorem{proposition}[theoremnumber]{\captionstringproposition}
\newtheorem{definition}[theoremnumber]{\captionstringdefinition}
\newtheorem{example}[theoremnumber]{\captionstringexample}
\newtheorem{conjecture}[theoremnumber]{\captionstringconjecture}

\newtheorem*{convention}{\captionstringconvention}

\theoremstyle{dotless2}
\newtheorem{remark}[theoremnumber]{}


%%% Math style file

% Sets the strictness of handling page breaks inside align-Umgebungen
% Level [1] breaks will be avoided if possible,
% Levels [2], [3], [4] are increasingly relaxed.
\allowdisplaybreaks[1]


%%% Everything tikz

\usepackage{tikz}
\usetikzlibrary{arrows}
\usetikzlibrary{cd}
\usetikzlibrary{babel}

% Predefine styles
\tikzset
{
	desc/.style=
	{
		fill=white,inner sep=2pt,font=\scriptsize
	}
}


\tikzstyle{v}=[draw, fill =black, circle, inner sep=0pt, minimum size=1.5pt]
\tikzstyle{b}=[draw=blue, fill =blue, circle, inner sep=0pt, minimum size=1.5pt]
\tikzstyle{r}=[draw=red, fill =red, circle, inner sep=0pt, minimum size=1.5pt]


\newcommand{\karos}[2]{
	\begin{tikzpicture}
	\draw[step=0.5cm,color=gray] (0,0) grid (#1 cm ,#2 cm);
	\end{tikzpicture}
}
\newcommand\MyGrid[3]{%
	\begin{tikzpicture}[remember picture,overlay]
	\draw[step=5mm,color=#1] (0,0) grid (#3,#2);
	\draw[color=#1] (0,#2) --  (#3,#2);
	\end{tikzpicture}%
}

\newlength\MaxHt
\newsavebox\mybox

\newenvironment{gridmp}[2][lightgray]
{\def\mycolor{#1}
	\begin{lrbox}{\mybox}%
		\begin{minipage}{#2}}
		{\end{minipage}%
	\end{lrbox}%
	\setlength\MaxHt{\dp\mybox}\addtolength\MaxHt{1.\ht\mybox}
	\noindent%
	\raisebox{-\dp\mybox}{\MyGrid{\mycolor}{\MaxHt}{\wd\mybox}}%
	\usebox{\mybox}
	\vspace{0.5cm}}

%%%% Styles for algorithms

\usepackage{listings}
\lstset{%
	basicstyle = \ttfamily\small,
	tabsize = 3
}
% Use theorem environment for algorithm descriptions
\newtheorem{algorithm}{Algorithmus}
% Change numbering of algorithms to include chapter
\renewcommand{\thealgorithm}{\Alph{chapter}.\Roman{algorithm}}
%%% Tables and figures
\numberwithin{table}{section}
\numberwithin{figure}{section}
\usepackage{graphicx}

\usepackage{tabularx}


%%% Text styles

% For the interrobang
\usepackage{textcomp}

% Additional underlining options, especially dotted underlining, line breaks in underlined text etc.
\usepackage[normalem]{ulem} % option not to change look of \emph{}
\newcommand*{\udot}{\dotuline}


% Skip lengths
\setlength{\parindent}{0em}
\setlength{\parskip}{0em}


% Pimp enumerate and itemize environments. More counter options and resuming numbering
\usepackage{enumitem}

% Style of enumerations and itemizations
\renewcommand{\labelenumi}{\alph{enumi}.)}  % Counter enumi wird immer als a.) b.) c.) dargestellt.
\renewcommand{\labelenumii}{\roman{enumii}.)}  % Counter enumii wird immer als i.) ii.) iii.) dargestellt.


\usepackage{csquotes}

%\usepackage{multicol}

\usepackage{siunitx}


% Für schönere Tabellen
\usepackage{booktabs}
\usepackage{array}
\newcolumntype{C}[1]{>{\centering}p{#1}}

%Zum Einbinden von externen pdf-Dokumenten
\usepackage{pdfpages}

%%%% Bibliography with BibTeX only

\usepackage[numbers]{natbib}
\bibliographystyle{plain}


%%%% Bibliography with biblatex + bibtex

\usepackage[
	style=alphabetic,
	sorting=nty,
	url=false,
	natbib=true,
	backend=biber,  % Biber backend
	defernumbers=true
]{biblatex}
% The .bib-file(s) for this document
\addbibresource{wgraph.bib}


%\input{_preamble/biblatex_biber.tex}

% !! Hyperref before imakeidx !!
%%% PDF stuff
%%%
%%% Include hyperref before imakeidx !!

\usepackage[
	pdfpagelabels=true,
	plainpages=false
	]{hyperref}
	
\hypersetup{
colorlinks=true,
linkcolor=red,
urlcolor=blue,
citecolor=blue,
linktocpage=true, % Page numbers will be the links in t.o.c instead of the headings themselves
pdfpagelayout={OneColumn},
pdfstartview= % empty to cause the viewer to use its preferred behaviour instead of dictating a behaviour at opening of the document
}

\makeatletter
{\hypersetup{
pdfinfo=
	{  
		Title={\@title},
		Author={Johannes Hahn, Andrea Hanke},
		Keywords={Irreducible representations, tensors, spherical harmonics, Gelfand-Tsetlin basis},
		Subject={Representation theory}
	}
}}
\makeatother
%\input{_preamble/indicies.tex}

\usepackage{exframe}
\exercisesetup{
	solutions=false,     % display solutions?
	solutionhref=true    % hyperlink solutions to problems
}
\exercisestyle{
	solutionbelow=sheet,% Display solutions under each sheet
	problembysheet,      % number problems with sheet, e.g. problem 2.4.
%	pagebysheet,         % Number pages with sheets
	pointsat=start,
	subpointsat=margin,
	fracpoints=true      % fractional points
}

%% Config of sheets
\exerciseconfig{toclevelsheet}{subsection}

\exerciseconfig{insertsheettitle}{\subsection*{\getexerciseconfig{termproblems}}}
\exerciseconfig{insertsheetafter}{}

%% Config of problems

\exerciseconfig{toclevelproblem}{subsubsection}
\exerciseconfig{styletitleproblem}{}

\exerciseconfig{skipproblemabove}{\bigskipamount}
\exerciseconfig{skipproblembelow}{\bigskipamount}
\exerciseconfig{skipproblemtitle}{1pt}

\exerciseconfig{composetitleproblem}[2]{\getexerciseconfig{termproblem} \getexerciseconfig{composeitemproblem}{#1}\exerciseifempty{#2}{}{ -- #2}}
\showprobleminfo{difficulty}

\exerciseconfig{composepointspair}[2]{%
  \ifdim#2pt=0pt% If zero bonus points
   \getexerciseconfig{composepoints}{#1}%
  \else\ifdim#1pt=0pt% If only bonus points
   +\getexerciseconfig{composepointsnum}{#2}~Bonus%
  \else% If both non-zero
   \getexerciseconfig{composepoints}{#1}+%
   \getexerciseconfig{composepointsnum}{#2}~Bonus%
  \fi\fi%
}
\exerciseconfig{composepointspairstart}[2]{\hfill(%
   \getexerciseconfig{composepointspair}{#1}{#2}%
)}
\exerciseconfig{composepointspairmargin}[2]{(%
  \ifdim#2pt=0pt% If zero bonus points
   \getexerciseconfig{composepointsnum}{#1}%
  \else\ifdim#1pt=0pt% If only bonus points
   +\getexerciseconfig{composepointsnum}{#2}~B%
  \else% If both non-zero
   \getexerciseconfig{composepointsnum}{#1}+%
   \getexerciseconfig{composepointsnum}{#2}~B%
  \fi\fi%
)}

%% Config of subproblems
\exerciseconfig{countersubproblem}{\alph{subproblem}.)}

%% Config of solutions
\exerciseconfig{styletitlesolutions}{\Large}
\exerciseconfig{insertsolutionsbefore}{\pagebreak}


%%% Language preamble for german

% Language itself
\usepackage[ngerman]{babel}
% Font encoding to represent umlauts correct in PDFs documents instead of combining them from other
% characters like "u instead of ü. Do not use with XeLaTeX or LuaTeX
\usepackage[T1]{fontenc}
% Encoding of the source code. Do not use with XeLaTeX or LuaTeX
\usepackage[utf8]{inputenc}


% Language specific strings
%   Names of theorem environments

\addto\captionsngerman{
	% \see command from makeidx
%	\@ifpackageloaded{makeidx}{
%		\renewcommand{\seename}{siehe}
%	}{}
	\renewcommand{\captionstringtheorem}{Satz}
	\renewcommand{\captionstringmaintheorem}{Hauptsatz}
	\renewcommand{\captionstringcentralquestion}{Zentrale Fragestellung}	
	\renewcommand{\captionstringlemma}{Lemma}
	\renewcommand{\captionstringcorollary}{Korollar}
	\renewcommand{\captionstringlemmadef}{Lemma und Definition}
	\renewcommand{\captionstringtheoremdef}{Satz und Definition}
	\renewcommand{\captionstringdefinition}{Definition}
	\renewcommand{\captionstringproposition}{Proposition}
	\renewcommand{\captionstringexample}{Beispiel}
	\renewcommand{\captionstringconjecture}{Vermutung}
	\renewcommand{\captionstringconvention}{Vereinbarung}
	\renewcommand{\captionstringremark}{Bemerkung}
}
\makeatletter
%\@ifpackageloaded{biblatex}{
%\DefineBibliographyStrings{german}{
%	bibliography = {Bibliographie},
%	references = {Referenzen}
%}
%}{}
\@ifpackageloaded{exframe}{
	\exerciseconfig{termsheet}{Blatt}
	\exerciseconfig{termsheets}{Blätter}
	\exerciseconfig{termproblem}{Aufgabe}
	\exerciseconfig{termproblems}{Aufgaben}
	\exerciseconfig{termsolution}{Lösung}
	\exerciseconfig{termsolutions}{Lösungen}
	\exerciseconfig{termpoint}{Punkt}
	\exerciseconfig{termpoints}{Punkte}
}
\makeatother



\begin{document}

\maketitle

\tableofcontents
\pagebreak
\setcounter{section}{-1}
\section{Wiederholung: Gruppen, Vektorräume und bilineare Abbildungen}
% !TeX root = spherical_harmonics.tex
% !TeX spellcheck = de_DE


\begin{definition}[Gruppen und Gruppenhomomorphismen]\label{gruppen:def}
Eine \udot{Gruppe} $(G,\cdot)$ besteht aus
\begin{itemize}
	\item einer Menge $G$,
	\item einer Abbildung $\cdot: G\times G, (g,h) \mapsto g\cdot h$, genannt \emph{Multiplikation},
\end{itemize}
die die Gruppen-Axiome in Tabelle \ref{gruppen:def_table} erfüllen.

\begin{table}[!ht]
	\setlength\extrarowheight{10pt} % for a bit of visual "breathing space"
	\begin{tabularx}{\textwidth}{p{7cm} X}
		
		\toprule
		\textbf{Gruppen-Axiome}                               & \textbf{Bedeutung} \\
		\midrule
        \hspace{1cm}Assoziativität                           & $\forall g_1,g_2,g_3\in G: g_1\cdot(g_2\cdot g_3) = (g_1\cdot g_2)\cdot g_3$ \\
		\hspace{1cm}Neutrales Element bzw. \enquote{Eins}    & $\exists 1\in V\forall g\in G: g\cdot 1 = g = 1\cdot g$  \\
		\hspace{1cm}Inverse Elemente                         & $\forall g \in G\exists h \in G: g\cdot h = 1 = h\cdot g$ \\
		Optional kann gelten & \\
        \hspace{1cm}Kommutativität                           & $\forall g,h\in G: g\cdot h=h\cdot g$ \\
        \textbf{Axiom von Gruppenhomomorphismen}             & \textbf{Bedeutung} \\
        \midrule
        Verträglichkeit mit Multiplikation & $\forall g,h\in G: f(g\cdot h) = f(g)\cdot f(h)$ \\
        \bottomrule
	\end{tabularx}
	\caption{Definierende Eigenschaften von Gruppen und Gruppenhomomorphismen}
    \label{gruppen:def_table}
\end{table}

Sind $G,H$ zwei Gruppen und $f: G\to H$ eine Abbildung, so heißt $f$ \emph{Gruppenhomomorphismus}, falls sie mit den beiden Multiplikationen verträglich ist.

Existiert ein Homomorphismus $f': W\to V$ mit $f\circ f'=f'\circ f=\id$, so nennt man $f$ \emph{Isomorphismus} der Gruppen.
\end{definition}

% !TeX spellcheck = de_DE
% !TeX root = spherical_harmonics.tex
\subsection{Vektorräume}
Wir kennen uns bereits mit Vektorräumen aus. Typische Beispiele sind $\IR^3$ oder der Raum der Polynome von einem bestimmten Grad $n$. Da uns in diesem Kurs noch weitere Vektorräume begegnen werden und zum Vergleich mit anderen mathematischen Objekten (z.B. Darstellungen), ist hier ihre Definition zusammengefasst.


\begin{definition}[Vektorräume und lineare Abbildungen]\label{vektorraeume:def}
Sei $K$ ein Körper. Ein \udot{$K$-Vektorraum} $(V,+,\cdot)$ besteht aus
\begin{itemize}
	\item einer Menge $V$,
	\item einer Abbildung $+: V \times V \to V, (v,w) \mapsto v+w$, genannt \emph{(Vektor)addition} und
	\item einer Abbildung $\cdot: K \times V \to V, (\lambda,v) \mapsto \lambda\cdot v$, genannt \emph{Skalarmultiplikation}.
\end{itemize}
die die Vektorraum-Axiome in Tabelle \ref{vektorraeume:def_table} erfüllen. Alles, was Element eines Vektorraums ist, kann \emph{Vektor} genannt werden.

\begin{table}[!ht]
	\setlength\extrarowheight{10pt} % for a bit of visual "breathing space"
	\begin{tabularx}{\textwidth}{p{7cm} X}
		
		\toprule
		\textbf{Vektorraum-Axiom}                            & \textbf{Bedeutung} \\
		\midrule
        $(V,+)$ ist eine abelsche Gruppe                     & Addition verhält sich wie erwartet \\
		\hspace{1cm}Assoziativität                           & $\forall u,v,w\in V: u+(v+w) = (u+v)+w$  \\
		\hspace{1cm}Neutrales Element bzw. \enquote{Null}    & $\exists 0\in V\forall v\in V: v+0=0+v$  \\
		\hspace{1cm}Inverse bzw. \enquote{negative} Elemente & $\forall v \in V\exists w \in V: v+w=0$ \\
        \hspace{1cm}Kommutativität                           & $\forall u,v\in V: u+v=v+u$ \\
        Eigenschaften der Skalarmultiplikation \\
		\hspace{1cm}Assoziativität                           & $\forall a,b\in K, v\in V: a\cdot(b\cdot v) = (a\cdot b)\cdot v$ \\
		\hspace{1cm}Normierung bzw. Nichttrivialität         & $\forall v \in V: 1 \cdot v = v$, wobei $1$ das Einselement des Körpers bezeichnet \\
        Verträglichkeit von Addition und Skalarmultiplikation \\
		\hspace{1cm}Distributivität bzgl. $(V,+)$            & $\forall a\in K, u,v\in V: a\cdot(u + v) = a\cdot u + a\cdot v$ \\
		\hspace{1cm}Distributivität bzgl. $(K,+)$            & $\forall a,b\in K, v\in V: (a + b)\cdot v = a\cdot v + b\cdot v$ \\
        \midrule
        \textbf{Axiome linearer Abbildungen}                  & \textbf{Bedeutung} \\
        \midrule
        Additivität & $\forall v_1,v_2\in V: f(v_1+v_2) = f(v_1)+f(v_2)$ \\
        Homogenität & $\forall \lambda\in K, v\in V: f(\lambda\cdot v) = \lambda \cdot f(v)$ \\
        \bottomrule
	\end{tabularx}
	\caption{Definierende Eigenschaften von Vektorräumen und linearen Abbildungen}
    \label{vektorraeume:def_table}
\end{table}

Sind $V,W$ zwei $K$-Vektorräume und $f: V\to W$ eine Abbildung, so heißt $f$ \emph{($K$-)lineare Abbildung} oder \emph{(Vektorraum-)Homomorphismus}, falls die beiden Axiome in Tabelle \ref{vektorraeume:def_table} erfüllt sind. Den Raum aller $K$-linearen Abbildungen von $V$ nach $W$ bezeichnen wir mit $\Hom_K(V,W)$. Eine $K$-linearen Abbildung von $V$ nach $V$ (also gleicher Definitions- und Zielraum) heißt \emph{(Vektorraum-)Endomorphismus}, der Raum aller solcher Abbildungen wird mit $\End_K(V)$ notiert.

Existiert ein Homomorphismus $f': W\to V$ mit $f\circ f'=f'\circ f=\id$, so nennt man $f$ \emph{Isomorphismus} der Vektorräume.
\end{definition}

\begin{remark}
Wir sind praktisch ausschließlich an $\IR$- und $\IC$-Vektorräumen interessiert in diesem Kurs.
\end{remark}

\begin{definition}[Kern \& Bild]
Ist $f: V\to W$ eine $K$-lineare Abbildung, so ist
\[\ker(f) := \Set{v\in V | f(v)=0}\]
der \emph{Kern von $f$} und
\[\im(f) := \Set{w\in W | \exists v\in V: f(v)=w}\]
das \emph{Bild von $f$}.
\end{definition}

\subsection{Bilineare Abbildungen}

\begin{remark}
	Es gibt i.A. keine Multiplikation zweier Vektoren miteinander in irgendeinem Sinne. Wir können immer Skalare mit Vektoren multiplizieren, aber nicht Vektoren mit Vektoren. Nichts desto trotz ist es \emph{manchmal} doch so, dass zusätzlich zu Addition und Skalarmultiplikation eine weitere Operation existiert, die ein sinnvolles Konzept von Multiplikation liefert, z.B.
	\begin{enumerate}
		\item Der Vektorraum der Polynome $\IR[X]$ hat die Polynommultiplikation, d.h.
		\[\left(\sum_{i=0}^n a_i X^i\right) \cdot \left(\sum_{j=0}^m b_j X^j\right) := \sum_{k=0}^{n+m} (\sum_{\substack{i,j \\ i+j=k}} a_i b_j) X^k\]
		\item Der Vektorraum der Funktionen $X\to\IC$ für einen festen Definitionsbereich $X$ hat die punktweise Multiplikation, d.h.
		\[f\cdot g := x\mapsto f(x)g(x)\]
		\item Die Hintereinanderausführung von linearen Abbildungen $\Hom_K(V,W) \times \Hom_K(U,V) \to \Hom_K(U,W), (f,g) \mapsto f\circ g$ ist eine Abbildung, die sich in vielerlei Hinsicht auch wie eine Multiplikation verhält. Für $U=V=W$ erhält man insbesondere eine Multiplikation $\End_K(V) \times \End_K(V) \to \End_K(V)$.
		
		Nach Wahl von je einer Basis können wir $\Hom_K(V,W)$, $\Hom_K(U,V)$ und $\Hom_K(U,W)$ mit $K^{n\times m}$, $K^{m\times p}$ und $K^{n\times p}$ identifizieren. Die Hintereinanderausführung von linearen Abbildung entspricht dann der Matrixmultiplikation.
		\item ...
	\end{enumerate}
	
\end{remark}

\begin{definition}[Bilineare \& multilineare Abbildungen]\label{bilineare_abb:def}
	Sind $V,W,X$ drei $K$-Vektorräume, so heißt eine Abbildung $\phi: V\times W\to X$ \emph{bilinear}, falls sie beiden Linearitätsbedingungen in Tabelle \ref{bilineare_abb:def_table} erfüllt, d.h. die Funktion ist separat linear, wenn man nur den ersten oder nur den zweiten Input variiert und den anderen festhält.
	
	\begin{table}[!ht]
		\setlength\extrarowheight{10pt} % for a bit of visual "breathing space"
		\hspace{-0.1\textwidth}
		\begin{tabularx}{1.2\textwidth}{p{6.5cm} p{9.5cm}}
			\toprule
			\textbf{Axiom}                  & \textbf{Bedeutung} \hspace{0.5cm} \\ 
			\midrule
			Linearität: \\
			\hspace{1cm}im ersten Argument  & $\forall v_1,v_2\in V, w\in W: \phi(v_1+v_2,w) = \phi(v_1,w) + \phi(v_2,w)$ \\
			& $\forall \lambda\in K, v\in W, w\in W: \phi(\lambda v, w) = \lambda \phi(v,w)$ \\
			\hspace{1cm}im zweiten Argument & $\forall v\in V, w_1, w_2\in W: \phi(v,w_1+w_2) = \phi(v,w_1) + \phi(v,w_2)$ \\
			& $\forall \lambda\in K, v\in W, w\in W: \phi(v, \lambda w) = \lambda \phi(v,w)$ \\
			Falls $\phi: V\times V \to X$, kann optional gelten: \\
			\hspace{1cm}Symmetrie bzw. Kommutativität & $\forall u,v\in V: \phi(u,v) = \phi(v,u)$ \\
			\hspace{1cm}Antisymmetrie & $\forall u,v\in V: \phi(u,v) = -\phi(v,u)$ \\
			Falls $\phi: V\times V\to V$, kann optional gelten: \\
			\hspace{1cm}Assoziativität & $\forall u,v,w\in V: \phi(u,\phi(v,w)) = \phi(\phi(u,v),w)$ \\
			\bottomrule
		\end{tabularx}
		\label{bilineare_abb:def_table}
		\caption{Eigenschaften von bilinearen Abbildungen}
	\end{table}
	
	Für Abbildungen $V_1\times V_2\times V_3\to X$, die von drei Inputvektoren abhängig sind, kann man entsprechend definieren, dass eine Abbildung \emph{trilinear} heißt, wenn sie die drei Distributivgesetze erfüllt. Für vier, fünf, ... $k$ Input-Vektoren spricht man entsprechend von \emph{$k$-fach linearen} Abbildungen.
\end{definition}

\begin{example}
	Die drei genannten \enquote{Multiplikationen} von oben sind bilinear. Polynommultiplikation und punktweise Multiplikation von Funktionen sind kommutativ und assoziativ. Die Komposition $\End_K(V)\times\End_K(V)\to\End_K(V)$ ist assoziativ, aber nicht kommutativ, falls $\dim(V) > 1$.
	
	Es gibt viele weitere, äußerst nützliche Beispiele.
	\begin{enumerate}[start=4]
		\item Richtungsableitungen sind bilinear in der Richtung und der Funktion:
		
		Sei $X\subseteq\IR^n$ ein geeigneter Definitionsbereich (z.B. eine offene Menge). Sei außerdem $f:X \to\IC$ eine differenzierbare Funktion. Dann existieren insbesondere alle Richtungsableitungen $(\partial_vf)(x_0) = \lim_{t\to 0} \frac{f(x_0)-f(x_0+tv)}{t}$ in allen Punkten $x_0\in X$. Die Abbildungsvorschrift $(v,f) \mapsto \partial_v f$ liefert eine bilineare Abbildung für mehrere Kombinationen von Vektorräumen, z.B. $\IR^n \times C^1(\IR^n) \to C^0(\IR^n)$ oder $\IR^n \times C^\infty(\IR^n) \to C^\infty(\IR^n)$.
		
		\item Allgemeiner: Lineare Differentialoperatoren:
		
		Für mehrfach differenzierbare Funktionen kann man natürlich auch mehrere Ableitungsschritte hintereinander ausführen. Auf diese Weise erhält man multilineare Abbildungen: $k$-faches Ableiten in $k$ Richtungen, also der Differentialoperator $\partial_{v_1} \partial_{v_2} \cdots \partial_{v_k}$ ist $k$-fach linear in den Vektoren $v_1, ..., v_k$ als Inputs. Die Anwendung auf eine Funktion ist entsprechend $(k+1)$-fach linear in den $k$ Richtungsvektoren und der Funktion als Inputs.
		
		Als $k$-lineare Abbildung ist dies eine symmetrische Abbildung, d.h. $\partial_v \partial w f = \partial_w\partial_v f$, sofern $f$ zweimal stetig differenzierbar ist. Das ist der Satz von Schwarz\footnote{Hermann Amandus Schwarz (1843 -- 1921), dt. Mathematiker.}.
		
		\item Kreuzprodukt:
		
		Es gibt eine besondere bilineare Abbildung $\textsc{x} : \IR^3 \times \IR^3 \to \IR^3$, genannt \udot{Kreuzprodukt}, die in anderen Dimensionen keine gute Analogie hat, nämlich
		
		\[\begin{pmatrix}x_1\\y_1\\z_1\end{pmatrix} \textsc{x} \begin{pmatrix}x_2\\y_2\\z_2\end{pmatrix} := \begin{pmatrix}
			y_1 z_2 - y_2 z_1 \\
			x_1 z_2 - x_2 z_1 \\
			x_1 y_2 - x_2 y_1
		\end{pmatrix}\]
		Es ist bilinear, aber weder kommutativ noch assoziativ. Stattdessen ist es antisymmetrisch.
	\end{enumerate}
\end{example}

% !TeX root = spherical_harmonics.tex
% !TeX spellcheck = de_DE
\begin{sheet}[title={Aufgaben, die in einer LA-Vorlesung dran gewesen sein könnten}]

\begin{problem}[title={Simultane Diagonalisierbarkeit}]\label{ex:simultan_diagonalisieren}
Es seien $\alpha_1,\ldots,\alpha_n$ diagonalisierbare Endomorphismen eines $K$-Vektorraums $V$. Es gelte $\alpha_i\circ\alpha_j = \alpha_j\circ\alpha_i$ für alle $1\leq i,j\leq n$.

Ziel dieser Aufgabe ist es, zu zeigen, dass die Endomorphismen \emph{simultan diagonalisierbar sind}, d.h. dass eine Basis von $V$ existiert mit der Eigenschaft, dass jeder Vektor in $B$ ein Eigenvektor für alle $\alpha_i$ ist.

\begin{subproblem}
Zeige, dass $\alpha_1$ sich zu einem Endomorphismus $\alpha_{1,\mu}$ von $\operatorname{Eig}_\mu(\alpha_2)$ einschränkt für $\lambda\in K$.
\end{subproblem}
\begin{subproblem}
Zeige, dass $\operatorname{Eig}_\lambda(\alpha_{1,\mu}) = \operatorname{Eig}_\lambda(\alpha_1) \cap \operatorname{Eig}_\mu(\alpha_2)$ und
\[\operatorname{Eig}_\lambda(\alpha_1) = \bigoplus_{\mu\in\sigma(\alpha_2)} \operatorname{Eig}_\lambda(\alpha_{1,\mu})\]
gilt.
\end{subproblem}
\begin{subproblem}
Folgere daraus die Aussage für $n=2$, indem du zeigst, dass $\alpha_{1,\mu}$ diagonalisierbar ist.
\end{subproblem}
\begin{subproblem}
Beweise den allgemeinen Fall per Induktion. 
\end{subproblem}
\begin{subproblem}
Beweise auch den Fall von unendlich vielen, paarweise kommutierenden $\alpha$ unter der Annahme, dass $V$ endlich-dimensional ist.
\end{subproblem}
\end{problem}

\begin{problem}[title={Hermite Normalform}]
$A\in\IC^{n\times n}$ sei beliebig. Zeige:
\begin{subproblem}
Es gibt eine unitäre Matrix $U$, sodass 
\[U^H A U = \begin{pmatrix}
\lambda_1&\ast\\0&A'
\end{pmatrix}\]
ist. Hinweis: Was heißt das für die erste Spalte von $U$?
\end{subproblem}
\begin{subproblem}
Folgere: Es gibt eine (andere) unitäre Matrix $U$, sodass $U^H A U$ obere Dreiecksgestalt hat.
\end{subproblem}
Hinweis: Benutze, dass alle Polynome über $\IC$ vollständig in Linearfaktoren zerfallen.
\end{problem}

\begin{problem}[title={Normale Matrizen}]\label{ex:normale_matrizen}
Eine Matrix $A\in\IC^{n\times n}$ heißt \udot{normal}, falls $AA^H=A^H A$ gilt. Beispiele für normale Matrizen sind hermitesche und unitäre Matrizen, insbesondere auch reelle symmetrische bzw. orthogonale Matrizen.

Zeige:
\begin{subproblem}
Wenn $A$ eine Dreiecksmatrix und gleichzeitig eine normale Matrix ist, dann ist $A$ automatisch eine Diagonalmatrix.
\end{subproblem}
\begin{subproblem}\label{ex:unitaere_matrizen_diagonalisierbar}
Folgere: Normale Matrizen sind immer unitär diagonalisierbar.
\end{subproblem}
\begin{subproblem}
Umgekehrt ist jede unitär diagonalisierbare Matrix auch normal.
\end{subproblem}
\end{problem}


\end{sheet}
\pagebreak

\section{Warum Tensoren?}
% !TeX root = spherical_harmonics.tex
% !TeX spellcheck = de_DE

\subsection{Wiederholung: Vektorräume}
% !TeX spellcheck = de_DE
% !TeX root = spherical_harmonics.tex
\subsection{Vektorräume}
Wir kennen uns bereits mit Vektorräumen aus. Typische Beispiele sind $\IR^3$ oder der Raum der Polynome von einem bestimmten Grad $n$. Da uns in diesem Kurs noch weitere Vektorräume begegnen werden und zum Vergleich mit anderen mathematischen Objekten (z.B. Darstellungen), ist hier ihre Definition zusammengefasst.


\begin{definition}[Vektorräume und lineare Abbildungen]\label{vektorraeume:def}
Sei $K$ ein Körper. Ein \udot{$K$-Vektorraum} $(V,+,\cdot)$ besteht aus
\begin{itemize}
	\item einer Menge $V$,
	\item einer Abbildung $+: V \times V \to V, (v,w) \mapsto v+w$, genannt \emph{(Vektor)addition} und
	\item einer Abbildung $\cdot: K \times V \to V, (\lambda,v) \mapsto \lambda\cdot v$, genannt \emph{Skalarmultiplikation}.
\end{itemize}
die die Vektorraum-Axiome in Tabelle \ref{vektorraeume:def_table} erfüllen. Alles, was Element eines Vektorraums ist, kann \emph{Vektor} genannt werden.

\begin{table}[!ht]
	\setlength\extrarowheight{10pt} % for a bit of visual "breathing space"
	\begin{tabularx}{\textwidth}{p{7cm} X}
		
		\toprule
		\textbf{Vektorraum-Axiom}                            & \textbf{Bedeutung} \\
		\midrule
        $(V,+)$ ist eine abelsche Gruppe                     & Addition verhält sich wie erwartet \\
		\hspace{1cm}Assoziativität                           & $\forall u,v,w\in V: u+(v+w) = (u+v)+w$  \\
		\hspace{1cm}Neutrales Element bzw. \enquote{Null}    & $\exists 0\in V\forall v\in V: v+0=0+v$  \\
		\hspace{1cm}Inverse bzw. \enquote{negative} Elemente & $\forall v \in V\exists w \in V: v+w=0$ \\
        \hspace{1cm}Kommutativität                           & $\forall u,v\in V: u+v=v+u$ \\
        Eigenschaften der Skalarmultiplikation \\
		\hspace{1cm}Assoziativität                           & $\forall a,b\in K, v\in V: a\cdot(b\cdot v) = (a\cdot b)\cdot v$ \\
		\hspace{1cm}Normierung bzw. Nichttrivialität         & $\forall v \in V: 1 \cdot v = v$, wobei $1$ das Einselement des Körpers bezeichnet \\
        Verträglichkeit von Addition und Skalarmultiplikation \\
		\hspace{1cm}Distributivität bzgl. $(V,+)$            & $\forall a\in K, u,v\in V: a\cdot(u + v) = a\cdot u + a\cdot v$ \\
		\hspace{1cm}Distributivität bzgl. $(K,+)$            & $\forall a,b\in K, v\in V: (a + b)\cdot v = a\cdot v + b\cdot v$ \\
        \midrule
        \textbf{Axiome linearer Abbildungen}                  & \textbf{Bedeutung} \\
        \midrule
        Additivität & $\forall v_1,v_2\in V: f(v_1+v_2) = f(v_1)+f(v_2)$ \\
        Homogenität & $\forall \lambda\in K, v\in V: f(\lambda\cdot v) = \lambda \cdot f(v)$ \\
        \bottomrule
	\end{tabularx}
	\caption{Definierende Eigenschaften von Vektorräumen und linearen Abbildungen}
    \label{vektorraeume:def_table}
\end{table}

Sind $V,W$ zwei $K$-Vektorräume und $f: V\to W$ eine Abbildung, so heißt $f$ \emph{($K$-)lineare Abbildung} oder \emph{(Vektorraum-)Homomorphismus}, falls die beiden Axiome in Tabelle \ref{vektorraeume:def_table} erfüllt sind. Den Raum aller $K$-linearen Abbildungen von $V$ nach $W$ bezeichnen wir mit $\Hom_K(V,W)$. Eine $K$-linearen Abbildung von $V$ nach $V$ (also gleicher Definitions- und Zielraum) heißt \emph{(Vektorraum-)Endomorphismus}, der Raum aller solcher Abbildungen wird mit $\End_K(V)$ notiert.

Existiert ein Homomorphismus $f': W\to V$ mit $f\circ f'=f'\circ f=\id$, so nennt man $f$ \emph{Isomorphismus} der Vektorräume.
\end{definition}

\begin{remark}
Wir sind praktisch ausschließlich an $\IR$- und $\IC$-Vektorräumen interessiert in diesem Kurs.
\end{remark}

\begin{definition}[Kern \& Bild]
Ist $f: V\to W$ eine $K$-lineare Abbildung, so ist
\[\ker(f) := \Set{v\in V | f(v)=0}\]
der \emph{Kern von $f$} und
\[\im(f) := \Set{w\in W | \exists v\in V: f(v)=w}\]
das \emph{Bild von $f$}.
\end{definition}

\subsection{Bilineare Abbildungen}

\begin{remark}
	Es gibt i.A. keine Multiplikation zweier Vektoren miteinander in irgendeinem Sinne. Wir können immer Skalare mit Vektoren multiplizieren, aber nicht Vektoren mit Vektoren. Nichts desto trotz ist es \emph{manchmal} doch so, dass zusätzlich zu Addition und Skalarmultiplikation eine weitere Operation existiert, die ein sinnvolles Konzept von Multiplikation liefert, z.B.
	\begin{enumerate}
		\item Der Vektorraum der Polynome $\IR[X]$ hat die Polynommultiplikation, d.h.
		\[\left(\sum_{i=0}^n a_i X^i\right) \cdot \left(\sum_{j=0}^m b_j X^j\right) := \sum_{k=0}^{n+m} (\sum_{\substack{i,j \\ i+j=k}} a_i b_j) X^k\]
		\item Der Vektorraum der Funktionen $X\to\IC$ für einen festen Definitionsbereich $X$ hat die punktweise Multiplikation, d.h.
		\[f\cdot g := x\mapsto f(x)g(x)\]
		\item Die Hintereinanderausführung von linearen Abbildungen $\Hom_K(V,W) \times \Hom_K(U,V) \to \Hom_K(U,W), (f,g) \mapsto f\circ g$ ist eine Abbildung, die sich in vielerlei Hinsicht auch wie eine Multiplikation verhält. Für $U=V=W$ erhält man insbesondere eine Multiplikation $\End_K(V) \times \End_K(V) \to \End_K(V)$.
		
		Nach Wahl von je einer Basis können wir $\Hom_K(V,W)$, $\Hom_K(U,V)$ und $\Hom_K(U,W)$ mit $K^{n\times m}$, $K^{m\times p}$ und $K^{n\times p}$ identifizieren. Die Hintereinanderausführung von linearen Abbildung entspricht dann der Matrixmultiplikation.
		\item ...
	\end{enumerate}
	
\end{remark}

\begin{definition}[Bilineare \& multilineare Abbildungen]\label{bilineare_abb:def}
	Sind $V,W,X$ drei $K$-Vektorräume, so heißt eine Abbildung $\phi: V\times W\to X$ \emph{bilinear}, falls sie beiden Linearitätsbedingungen in Tabelle \ref{bilineare_abb:def_table} erfüllt, d.h. die Funktion ist separat linear, wenn man nur den ersten oder nur den zweiten Input variiert und den anderen festhält.
	
	\begin{table}[!ht]
		\setlength\extrarowheight{10pt} % for a bit of visual "breathing space"
		\hspace{-0.1\textwidth}
		\begin{tabularx}{1.2\textwidth}{p{6.5cm} p{9.5cm}}
			\toprule
			\textbf{Axiom}                  & \textbf{Bedeutung} \hspace{0.5cm} \\ 
			\midrule
			Linearität: \\
			\hspace{1cm}im ersten Argument  & $\forall v_1,v_2\in V, w\in W: \phi(v_1+v_2,w) = \phi(v_1,w) + \phi(v_2,w)$ \\
			& $\forall \lambda\in K, v\in W, w\in W: \phi(\lambda v, w) = \lambda \phi(v,w)$ \\
			\hspace{1cm}im zweiten Argument & $\forall v\in V, w_1, w_2\in W: \phi(v,w_1+w_2) = \phi(v,w_1) + \phi(v,w_2)$ \\
			& $\forall \lambda\in K, v\in W, w\in W: \phi(v, \lambda w) = \lambda \phi(v,w)$ \\
			Falls $\phi: V\times V \to X$, kann optional gelten: \\
			\hspace{1cm}Symmetrie bzw. Kommutativität & $\forall u,v\in V: \phi(u,v) = \phi(v,u)$ \\
			\hspace{1cm}Antisymmetrie & $\forall u,v\in V: \phi(u,v) = -\phi(v,u)$ \\
			Falls $\phi: V\times V\to V$, kann optional gelten: \\
			\hspace{1cm}Assoziativität & $\forall u,v,w\in V: \phi(u,\phi(v,w)) = \phi(\phi(u,v),w)$ \\
			\bottomrule
		\end{tabularx}
		\label{bilineare_abb:def_table}
		\caption{Eigenschaften von bilinearen Abbildungen}
	\end{table}
	
	Für Abbildungen $V_1\times V_2\times V_3\to X$, die von drei Inputvektoren abhängig sind, kann man entsprechend definieren, dass eine Abbildung \emph{trilinear} heißt, wenn sie die drei Distributivgesetze erfüllt. Für vier, fünf, ... $k$ Input-Vektoren spricht man entsprechend von \emph{$k$-fach linearen} Abbildungen.
\end{definition}

\begin{example}
	Die drei genannten \enquote{Multiplikationen} von oben sind bilinear. Polynommultiplikation und punktweise Multiplikation von Funktionen sind kommutativ und assoziativ. Die Komposition $\End_K(V)\times\End_K(V)\to\End_K(V)$ ist assoziativ, aber nicht kommutativ, falls $\dim(V) > 1$.
	
	Es gibt viele weitere, äußerst nützliche Beispiele.
	\begin{enumerate}[start=4]
		\item Richtungsableitungen sind bilinear in der Richtung und der Funktion:
		
		Sei $X\subseteq\IR^n$ ein geeigneter Definitionsbereich (z.B. eine offene Menge). Sei außerdem $f:X \to\IC$ eine differenzierbare Funktion. Dann existieren insbesondere alle Richtungsableitungen $(\partial_vf)(x_0) = \lim_{t\to 0} \frac{f(x_0)-f(x_0+tv)}{t}$ in allen Punkten $x_0\in X$. Die Abbildungsvorschrift $(v,f) \mapsto \partial_v f$ liefert eine bilineare Abbildung für mehrere Kombinationen von Vektorräumen, z.B. $\IR^n \times C^1(\IR^n) \to C^0(\IR^n)$ oder $\IR^n \times C^\infty(\IR^n) \to C^\infty(\IR^n)$.
		
		\item Allgemeiner: Lineare Differentialoperatoren:
		
		Für mehrfach differenzierbare Funktionen kann man natürlich auch mehrere Ableitungsschritte hintereinander ausführen. Auf diese Weise erhält man multilineare Abbildungen: $k$-faches Ableiten in $k$ Richtungen, also der Differentialoperator $\partial_{v_1} \partial_{v_2} \cdots \partial_{v_k}$ ist $k$-fach linear in den Vektoren $v_1, ..., v_k$ als Inputs. Die Anwendung auf eine Funktion ist entsprechend $(k+1)$-fach linear in den $k$ Richtungsvektoren und der Funktion als Inputs.
		
		Als $k$-lineare Abbildung ist dies eine symmetrische Abbildung, d.h. $\partial_v \partial w f = \partial_w\partial_v f$, sofern $f$ zweimal stetig differenzierbar ist. Das ist der Satz von Schwarz\footnote{Hermann Amandus Schwarz (1843 -- 1921), dt. Mathematiker.}.
		
		\item Kreuzprodukt:
		
		Es gibt eine besondere bilineare Abbildung $\textsc{x} : \IR^3 \times \IR^3 \to \IR^3$, genannt \udot{Kreuzprodukt}, die in anderen Dimensionen keine gute Analogie hat, nämlich
		
		\[\begin{pmatrix}x_1\\y_1\\z_1\end{pmatrix} \textsc{x} \begin{pmatrix}x_2\\y_2\\z_2\end{pmatrix} := \begin{pmatrix}
			y_1 z_2 - y_2 z_1 \\
			x_1 z_2 - x_2 z_1 \\
			x_1 y_2 - x_2 y_1
		\end{pmatrix}\]
		Es ist bilinear, aber weder kommutativ noch assoziativ. Stattdessen ist es antisymmetrisch.
	\end{enumerate}
\end{example}


\subsection{Tensoren}

\begin{remark}
Es gibt i.A. keine Multiplikation zweier Vektoren in irgendeinem Sinne. Wir können immer Skalare mit Vektoren multiplizieren, aber nicht Vektoren mit Vektoren. Nichts desto trotz ist es \emph{manchmal} doch so, dass ein sinnvolles Konzept von Multiplikation existiert für spezielle Vektorräume, z.B.
\begin{enumerate}
	\item Der Vektorraum der Polynome $\IR[X]$ hat die übliche Multiplikation,
	\item Der Vektorraum der Funktionen $X\to\IC$ für einen festen Definitionsbereich $X$ hat die punktweise Multiplikation,
	\item Die Hintereinanderausführung von linearen Abbildungen $\Hom_K(V,W) \times \Hom_K(U,V) \to \Hom_K(U,W), (f,g) \mapsto f\circ g$ ist eine Abbildung, die sich in vielerlei Hinsicht auch wie eine Multiplikation verhält
	\item 
\end{enumerate}

\end{remark}

\begin{definition}[Bilineare \& multilineare Abbildungen]
Sind $V,W,X$ drei $K$-Vektorräume, so heißt eine Abbildung $b: V\times W\to X$ \emph{bilinear}, falls sie die folgenden beiden Distributivgesetze erfüllt:
\begin{description}
	\item $\forall v,v'\in V, w\in W, \lambda,\lambda'\in K: b(\lambda v+ \lambda' v', w) = \lambda b(v,w) + \lambda' b(v',w)$
	\item $\forall v\in V,  w,w'\in W, \lambda,\lambda'\in K: b(v,\lambda w+\lambda'w') = \lambda b(v,w) + \lambda' b(v,w')$
\end{description}
Zusammengefasst: Die Funktion ist separat linear, wenn man nur den ersten oder nur den zweiten Input variiert und den anderen festhält.

Für Abbildungen $V_1\times V_2\times V_3\to X$, die von drei Inputvektoren abhängig sind, kann man entsprechend definieren, dass eine Abbildung \emph{trilinear} heißt, wenn sie die drei Distributivgesetze erfüllt.
\end{definition}

\begin{example}
Die drei genannten \enquote{Multiplikationen} sind bilinear. Es gibt viele weitere, äußerst nützliche Beispiele.
\begin{enumerate}[resume]
	\item Richtungsableitungen sind bilinear in der Richtung und der Funktion:
	
	 Sei $X\subseteq\IR^n$ ein geeigneter Definitionsbereich (z.B. eine offene Menge) und $x_0\in X$ ein Punkt. Sei außerdem $f:X \to\IC$ eine differenzierbare Funktion. Dann existieren insbesondere alle Richtungsableitungen $(\partial_vf)(x_0) = \lim_{t\to 0} \frac{f(x_0)-f(x_0+tv)}{t}$. Diese Operation ist bilinear in $v$ und $f$.
     
     \item Allgemeiner Differentialoperatoren:
     
     Für mehrfach differenzierbare Funktionen kann man natürlich auch mehrere Ableitungsschritte hintereinander ausführen. Zweifaches Ableitung in mehrere Richtungen $\partial_{v_1} \partial_{v_2}$ ist bilinear 
\end{enumerate}
\end{example}

The above definition has already hinted at an operation, that a tensor space may lack: A bilinear multiplication of two vectors $\cdot : V \times V \to V$. We can formally define the properties of such a binary operation:
%\begin{definition}[Bilinear multiplication in a vector space]
%	Let $F$ be a field, and let $V$ be a vector space over $F$. Denote with $\cdot$ an additional binary operation, $\cdot : V \times V \to V$ (i.e. if $\vc{v}$ and $\vc{w}$ are any two elements of $V$, $\vc{v}\cdot \vc{w}$ is the product of $\vc{v}$ and $\vc{w}$ and also lies in $V$). If this operation fullfills all properties in the following table for all elements $\vc{u},\vc{v},\vc{w}\in V$, $a,b,\in F$, then we call it a \emphName{bilinear multiplication} of two vectors. 
%	\begin{table}[!ht]
%		\setlength\extrarowheight{10pt} % for a bit of visual "breathing space"
%		\begin{tabularx}{\textwidth}{X X}
%			\toprule
%			\textbf{Axiom}             & \textbf{Bedeutung}                                                           \\ 
%			\midrule
%			Rechts-Distributivität       & $\begin{aligned} \phi((\vc{u} +\vc{v}),\vc{w}) &= \phi(\vc{u},\vc{w}) + \phi(\vc{v},\vc{w})\\ (\vc{u} +\vc{v})\cdot\vc{w} &= \vc{u}\cdot\vc{w} + \vc{v}\cdot\vc{w} \end{aligned}$       \\
%			Links-Distributivität        &$\begin{aligned} \phi(\vc{w},(\vc{u} +\vc{v})) &= \phi(\vc{w},\vc{u}) + \phi(\vc{w},\vc{v})\\ \vc{w}\cdot(\vc{u} +\vc{v}) &= \vc{w}\cdot\vc{u}\cdot + \vc{w}\cdot\vc{v} \end{aligned}$ \\
%			Assoziativität der Skalarmultiplikation &  $\begin{aligned} \phi((a\vc{w}),(b\vc{u})) &= (ab)\phi((\vc{w},\vc{u})) \\ (a\vc{w})\cdot(b\vc{u}) &= (ab)(\vc{w}\cdot\vc{u})\end{aligned}$   
%			\\              
%			Optional: Assoziativität der Multiplikation  & $\begin{aligned} \phi(\vc{v},\phi((\vc{w},\vc{u}))) = \phi(\phi(\vc{v} , \vc{w}), \vc{u})\\ \vc{v}\cdot(\vc{w}\cdot\vc{u}) = (\vc{v} \cdot \vc{w}) \cdot \vc{u}\end{aligned}$     
%			\\
%			\bottomrule
%		\end{tabularx}
%		\caption{Eigenschaften von einer bilinearen Abbildung $\phi$ bzw. bilinearen Multiplikation}
%	\end{table}

	\begin{table}[!ht]
	\setlength\extrarowheight{10pt} % for a bit of visual "breathing space"
	\hspace{-0.1\textwidth}\begin{tabularx}{1.2\textwidth}{>{\hsize=.8\hsize}X >{\hsize=1.1\hsize}X >{\hsize=1.1\hsize}X}
		\toprule
		\textbf{Axiom}             & \textbf{Bedeutung} \hspace{0.5cm} &        \textbf{Multiplikation}                                                   \\ 
		\midrule
		Rechts-Distributivität       & $ \phi((\vc{u} +\vc{v}),\vc{w}) = \phi(\vc{u},\vc{w}) + \phi(\vc{v},\vc{w})$ & $(\vc{u} +\vc{v})\cdot\vc{w} = \vc{u}\cdot\vc{w} + \vc{v}\cdot\vc{w} $       \\
		Links-Distributivität        &$ \phi(\vc{w},(\vc{u} +\vc{v})) = \phi(\vc{w},\vc{u}) + \phi(\vc{w},\vc{v}) $ & $ \vc{w}\cdot(\vc{u} +\vc{v}) = \vc{w}\cdot\vc{u}\cdot + \vc{w}\cdot\vc{v} $ \\
		Assoziativität der Skalarmultiplikation &  $ \phi((a\vc{w}),(b\vc{u})) = (ab)\phi((\vc{w},\vc{u})) $ & $ (a\vc{w})\cdot(b\vc{u}) = (ab)(\vc{w}\cdot\vc{u})$   
		\\              
		Optional: Assoziativität der Multiplikation  & $ \phi(\vc{v},\phi((\vc{w},\vc{u}))) = \phi(\phi(\vc{v} , \vc{w}), \vc{u}) $ & $ \vc{v}\cdot(\vc{w}\cdot\vc{u}) = (\vc{v} \cdot \vc{w}) \cdot \vc{u}$     
		\\
		\bottomrule
	\end{tabularx}
	\caption{Eigenschaften von einer bilinearen Abbildung $\phi$ bzw. bilinearen Multiplikation}
\end{table}


\end{definition}
Note, that commutativity is a not required property of a bilinear multiplication (hence the distinction between right and left distributivity). Depending on the situation, the multiplication may or may not have to fullfill the requirement of associativity.

Typical examples for vector spaces, that are equipped with a bilinear multiplication, include: $\IR^3$ with the cross product, $\IR^{n\times n}$ with the matrix product. A vector space that is typically not equipped with a bilinear multiplication is the space of polynomials up to degree $n$, since multiplying two polynomials can result in a polynomial of higher degree than $n$. However, we can extend this vector space to the space of all polynomials $P$. $P$ is easily equipped with a bilinear multiplication (defined as you would expect the multiplication of two polynomials to work), which even happens to be commutative.

The concept of a vector space equipped with a bilinear multiplication has its own name:
\begin{definition}[Algebra]
	Let $F$ be a field, and let $V$ be a vector space over $F$. If $V$ can be equipped with a bilinear multiplication, then $V$ is an \emphName{algebra} over $F$. 
\end{definition}
An algebra over $F$ is sometimes also called an $F$-algebra, and $F$ is called the base field of $V$. The binary multiplication is often simply referred to as multiplication in $V$. 

\subsubsection{The tensor product}
Let us now look at a new kind of multiplication $\otimes$ between two vectors $\vc{u},\vc{v} \in \IR^3$: The tensor product. There are many ways of motivating and defining it, but for our purposes we shall be satisfied by starting with a bilinear, associative, non-commutative multiplication of two vector spaces $\otimes: U \times V \to U \otimes V$ whose result is called a tensor and written as $\vc{U}\otimes\vc{V}$ (spoken "U tensor V").
\[\otimes: U \times V \to U \otimes V,\vc{u}\times\vc{v} \mapsto \vc{u}\otimes\vc{v}\]

$U\otimes V$ is in itself a vector space (the so called tensor space) of dimensionality 
\[\dim{U \otimes V} = \dim{U} \cdot \dim{V}.\]
Choosing basis sets for $U$ and $V$, $B_U = \{\vc{b}_i^u| i=1, \cdots, \dim{U}\}$ and $B_V=\{\vc{b}_i^v| i=1, \cdots, \dim{V}\}$ respectively, allows to immediately choose a basis for $U\otimes V$: 
\[\{\vc{b}_i^u\otimes \vc{b}_j^v| i=1, \cdots, \dim{U, j=1,\cdots,\dim{V}}\}.\]
Similar to denoting the coordinates of a vector $v\in V$ e.g. as $v_i$, we can express the coordinates of a tensor with respect to the above basis as $t_{ij}$.


Given a tensor 
\[t=\vc{u}\otimes\vc{v},\] 
one may multiply it with yet another vector $\vc{w}$, resulting in a product with more terms (i.e. vectors)
\[f=\vc{u}\otimes\vc{v}\otimes\vc{w}\] 
and so on. This is similar to the multiplication of polynomials. Indeed, just like polynomials have degrees, tensors can be of different order. The order of a tensor is determined by the number of vectors that are multiplied with each other, the tensor $t$ is of order 2, the tensor $f$ is of order 3. This nomenclature applies to tensor spaces as well, $V\otimes W$ is a tensor space of second order. We are practically exclusively interested in tensor spaces of k-th order of only one vector space $V$, i.e. 
\[\underbrace{V\otimes V \otimes \cdots \otimes V}_{k \text{ times}},\]
which can be denoted as $V^{\otimes k}$. For our purposes we choose $V=\IR^n$ with mostly $n=3$.

\begin{remark}[Tensor algebra]
	Much like the space of all polynomials is an algebra, the vector space of the sum of all tensor products of a vector space $V$,
	\[\mathcal{T}: \bigoplus_{i=0}^{\infty} V^{\otimes i},\]
	is equipped with the multiplication $\otimes$ and hence is an algebra.
\end{remark}

\begin{definition}[Elementary Tensor]
	An elementary tensor is a tensor that can be expressed as a single product of vectors, e.g. $v_1\otimes v_2 $, (as opposed to a sum of products, e.g. $v_1\otimes v_2 - v_2 \otimes v_1$). In the following definitions we will often use elementary tensors, the linear continuation for sums of elementary tensors is always implied.
\end{definition}

\begin{definition}[Casimir Element or $\delta_{ij}$-Tensor]
	Let $\Omega_V$ denote the symmetric, isotropic 2-tensor of the vector space $V$ of dimensionality $k$. With any complete orthonormal basis set $\{\vc{b}_1,\cdots,\vc{b}_k\}$ for $V$, $\Omega_V$ can be expressed as
	\begin{equation}
		\Omega_V = \sum_{i=1}^k \vc{b}_i \otimes \vc{b}_i
	\end{equation}
	Since we will mostly work with $V=\IR^3$ with some orthonormal basis set $\{\vc{b}_1,\vc{b}_2,\vc{b}_3\}$, the corresponding $\Omega_V$ shall be written as $\Omega$,
	\[\Omega := \Omega_{\IR^3} = \sum_{i=1}^3 \vc{b}_i \otimes \vc{b}_i.\]
\end{definition}
\begin{lemma}
	The Casimir element does not depend on the choice of the orthonormal basis for $V$, i.e. it is indeed isotropic.
\end{lemma}
\begin{proof}
	Take $B:=\left\{e_1, \cdots e_d\right\}$, $\tilde{B}:=\left\{\tilde{e_1}, \cdots \tilde{e_d}\right\}$ to be two orthonormal basis sets and the basis transformation $\phi:V\to V, v \mapsto \sum_{i=1}^{d}\braket{v,\tilde{e_i}}\tilde{e_i}$. Then $s_{ij}=\braket{e_j,\tilde{e_i}}$ are the entries of an orthogonal matrix and we have
	\begin{align*}
		\Omega_V &= \sum_{i=1}^d e_i \otimes e_i\\
		\implies \phi\left(\Omega_V\right) &= \phi\left(\sum_{i=1}^d e_i \otimes e_i\right)\\
		&= \sum_{i=1}^d \phi\left(e_i\right) \otimes \phi\left(e_i\right)\\
		&=\sum_{j=1}^d \left(\sum_{i=1}^d \braket{e_j,\tilde{e_i}}\tilde{e_i}\right) \otimes \left(\sum_{i=1}^d \braket{e_j,\tilde{e_i}}\tilde{e_i}\right) \\
		&=\sum_{i=1}^d\sum_{i'=1}^d\underbrace{\left(\sum_{j=1}^d \braket{e_j,\tilde{e_i}}\cdot\braket{e_j,\tilde{e_i}}\right)}_{=(SS^T)_{ii'}=1_{ii'}=\delta_{ii'}} \tilde{e_i}\otimes \tilde{e_i'} \\
		&=\sum_{i=1}^d\tilde{e_i}\otimes \tilde{e_i} = \Omega_V
	\end{align*}
\end{proof}

\subsection{The notion of meaningfull linear operations with tensors - light introduction to invariant theory}
In the context of our physical world, we can use tensors to measure quantities. A possibly familiar example is stress within a material or gas, which is measured with a tensor of second order. 
The fundamental property we expect from all quantities that we measure, is that they are independent of the orientation of the orthonormal coordinate system - rotating, inverting or mirroring the basis vectors should not change the value of the properties we are interested in (though this may change the numbers with which we express that quantity).
A velocity vector for example should always point in the same direction, regardless of wether its coordinates are $(3,4,0)^T$ or $(0,0,5)^T$. The same goes for its length or the determinant of a matrix.

%The first question we need to answer is: How does a length preserving change of the coordinate system such as a rotation act on a tensor? 
\begin{definition}[Orthogonal Group $O^3$]
	Let $O^3$ be the set of all orthogonal matrices \[O^3:=\{r\in\IR^{3\times 3}: |\det{r}| = 1 \}.\] 
	All of the operations described above can be described with a corresponding matrix in $O^3$. Furthermore, equipped with the typical matrix-matrix multiplication, $O^3$ fullfills all requirements of a group - hence the name \emphName{orthogonal group}. $O^3$ has infinitely many elements, but is still compact - allowing for the notion of taking an average over the whole group.
\end{definition}

\begin{definition}[Action of $O^3$ on the tensor space]
	An element $r\in O^3$ acts on a vector $v\in\IR^3$ via the matrix-vector product $r\cdot v$, resulting in the rotation or mirroring of the given vector. This is extended to tensors - An element $r\in O^3$ acts on a tensor in $\mathcal{T}$ through the matrix-vector multiplication of each factor of that tensor:
	\begin{align*}
		r(\cdot):\mathcal{T} &\to \mathcal{T}\\
		v_1 \otimes \cdots \otimes v_n&\mapsto r(v_1 \otimes \cdots \otimes v_n) :=r\cdot v_1 \otimes \cdots \otimes r\cdot v_n.
	\end{align*}
\end{definition}

\begin{definition}[Orthogonal Group $O^2_{\vc{n}}$ as subgroup of $O^3$]
	Let $\vc{n}\in \IR^3$. $O^2_{\vc{n}}$ is the subgroup of $O^3$, that acts like the identity on $\vc{n}$,  \[O^2_{\vc{n}}:=\{r\in O^3 | r(\vc{n}) = \vc{n} \}.\] 
	$O^2_{\vc{n}}$  is the embedding of $\IR^2$ and $O^2:= \{r\in\IR^{2\times 2}: |\det r| = 1\}$ as a plane $E$ into $\IR^3$ with $E\perp\vc{n}$.
\end{definition}


\begin{definition}[$O^3$-Invariant linear map of the tensor space]
	A linear map $f: \mathcal{T}\to \mathcal{T}$ is called \emphName{compatible (or invariant) with respect to $O^3$}, iff for all actions $r\in O^3$  and for any tensor $\ten{T}\in\mathcal{T}$ the relationship
	\[f \circ r (\ten{T}) = r \circ f (\ten{T})\]
	holds. This precisely expresses the above notion of independency (or invariancy) of a quantity or tensor of the choice of coordinate system.
\end{definition}
Finding $O^3$-invariant linear maps and being sure, that one has found all possibilities, requires some very interesting and complicated piece of mathematics known as \emphName{invariant theory}. This goes far beyond the scope of this thesis, we will restrict ourselves to introducing and applying the results. The interested reader may start by inquiring for the \emphName{Brauer algebra}. 

\subsubsection{Operations on one tensor}\label{chap:tensorOperations}
Below is a list of all types of $O^3$-invariant linear maps between tensors, defined with the elementary tensor $v_1 \otimes \cdots \otimes v_n$. All combinations of them is the full set of of $O^3$-invariant linear maps between tensors:
\begin{description}
	\item[Multiplication with a scalar $s\in\IR$.] This does not change the degree of the tensor.
	
	\item[Permutation of the factors of a tensor.] We have seen this in tensors of second degree, where we called it transpose in correspondence to matrices. Formally, $\sigma \in S_n$ acts on a tensor $v_1 \otimes \cdots \otimes v_n$:
	\[\sigma(\cdot): \mathcal{T}\to \mathcal{T}, v_1 \otimes \cdots \otimes v_n \mapsto v_{\sigma^{-1}(1)} \otimes \cdots \otimes v_{\sigma^{-1}(n)}\]
	This does not change the degree of the tensor or any of the vectors $v_i$, it only reorders them. When using index notation, this operation is represented by a permutation of indices.
	
	\item[Taking the trace $tr$ between two factors of a tensor.] We can pick two vectors of a tensor and replace them with their scalar product:
	\[tr_{i;j} (\cdot): \mathcal{T} \to \mathcal{T}, v_1 \otimes \cdots \otimes v_i \otimes \cdots \otimes v_j \otimes \cdots \otimes v_n \mapsto \braket{v_i,v_j} v_1 \otimes \cdots \otimes v_{i-1}\otimes v_{i+1} \otimes \cdots \otimes v_{j-1} \otimes v_{j+1} \otimes \cdots \otimes v_n\]
	This reduces the degree of the tensor by $2$. When using index notation, this operation is represented by the sum over two identical indices, which is often abbreviated with Einstein's sum convention.
	
	\item[Tensor multiplication with the Casimir element $\Omega$.]
	\[\ten{\Omega} (\cdot):  \mathcal{T}\to \mathcal{T}, v_1 \otimes \cdots \otimes v_n \mapsto \Omega \otimes v_{1} \otimes \cdots \otimes v_{n} \]
	This increases the degree of the tensor by $2$. When using index notation, this operation is represented by the multiplication with $\delta_{ij}$.
\end{description}

\subsubsection{Operations between two tensors}
Given two tensors $u,v \in \mathcal{T}$, the question of finding invariant linear maps with respect to $O^3$ $\mathcal{T} \times \mathcal{T} \to \mathcal{T}$ is easily reduced to the already described case of operations of one tensor. We just have to identify all possible maps $\mathcal{T} \otimes \mathcal{T} \to \mathcal{T} = \mathcal{T} \to \mathcal{T}$ due to the isomorphism $\mathcal{T} \times \mathcal{T} \to \mathcal{T} \cong \mathcal{T} \otimes \mathcal{T} \to \mathcal{T}$ for bilinear and linear maps. To be more specific: Given two tensors $u,v$, take the tensor product $u\otimes v$ and now aply any combination of the operations described for one tensor. Nevertheless, some operations are very common and worth mentioning.

\begin{definition}[Contraction between two tensors]
	Taking one or multiple traces between two tensors is given a special name: Contraction.
	\begin{align*}
		tr_{i;j} (\cdot,\cdot): \mathcal{T} &\times \mathcal{T} \to \mathcal{T},
		\\
		v_1 \otimes \cdots \otimes v_i \otimes \cdots\otimes v_n &\times u_1 \otimes \cdots \otimes u_i \otimes u_j \otimes \cdots \otimes u_n 
		\\
		\mapsto \braket{v_i,u_j} v_1 \otimes \cdots \otimes v_{i-1}\otimes v_{i+1} \otimes \cdots \otimes v_n &\otimes u_1 \otimes u_{j-1} \otimes u_{j+1} \otimes \cdots \otimes u_n
	\end{align*}
\end{definition}
\begin{definition}[Standard scalar product]
	The standard scalar product $\braket{\cdot,\cdot}$ between two tensors of same order is defined as the (ordered) full contraction between the two tensors:
	\begin{align*}
		\braket{\cdot,\cdot} &: \  V^{\otimes n} \times V^{\otimes n} \to \IR,\\
		\braket{u_1 \otimes \cdots \otimes u_n,v_1 \otimes \cdots \otimes v_n} &\mapsto tr_{1,\cdots,n;1,\cdots,n} (u_1 \otimes \cdots \otimes u_n,v_1 \otimes \cdots \otimes v_n) \\
		&= \braket{u_1,v_1}\cdot\braket{u_2,v_2} \cdots \braket{u_n,v_n}.
	\end{align*}
\end{definition}
\subsection{Symmetric tensor space}
\begin{definition}[Symmetric tensors]
	A tensor $v$ of $n$-th order is called symmetric, iff for all permutations $\sigma \in S_n$ holds:
	\[\sigma(v) = v.\]
\end{definition}
The space spanned by all symmetric tensors of degree $n$, $\Sym{n}$, is an invariant subspace of the tensor space due to the completely decoupled nature of the actions of $S_n$ and $O^3$ on a tensor. We can take any tensor and project it onto this subspace by taking the average of all possible permutations:
\begin{align*}
	q: {\IR^3}^{\otimes n} &\to \Sym{n}\\
	q\left(v_1 \otimes \cdots  \otimes v_n\right) &\mapsto \frac{1}{n!} \sum_{\sigma \in S_n} v_{\sigma(1)} \otimes \cdots \otimes v_{\sigma(n)}
\end{align*}
and linear extension. In index notation this is denoted by round brackets surrounding the indices. $\Sym{n}$ is of dimension $\frac{1}{2}(n+1)(n+2)$.

\begin{note}[]
	It is easy to show for the standard scalar product on $\IR^{\otimes n}$, that for all $v,u \in \IR^{\otimes n}$ the following identity holds:
	\begin{equation}
		\braket{q(v),q(u)} = \braket{v,q(u)} = \braket{q(v),u}.
		\label{eq:scalarqprod}
	\end{equation}
\end{note}
\begin{definition}[Trace and contraction of symmetric tensors]
	Since for symmetric tensors all the positions $i,j (i\neq j)$ of the two vectors that are used to take the trace $tr_{i;j}$ (contraction) are equivalent to each other, we can use a notation that only shows the relevant information: The \emphProp{number} of traces (contractions) taken of one tensor (between two tensors). \\
	$k$-fold trace of a symmetric tensor:
	\begin{align*}
		tr^{k} (\cdot): \Sym{n} &\to \Sym{n-2k},\\
		tr^{k} (s) & \mapsto tr_{1,3,5\cdots, 2k-1;2,4,6,\cdots,2k} (s)
	\end{align*}
	$k$-fold contraction between two symmetric tensors:
	\begin{align*}
		tr^{k} (\cdot,\cdot): \Sym{n}\times\Sym{m} &\to \Sym{n-k}\otimes\Sym{m-k},\\
		tr^{k} (s,t) & \mapsto tr_{1,2,3\cdots, k;1,2,3\cdots,k} (s,t)
	\end{align*}		
\end{definition}

Due to the polynomial nature of the ansatz for the distribution function, we will almost exclusively work with symmetric tensor spaces (reference chapter \ref{chap:isomorphismPolyTensor} to see why). Casually spoken, $O^3$ imposes a specific structure on $\Sym{n}$, which brings us to the next section.

\subsection{Level 1 invariant theory: Irreducable and invariant subspaces of the symmetric tensor space}
\begin{definition}[Invariant subspaces] A subspace $U$ of any vector space $V$ is called \emphName{invariant} with respect to the action of a group $G$, or $G$-invariant, iff for all $r\in G$ and for all $u\in U$, the action of $r$ on $u$ still is in $U$, $r(u)\in U$ or equivilantly: $\forall r\in G: r(U)=U$.	
\end{definition}

\begin{example}[$O^2_{\vc{n}}$-invariant subspaces of $\IR^3$]
	We pick $\IR^3$ as vector space and the subgroup $O^2_{\vc{n}} \subset O^3$ along with a direction $n\in\IR^3$, for which by definition $\forall r \in O^2_{\vc{n}}: r(n)=n$.
	Then the subspace $\spn{\vc{n}}= \{\alpha \vc{n}|\alpha\in\IR\}$ is a one-dimensional invariant subspace, since 
	\[\forall r \in O^2_{\vc{n}}, \forall \alpha\in\IR: r(\alpha \vc{n}) = \alpha r(\vc{n}) = \alpha\vc{n}.\]
	Given an $O^3$-compatible scalar product, we can define the plane $E\subset\IR^3$ orthogonal to $\vc{n}$ and decompose $\IR^3$ into \[\IR^3 = \spn{\vc{n}}\bigoplus E.\] It is easily shown that $E$ is also an invariant, 2-dimensional subspace under the action of $O^2_{\vc{n}}$.
	
	% a general vector $\vc{v}\in\IR^3$ into \[\vc{v}=\underbrace{\vc{v}-\braket{\vc{v},\vc{n}}\vc{n}}_{\in E} + \braket{\vc{v},\vc{n}}\vc{n} =: \vc{v}_{E} + \vc{v}_{\vc{n}}\]
\end{example}

\begin{example}[$O^3$-invariant subspaces of the tensor space]
	These are subspaces $U\subset\mathcal{T}$ that fullfill $\forall r\in O^3: r(U)=U$. One such subspace is the above described space of symmetric tensors, which is of relatively large dimensionality for high $n$. Another example of an invariant subspace would be $\spn{\Omega}\subset\Sym{2}$ with a dimensionality of $1$.
	One may wonder, if $\Sym{n}$ can be further decomposed into "smallest", invariant subspaces, and if such a decomposition is unique.
\end{example}
\begin{definition}[Irreducable (and invariant) subspaces] 
	A subspace $U$ of any vector space $V$ is called \emphName{irreducable} with respect to the action of a group $G$, or \emphName{$G$-irreducable}, iff it is invariant and there is no invariant subspace $W\subset U$ with the properties $W\neq U$ and $W\neq 0$. If a subspace $U$ is invariant and $\dim U=1$, then it is also irreducable, however an irreducable subspace can be of dimension larger than $1$.
	
	Two irreducable subspaces $U_1, U_2, U_1 \neq U_2$ of a vector space $V$ are orthogonal to each other with respect to any scalar product of $V$ that is compatible with the action of the group $G$. We can project from the full vector space onto an irreducable subspace. Due to the universal orthogonality, such a projection is always unique up to a constant (or a zero).
	
\end{definition}
\begin{example}[$O^2_{\vc{n}}$-irreducable subspaces of $\IR^3$]
	We already know that $E$ and $\spn{\vc{n}}$ are invariant subspaces. Since $\dim{\spn{\vc{n}}}=1$, $\spn{\vc{n}}$ is also an irreducable subspace.
	For the 2-dimensional $E$ we cannot find a lower-dimensional invariant subspace, since \[\exists r\in O^2_{\vc{n}}: \forall \vc{v}\neq 0 \in E, \forall \alpha\in\IR: r(\vc{v})\neq \alpha \vc{v}.\] This means that $E$ is irreducable as well and that we have found all $O^2_{\vc{n}}$-irreducable subspaces of $\IR^3$.
\end{example}
\begin{example}[$O^2_{\vc{n}}$-irreducable subspaces of $\Sym{n}$]
	We note:
	\[\Sym{n} = q\left(\left(\IR^3\right)^{\otimes n}\right).\]
	Insert the splitting of $\IR^3$ into $E$ and $\spn{\vc{n}}$:
	\begin{align*}
		\Sym{n} &= q\left(\left(E\bigoplus\spn{\vc{n}}\right)^{\otimes n}\right) \\
		&=\bigoplus_{i=0}^n q\left(E^{\otimes i}\otimes \left(\spn{\vc{n}}\right)^{\otimes n-i}\right)
	\end{align*}
	The orthogonality between $E$ and $\spn{\vc{n}}$ leads to an orthogonality between each of these tensor spaces, the invariancy of each $q\left(E^{\otimes i}\otimes \left(\spn{\vc{n}}\right)^{\otimes n-i}\right)$ follows from the invariancy of the "building blocks" $E$ and $\spn{\vc{n}}$. However, for general $n,i$ these subspaces are not irreducable. For instance we find that the invariant tensor $q(\Omega_E^{\otimes j})$ spans a 1-dimensional (and therefore irreducable) subspace  of $q(E^{\otimes 2j})$. In general, since $q\left(E^{\otimes i}\otimes \left(\spn{\vc{n}}\right)^{\otimes n-i}\right)\cong q(E^{\otimes i})\cong\sym{i}$, finding all irreducable subspaces of $\sym{i}$ immediately leads to all irreducable subspaces of $q\left(E^{\otimes i}\otimes \left(\spn{\vc{n}}\right)^{\otimes n-i}\right)$. 
\end{example}
\begin{remark}[Invariant tensors]
	Sometimes one may work with a tensor $\ten{T}$ that is invariant with respect to a group $G$, i.e. $\forall r\in G: r(T) = T$. For example:
	\begin{itemize}
		\item $G=S_n$. Then all tensors $\ten{T}\in\Sym{n}$ are invariant
		\item $G=O^3$. Then all tensors $\ten{T}\in(\IR^3)^{\otimes 2i}$ that are a linear combination of permutations $\Omega^{\otimes i}$ are invariant. If in addition $\ten{T}\in\Sym{2i}$: $\ten{T}=\alpha q\left(\Omega^{\otimes i}\right), \alpha\in\IR$.
		\item $G=O^2_{\vc{n}}$ and $\ten{T}\in\Sym{n}$. Not all tensors in $q\left(E^{\otimes m}\right)$ are invariant, in fact the only invariant tensors are $\alpha q\left(\Omega_E^{\otimes i}\right), \alpha\in\IR\implies m=2i$. All tensors in $\spn{\vc{n}}$ are invariant, so an $O^2_{\vc{n}}$-invariant, symmetric tensor must come from the subspace spanned by $\left\{q\left(\Omega_E^{\otimes i}\otimes \vc{n}^{\otimes n-2i}\right)\right\}$, i.e.
		\begin{equation}
			\ten{T}=\sum_{i=0}^{\frac{n}{2}} \alpha_i q\left(\Omega_E^{\otimes i}\vc{n}^{\otimes n-2i}\right), \alpha_i \in \IR.
			\label{eq:O2invariantTensors}
		\end{equation}
	\end{itemize}
	Notice, that not all tensors from an invariant subspace are themselves invariant. Even that a tensor lies in an irreducable subspace is no guarantee that the tensor itself is invariant. 
\end{remark}

\begin{example}[$O^3$-irreducable subspaces of $\Sym{n}$]
	It turns out, that $\Sym{n}$ can be expressed as the direct sum of $\left\lfloor\frac{n}{2}\right\rfloor+1$ $O^3$-irreducable subspaces (i.e. $\Sym{n}$ is \emphName{semi-simple}). Most of these subspaces are of dimensionality greater than $1$. Furthermore, these irreducable subspaces are \emphProp{pairwise orthogonal} with respect to \emphProp{any} scalar product of the $\Sym{n}$ that is compatible with $O^3$ (which is a rather redundant requirement, since we wouldn't want to choose a scalar product that is not compatible with $O^3$ anyway).
	
	For every order $n$, we find that the largest irreducable subspace of $\Sym{n}$ consists of trace-free tensors.
	\begin{definition}[Symmetric trace-free tensors]
		A symmetric tensor $v$ of order $n>1$ is called symmetric trace-free, iff $tr_{1,2}(v)=0$. Tensors of order 1 are always symmetric tracefree.
		The space spanned by all symmetric trace-free tensors of order $n$ is denoted as $\STF{n}$ and has dimension $2n+1$. On $\STF{n}$ all $O^3$-compatible scalar products of the tensor space are equivalent to each other (they give the same result up to a constant factor). That has the effect that an orthogonal basis of $\STF{n}$ chosen with respect to one scalar product is orthogonal with respect to any other scalar product (as long as both are $O^3$-compatible). In index notation, the projection onto the space of symmetric trace-free tensor is denoted as square brackets around the indices.
	\end{definition}
	The other irreducable subspaces are related to symmetric tracefree tensors:
	\begin{definition}[Symmetric traceable tensors]
		$\ST{n,a}$ is the space of symmetric tensors of order $n+2a$, that result in a trace-free tensor (of order $n$) after taking $i$ traces. Equivilantly: $\ST{n,a}$ is the space of symmetric tensors of order $n+2a$, that can be expressed as the result of taking trace-free tensors (of order $n$), and applying the linear map
		\[q\circ \underbrace{\ten{\Omega}\circ \cdots \circ \ten{\Omega}}_{i \text{ times}}.\]
		This means that there is a bijective map between $\ST{n,i}$ and $\STF{n}$, i.e. $\ST{n,i}$ and $\STF{n}$ are isomorph to each other.
	\end{definition}
	
	\begin{lemma}[Decomposition of $\Sym{n}$ into its $O^3$ irreducable subspaces]
		Each $\STF{i}$ is a building block for subspaces of $\Sym{n}, i\leq n$,
		\[\Sym{n} = \bigoplus_{i=0}^{\lfloor\frac{n}{2}\rfloor} q\left(\Omega^{\otimes i}\otimes \STF{n-2i}\right)=\bigoplus_{i=0}^{\lfloor\frac{n}{2}\rfloor}\ST{n-2i,i}\]
		
	\end{lemma}
	
	When building an orthogonal basis for $\Sym{n}$ for a physical problem (i.e. $O^3$ acts on $\Sym{n}$), the natural choice is the composition of all orthonormal basis sets from each of the subspaces. 
	
\end{example}

\subsection{Level 2 invariant theory: Linear and Bilinear Maps between Irreducable, invariant subspaces of the symmetric tensor space}
\begin{lemma}[Compatible linear Maps between irreducable subspaces (Schur's Lemma)]
	Let $U_1$ and $U_2$ be two invariant, irreducable subspaces of a vector space. A compatible linear map $\phi: U_1 \to U_2$ is only non-zero, iff $U_1$ and $U_2$ are isomorph to each other. All non-zero linear maps $U_1 \to U_2$ are the same up to a constant.
\end{lemma}
Applied to the space of all symmetric tensors, we find non-zero, $O^3$-compatible linear maps only of the form $\ST{n,a}\to\ST{n,b}, n,a,b \in \IN_0$.
\begin{lemma}[$O^3$-compatible, bilinear maps between symmetric trace free tensors]
	An $O^3$-compatible bilinear map $\phi^{n\tilde{n}\hat{n}}: \STF{\tilde{n}}\times\STF{\hat{n}}\to\STF{n}$ can only be non-zero, iff the following conditions apply:
	\begin{align*}
		\tilde{n}+\hat{n} + n \text{ even}\\
		|\tilde{n}-\hat{n}|\leq n\leq \tilde{n}+\hat{n}
	\end{align*}
	Then $\phi^{n\tilde{n}\hat{n}}$ is unique (up to a constant).
\end{lemma}
\begin{proof}
	$\phi^{n\tilde{n}\hat{n}}$ is isomorph to an $O^3$-compatible linear map $\varphi^{n\tilde{n}\hat{n}}: \STF{\tilde{n}}\otimes\STF{\hat{n}}\to\STF{n}$. $\varphi$ can only be non-zero, iff there exists a subspace of $\STF{\tilde{n}}\otimes\STF{\hat{n}}$, that is isomorph to $\STF{n}$, so we are looking for a projection of $\STF{\tilde{n}}\otimes\STF{\hat{n}}$ onto $\ST{n,i}$ with condition \[n+2i=\tilde{n}+\hat{n} \iff \tilde{n}+\hat{n} + n \text{ even}.\]
	Such a projection is always linear, unique, and compatible with $O^3$. Due to the isomorphism between $\STF{n}$ and $\STF{n,i}$, this makes $\varphi$ unique up to a constant, if it exists. 
	
	Let $s$ be any tensor of degree $n$, and $t$ be the resulting tensor from the projection of $t$ onto $\STF{n}$. Projecting a tensor of the form $\Omega \otimes s$ (or some permutation) onto $\STF{n+2}$ is equivalent to projecting $t$ onto $\STF{n+2}$. Since the only linear map from $\STF{n}$ to $\STF{n+2}$ is the zero-map, the projection of $\Omega \otimes s$ onto $\STF{n+2}$ will always result in $0$. This imposes the condition
	\[n\leq \tilde{n}+\hat{n}\]
	onto $n, \tilde{n},$ and $\hat{n}$.
	
	In chapter \ref{chap:tensorOperations} we listed all types of linear, $O^3$-compatible maps, $\varphi^{n\tilde{n}\hat{n}}$ must be a combination of these. The only map that decreases the order of a given tensor is taking the trace of that tensor. For a tensor of the specific form $\STF{\tilde{n}}\otimes\STF{\hat{n}}$, taking the trace within the first $\tilde{n}$ or the last $\hat{n}$ factors will always be zero. The only traces that yield a non-zero result are those taken between the first $\tilde{n}$ and the last $\hat{n}$ factors, i.e. $k$-fold contractions between the two arguments of $\phi$. Since the two arguments are each symmetric, all $k$-fold contractions between them are equivalent. The number of traces $k$ with a non-zero result we can take of a tensor from the space $\STF{\tilde{n}}\otimes\STF{\hat{n}}$ is limited by $\tilde{n}$ and $\hat{n}$, $k\leq \min{\tilde{n},\hat{n}}$. Since $n=\tilde{n}+\hat{n}-2k$, that imposes the condition on $n$:
	\[n \geq \tilde{n}+\hat{n}-2\min(\tilde{n},\hat{n}) = \max(\tilde{n}-\hat{n},\hat{n}-\tilde{n}) = |\tilde{n}-\hat{n}|.\]
	
	%5For all $n$ fullfilling the above conditions, $\ST{n,i}$ with $i=\frac{\tilde{n}+\hat{n}-n}{2}$ is one of the invariant, irreducable subspaces of $\STF{\tilde{n}}\otimes\STF{\hat{n}}$, i.e. $\varphi$ is a projection (up to a constant) onto $\ST{n,i}$ (or 0), making $$.
\end{proof}
\begin{lemma}[$O^3$-compatible, bilinear maps between symmetric tensors]\label{lem:bilinearMapDecomp}
	Every $O^3$-compatible bilinear map $\psi: \Sym{\tilde{m}}\times\Sym{\hat{m}}\to\Sym{m}$ with $m,\tilde{m},\hat{m}\in\IN_0$  can be expressed as the sum of $O^3$-compatible bilinear maps between all combinations of invariant, irreducable subspaces of $\Sym{\tilde{m}}$, $\Sym{\hat{m}}$, and $\Sym{m}$. With
	\begin{align*}
		m&=n+2a, & \tilde{m}&=\tilde{n}+2b, & \hat{m} &= \hat{n} +2c & 	\tilde{n},b,\hat{n},c,n,a &\in \IN_0
	\end{align*}
	we can write
	\begin{align*}
		\psi: &\Sym{\tilde{m}}\times\Sym{\hat{m}}\to\Sym{m}
		\\
		& = \bigoplus_{b=0}^{\lfloor\frac{\tilde{m}}{2}\rfloor}\ST{\tilde{n},b} \times \bigoplus_{c=0}^{\lfloor\frac{\hat{m}}{2}\rfloor}\ST{\hat{n},c} \to \bigoplus_{a=0}^{\lfloor\frac{m}{2}\rfloor}\ST{n,a}
		\\
		&=
		\bigoplus_{b=0}^{\lfloor\frac{\tilde{m}}{2}\rfloor} \bigoplus_{c=0}^{\lfloor\frac{\hat{m}}{2}\rfloor}
		\bigoplus_{a=0}^{\lfloor\frac{m}{2}\rfloor}
		\ST{\tilde{n},b} \times \ST{\hat{n},c} \to \ST{n,a}\\
		&=:\bigoplus_{b=0}^{\lfloor\frac{\tilde{m}}{2}\rfloor} \bigoplus_{c=0}^{\lfloor\frac{\hat{m}}{2}\rfloor}
		\bigoplus_{a=0}^{\lfloor\frac{m}{2}\rfloor} \psi_{abc}^{n\tilde{n}\hat{n}}
	\end{align*}
	Each of the $\psi_{abc}^{n\tilde{n}\hat{n}}$ are unique up to a constant and only non-zero, iff the following conditions apply:
	\begin{align*}
		\tilde{n}+\hat{n} + n \text{ even}\\
		|\tilde{n}-\hat{n}|\leq n\leq \tilde{n}+\hat{n}.
	\end{align*}
	Given a set of all maps $\psi_{abc}^{n\tilde{n}\hat{n}}$, two $\psi$ can only differ in the constants multiplied with those maps. Furthermore, due to the isomorphism between $\STF{n}$ and $\ST{n,i}$, all $\psi_{abc}^{n\tilde{n}\hat{n}}$  are isomorph to a map $Q^{n\tilde{n}\hat{n}}: \STF{\tilde{n}}\otimes\STF{\hat{n}}\to\STF{n}$, which is unique up to a constant. 
	Given a set of all maps $Q^{n\tilde{n}\hat{n}}$, two tensors of the form $U:=q\left(\Omega^{\otimes b}\otimes u\right)$ and $V:=q\left(\Omega^{\otimes b}\otimes v\right)$ with $u\in\STF{\tilde{n}},v\in\STF{\hat{n}}$, we can express any $\psi_{abc}^{n\tilde{n}\hat{n}}(U,V)$ as
	
	%	Given a set of all maps $Q^{n\tilde{n}\hat{n}}$, we can express any $\psi_{abc}^{n\tilde{n}\hat{n}}$ as
	%	\begin{align*}
	%	\psi_{abc}^{n\tilde{n}\hat{n}}(u,v) = S_{abc}^{n\tilde{n}\hat{n}} q\circ\underbrace{\ten{\Omega}\circ\cdots\circ\ten{\Omega}}_{a \text{ times}}\circ Q^{n\tilde{n}\hat{n}}(tr^b(u),tr^c(v)), S_{abc}^{n\tilde{n}\hat{n}} \in \IR.
	%	\end{align*}
	%	Given two tensors of the form $q\left(\Omega^{\otimes b}\otimes u\right)$ and $q\left(\Omega^{\otimes b}\otimes v\right)$ with $u\in\STF{\tilde{n}},v\in\STF{\hat{n}}$, this simplifies to:
	
	\begin{align}
		\psi_{abc}^{n\tilde{n}\hat{n}}\left(q\left(\Omega^{\otimes b}\otimes u\right),q\left(\Omega^{\otimes b}\otimes v\right)\right) = S_{abc}^{n\tilde{n}\hat{n}} \cdot q\left(\Omega^{\otimes a}\otimes Q^{n\tilde{n}\hat{n}}(u,v)\right).
		\label{eq:bilinearMapDecomp}
	\end{align}
	
	\textbf{The only degree of freedom for $O^3$-compatible, bilinear maps $\psi$ between symmetric tensors lies in the choice for the constants $S_{abc}^{n\tilde{n}\hat{n}}$.}
\end{lemma}

% !TeX root = spherical_harmonics.tex
% !TeX spellcheck = de_DE

\begin{sheet}

\begin{problem}[title={Casimir-Elemente von euklidischen Räumen}]\label{tensoren:ex:casimir_wohldef}
Es sei $V$ ein endlich-dimensionaler $\IR$-Vektorraum mit Skalarprodukt und $e_1, ..., e_n$ eine Orthonormalbasis.

\begin{subproblem}
Zeige, dass der Casimir-Tensor
\[\Omega_V := \sum_{i=1}^n e_i\otimes e_i\]
unabhängig von der Basiswahl ist, d.h. wenn $e_1', ..., e_n'$ eine weitere Orthonormalbasis von $V$ ist, dann gilt:
\[\sum_{i=1}^n e_i\otimes e_i = \sum_{i=1}^n e_i'\otimes e_i'\]

Hinweis: Orthogonale Matrizen.
\end{subproblem}

\begin{subproblem}
Zeige, dass $\Omega_V$ \enquote{isotrop} ist, d.h. für alle Isometrien $\rho: V\to V$ gilt: Die lineare Fortsetzung $\rho^{\otimes 2}$ von $v_1\otimes v_2 \mapsto \rho(v_1)\otimes\rho(v_2)$ erfüllt $\rho^{\otimes 2}(\Omega_V)=\Omega_V$.
Hinweis: Benutze a.
\end{subproblem}
\end{problem}

\begin{problem}[title={Casimir-Elemente allgemein}]
Es sei $V$ ein endlich-dimensionaler $K$-Vektorraum, $b_1, ..., b_n$ eine beliebige Basis von $V$ und $b_1^\ast, ..., b_n^\ast$ die dazu passende duale Basis von $V^\ast$.

\begin{subproblem}
Zeige, dass
\[\Omega := \sum_{i=1}^n b_i \otimes b_i^\ast \in V\otimes V^\ast\]
unabhängig von der Basiswahl ist.

Hinweis: Wenn $A$ eine Basiswechselmatrix zwischen zwei Basen von $V$ ist, wie sieht dann die Basiswechselmatrix der beiden dazugehörigen dualen Basen von $V^\ast$ aus?
\end{subproblem}
\begin{subproblem}
Wie entspricht das dem Casimir-Element euklidischer Räume?
\begin{enumerate}[label=\roman*.)]
\item Zeige zunächst, dass die Abbildung $V\to V^\ast, v\mapsto \braket{v,-}$ ein Isomorphismus $V\to V^\ast$ ist.
\item Was tut diese Abbildung mit einer Orthonormalbasis?
\end{enumerate}
\end{subproblem}
\end{problem}

\begin{problem}[title={Aber Tensoren sind doch so Buchstaben mit Indizes}]
	\label{ex:TensorenMitIndizes}
	Häufig wird einem von Physikern oder Ingenieuren ein Tensor lediglich als ein Buchstabe mit Indizes untergejubelt - z.B. der Spannungstensor $\sigma_{ij}$. Wir wollen verstehen, wie der Zusammenhang mit unserer Definition ist.

	Es sei $V$ ein endlich-dimensionaler $\IR$-Vektorraum mit Skalarprodukt und $e_1, ..., e_n$ eine Orthonormalbasis. Wir betrachten das $m$-fache Tensorprodukt $V^{\otimes m}$. Wer mag, kann zur Vereinfachung $n=3$ und $K=\IR$ wählen.
	\begin{subproblem}
		Einen beliebigen Tensor aus $V^{\otimes m}$ schreiben wir z.B. als $T\in V^{\otimes m}$, während er andernorts mit $T_{j_1 \cdots j_m}$ bezeichnet wird, was streng genommen nur eine Kollektion besonders nummerierter Zahlen aus $K$ ist. Wie ist der Zusammenhang zwischen $T$ und $T_{j_1 \cdots j_m}$?

		Hinweise: Von $V$ induzierte Basiswahl für $V^{\otimes m}$, Vergleiche mit einem Vektor $v\in V$ und $v_i$.
	\end{subproblem}
	\begin{subproblem}
		Ein sehr häufig verwendeter \enquote{Buchstabe mit Indizes} ist das Kronecker-$\delta$, oder auch der $\delta_{ij}$-Tensor.
		Um welchen Tensor handelt es sich hier?
		Hinweis: Übersetze in die Schreibweise mit dem Tensorprodukt $\otimes$.
	\end{subproblem}
	\begin{subproblem}
		Die Spur eines Tensors zwischen seinem $k$-ten und $l$-ten Faktor wird in Indexschreibweise als Dopplung eines bestimmten Indexes an den entsprechenden Stellen notiert, $T_{j_1\cdots j_{k-1} i j_{k+1} \cdots j_{l-1} i j_{l+1}  \cdots j_m}$, die eine Summe über $i$ von 1 bis $n$ impliziert (a.k.a. \emph{Einstein'sche Summenkonvention}). Überzeuge dich, dass dies unserer Definition von Spur entspricht.

		Üblich zum Spur nehmen ist auch eine Schreibweise mit dem Kronecker-$\delta$: \[T_{j_1\cdots j_{k-1} i j_{k+1} \cdots j_{l-1} r j_{l+1}  \cdots j_m} \delta_{ir},\] ebenfalls mit impliziter Summe über gedoppelte Indizes. Zeige, dass dies die gleiche Operation beschreibt.
	\end{subproblem}
	\begin{subproblem}
		Eine weitere häufiger zu findende Schreibweise mit dem Kronecker-$\delta$ ist die folgende:
		\[T_{j_1\cdots j_{k-1} j_{k} j_{k+1} \cdots j_{l-1} j_l j_{l+1}  \cdots j_m} \delta_{ir},\]
		was üblicherweise gekürzt wird auf $T_{j_1\cdots j_m} \delta_{j_{m+1} j_{m+2}}$. Was ist der Unterschied zu c.)? Schreibe diesen Tensor ohne Indizes auf.
	\end{subproblem}
\end{problem}

\begin{problem}[title={Was denn für Indizes?}]
	Schön, dass dir die Index-Schreibweise noch nicht begegnet ist. Da wir in unserem Kurs so wenig wie möglich mit dieser Schreibweise arbeiten wollen, kann dies zu deinem Vorteil sein.

	Es sei $V$ ein endlich-dimensionaler $\IR$-Vektorraum mit Skalarprodukt und $e_1, ..., e_n$ eine Orthonormalbasis. Wir betrachten das $m$-fache Tensorprodukt $V^{\otimes m}$. Wer mag, kann zur Vereinfachung $n=3$ und $K=\IR$ wählen. Zeige, dass $\tr_{ir}(T)=\tr_{i,m}\circ\tr_{r,m+2}\circ\casimir(T)$ gilt.

\end{problem}

\begin{problem}[title={Brauer-Diagramme}]
	Wir wollen uns im Folgenden anhand ein paar Beispielen überzeugen, dass die Multiplikation zweier Brauer-Diagrammen tatsächlich dem Hintereinanderausführen der dazugehörigen linearen Abbildungen entspricht.
	\begin{subproblem}
		Schreibe die Brauer-Diagramme von den beiden Abbildungen $\Pi_\sigma$ und $\casimir$ aus Gleichung \ref{eq:sigmaOfCasimir} auf und berechne das Brauer-Diagramm der  Abbildung $\Pi_\sigma \circ\casimir$.
	\end{subproblem}
	\begin{subproblem}
		Wir haben für die Multiplikation zweier Brauer-Diagramme definiert, dass für jeden geschlossenen Kreis das Ergebnis-Diagramm mit dem Faktor $\dim{V}$ multipliziert wird. Dieser Faktor muss so festgelegt werden, um Brauer-Diagramme als äquivalente Schreibweise für die vorgestellten linearen Abbildungen zwischen Tensoren benutzen zu können. Warum?
		
		Hinweis: Spur vom Casimir-Element
	\end{subproblem}
	\begin{subproblem}
		Gegeben seien die folgenden Brauer-Diagramme:
		\begin{align*}
			B_5&=\begin{tikzpicture}[baseline={([yshift=-.5ex]current bounding box.center)}]
				\foreach\x in {1,2,...,4}{
					\node[v] (s\x) at (\x,1){};
				}
				\foreach\x in {1,2,...,6}{
					\node[v] (h\x) at (\x,0){};
				}
				\foreach\x in {1,2}{
					\draw (s\x) to (h\x);
				}
			\draw (s3) to (h5);
			\draw (s4) to (h6);
			\draw (h3) to [bend left=30] (h4);
			\end{tikzpicture}
			&
			B_7&= 
			\begin{tikzpicture}[baseline={([yshift=-.5ex]current bounding box.center)}]
				\foreach\x in {1,2,...,6}{
					\node[v] (h\x) at (\x,0){};
				}
				\foreach\x in {1,3,4,6}{
					\node[v] (s\x) at (\x,1){};
				}
				\foreach\x in {1,3,4,6}{
					\draw (s\x) to (h\x);
				}
				\draw (h2) to [bend left=30] (h5);
			\end{tikzpicture}
			\\ \\ \\
			B_6&=
			\begin{tikzpicture}[baseline={([yshift=-.5ex]current bounding box.center)}]
				\foreach\x in {1,2,...,4}{
					\node[v] (h\x) at (\x,0){};
				}
				\foreach\x in {1,2,...,6}{
					\node[v] (s\x) at (\x,1){};
				}
				\draw (s1) to (h1);
				\draw (s4) to (h2);
				\draw (s5) to (h3);
				\draw (s6) to (h4);
				\draw (s2) to [bend right=30] (s3);
			\end{tikzpicture}
			\hspace{1cm}
			&
			B_8&=
			\begin{tikzpicture}[baseline={([yshift=-.5ex]current bounding box.center)}]
				\foreach\x in {1,2,...,4}{
					\node[v] (h\x) at (\x,0){};
				}
				\foreach\x in {1,2,...,6}{
					\node[v] (s\x) at (\x,1){};
				}
				\foreach\x in {1,2,...,4}{
					\draw (s\x) to (h\x);
				}
				\draw (s5) to [bend right=30] (s6);
			\end{tikzpicture}
		\end{align*}
		Zeige, dass das Ergebnis der Multiplikation $B_6\circ B_5$ der linearen Abbildung, die durch Hintereinanderausführen von $B_5$ und $B_6$ gegeben ist, entspricht. Zeige dies auch für $B_7 \circ B_8$.
	\end{subproblem}
\end{problem}


\begin{problem}[title={Vom Sinn und Unsinn der Basen}]
\begin{subproblem}
Anna und Bernd stehen einer flachen Wand gegenüber. Anna hat die Position $(9.5, 4,0)^T$ und  einen Ball in der Hand. Bernd steht bei $(5.5, 2, 0)^T$ und die Wand hat Normalengleichung \[2x-1.5y = 12.5\] mit Normalenvektor $\vc{n}=(2,-1.5,0)^T$ und Stützvektor $\vc{p}=(4,-3,0)^T$. 

\smallbreak
Anna möchte Bernd den Ball so zuwerfen, dass er einmal an der Wand abprallt und dann in Bernds Händen landet. Dabei soll der Ball den Boden nicht berühren. Berechne die Trajektorie vom Ball und die Kraft, mit der Anna den Ball abwerfen muss unter der Annahme von Reibungsfreiheit, Vakuum etc.
\end{subproblem}

Wir wollen jetzt nicht die Lösung für dieses Problem finden, sondern vielmehr fragen: Ist die hier gewählte Basis zur Beschreibung der Ausgangssituation oder für die Berechnung hilfreich? Wäre es nicht viel einfacher, die Situation basisfrei zu beschreiben? Wie sähe eine solche Beschreibung aus?

\smallbreak
Die Frage selbst können wir basisfrei z.B. so formulieren:
\begin{subproblem}
Anna und Bernd stehen 2.5 Meter voneinander entfernt in einer ebenen Wiese einer flachen Wand gegenüber. Beide haben einen Abstand zur Wand von 5 Metern. Anna hat einen Ball in der Hand und möchte ihn Bernd so zuwerfen, dass er einmal an der Wand abprallt und dann in Bernds Händen landet. Dabei soll der Ball den Boden nicht berühren. Berechne die Trajektorie vom Ball und die Kraft, mit der Anna den Ball abwerfen muss unter der Annahme von Reibungsfreiheit, Vakuum etc.
\end{subproblem}

Welche Basis würde man am ehesten wählen, um diese Aufgabe zu lösen und warum? Welche Wahlfreiheiten haben wir? Gibt es mehr als eine sinnvolle Basis?
\end{problem}

\begin{problem}[title={Wilde Behauptungen}, difficulty={schwer}]
Im Skript wurde behauptet:

\enquote{alle physikalisch sinnvollen Dinge [sind] von sich aus basisfrei und somit zwangsläufig auch basisfrei berechenbar, wenn sie überhaupt berechenbar sind}

Überzeuge dich davon, dass das nicht nur so daher gesagt ist, sondern ein beweisbarer Fakt ist. Insbesondere ist hier als Teilbehauptung enthalten: Es ist möglich, physikalisch sinnvolle Daten (sowohl die Input- als auch Output-Daten der Berechnung) basisfrei so zu repräsentieren, dass damit immer noch Berechnungen möglich sind.

\medbreak
\textcolor{red}{(Und es sei erneut davor gewarnt, dass \enquote{berechnen} nicht \enquote{effizient berechnen} bedeutet)}
\end{problem}


\begin{problem}[title={Komplexifizierung}]
Offensichtlich ist $\IR^n$ eine Teilmenge eines $\IC$-Vektorraums, nämlich $\IC^n$. Dies tritt häufiger auf: Der $\IR$-Vektorraum der Polynome mit reellen Koeffizienten $\IR[x_1,\ldots,x_n]$ Teilmenge des $\IC$-Vektorraums der Polynome mit komplexen Koeffizienten $\IC[x_1,\ldots,x_n]$, der $\IR$-Vektorraum der (stetigen/differenzierbaren/...) Funktionen $X\to\IR$ ist Teilmenge des $\IC$-Vektorraums der (stetigen/differenzierbaren/...) Funktionen $X\to\IC$.

\smallbreak
Ziel dieser Aufgabe ist es, zu zeigen, dass das kein Zufall ist, sondern dass wir jeden $\IR$-Vektorraum auf natürliche, basisfreie Weise als Teilmenge eines $\IC$-Vektorraums betrachten können, der sich \enquote{nur durch die Wahl der Koeffizienten von $V$ unterscheidet}.

\begin{subproblem}
Vorbereitung: Erinnere dich daran, dass $\IC$ auch als $\IR$-Vektorraum aufgefasst werden kann.

Allgemeiner: Erinnere dich daran, dass jeder $\IC$-Vektorraum auch als $\IR$-Vektorraum aufgefasst werden kann. (Dies nennt man \udot{Restriktion der Skalare})
\end{subproblem}

\begin{subproblem}
Definiere die \udot{Komplexifizierung} $V_\IC$ (auch \udot{Skalarerweiterung} genannt) als den folgenden $\IC$-Vektorraum: Seine Elemente sind dieselben wie die des $\IR$-Vektorraums $\IC\otimes_\IR V$ (=Tensorprodukt von $\IR$-Vektorräumen). Die Addition ist auch die des Tensorprodukts. Die Multiplikation mit Skalaren ist hingegen gegeben durch
\[\forall z,w\in\IC, v\in V: z\cdot(w\otimes v) := (zw)\otimes v\]
Zeige, dass diese Addition und Skalarmultiplikation wirklich einen $\IC$-Vektorraum definiert.
\end{subproblem}

\begin{subproblem}
Überzeuge dich davon, dass die obigen Beispiele alle Komplexifizierungen sind.
\end{subproblem}

\begin{subproblem}
Zeige, dass Skalarerweiterung die Dimension erhält in folgendem Sinne: $\dim_\IR(V) = \dim_\IC(V_\IC)$ (Wer es für nötig erachtet, darf sich wieder \enquote{für endlich-dimensionale $V$} dazudenken). Präziser: Zeige, dass $\set{1\otimes b | b\in B}$ eine $\IC$-Basis von $V_\IC$ ist, wenn $B$ eine $\IR$-Basis von $V$ ist.
\end{subproblem}

\begin{subproblem}
Was tut Restriktion der Skalare hingegen mit der Dimension?
\end{subproblem}

\begin{subproblem}
Komplexifizierung ist \enquote{funktoriell}, also mit linearen Abbildungen verträglich: Ist $f:V\to W$ eine $\IR$-lineare Abbildung, dann ist $f_\IC: V_\IC\to W_\IC, z\otimes v \mapsto z\otimes f(v)$ eine $\IC$-lineare Abbildung. Es gilt $(\id_V)_\IC=\id_{V_\IC}$ und $(f\circ g)_\IC=f_\IC\circ g_\IC$.
\end{subproblem}
\end{problem}

\end{sheet}

\pagebreak
\section{Darstellungstheorie -- Warum und was ist das überhaupt?}
\input{motivationdarstellungstheorie.tex}
% !TeX root = spherical_harmonics.tex
% !TeX spellcheck = de_DE

\subsection{Was ist eine Darstellung?}

\begin{definition}[Darstellungen]\label{darstellungen:def}
Sei $G$ eine Gruppe und $\IK$ ein Körper. Eine \emph{Darstellung von $G$ über $\IK$} besteht aus
\begin{itemize}
	\item einem $\IK$-Vektorraum $T$,
	\item einer Abbildung $\cdot: G \times T \to T, (g,t) \mapsto g\cdot t$, genannt \enquote{Operation} (der Gruppe auf dem Vektorraum),
\end{itemize}
die die Axiome in Tabelle \ref{darstellungen:def_table} erfüllen. Man sagt auch \enquote{$G$ operiert auf dem Vektorraum $T$} statt \enquote{$T$ ist eine Darstellung von $G$}.

\begin{table}[!ht]
	\setlength\extrarowheight{10pt} % for a bit of visual "breathing space"
	\begin{tabularx}{\textwidth}{p{7cm} X}
		
		\toprule
		\textbf{Axiome von Darstellungen}                    & \textbf{Bedeutung} \\
		\midrule
        \hspace{1cm}Linearität                               & $\forall g\in G: (T\to T, v\mapsto g\cdot v)$ ist eine lineare Abbildung \\
		\hspace{1cm}Assoziativität                           & $\forall g,h\in G\forall v\in T: g\cdot (h\cdot v)=(g\cdot h)\cdot v$  \\
		\hspace{1cm}Nichttrivialität/Normierung              & $\forall v\in T: 1_G\cdot v=v$  \\
		\textbf{Axiome von Homomorphismen}                   & \textbf{Bedeutung} \\
        \midrule
        \hspace{1cm}$\IK$-Linearität & $f$ ist eine lineare Abbildung $U \to T$ \\
        \hspace{1cm}$G$-Linearität & $\forall g\in G, u\in U: f(g\cdot u) = g\cdot f(u)$ \\
        \bottomrule
	\end{tabularx}
	\caption{Definierende Eigenschaften von Darstellungen und Homomorphismen zwischen Darstellungen}
    \label{darstellungen:def_table}
\end{table}

Sind $U,T$ zwei Darstellungen von $G$ und $f: U\to T$ eine Abbildung, so heißt $f$ \emph{Homomorphismus} oder \emph{$\IK G$-lineare Abbildung}, falls die beiden Axiome in Tabelle \ref{darstellungen:def_table} erfüllt sind. Den Raum aller $\IK G$-linearen Abbildungen von $T$ nach $W$ bezeichnen wir mit $\Hom_{\IK G}(U,T)$. Eine $\IK G$-linearen Abbildung von $T$ nach $T$ (also gleicher Definitions- und Zielraum) heißt \emph{Endomorphismus}, der Raum aller solcher Abbildungen wird mit $\End_{\IK G}(T)$ notiert.

Existiert ein Homomorphismus $f': V\to U$ mit $f\circ f'=f'\circ f=\id$, so nennt man $f$ \emph{Isomorphismus} der Darstellungen.
\end{definition}

\begin{convention}
Wir werden ausschließlich endlich-dimensionale Darstellungen betrachten. Es gibt auch eine reichhaltige Theorie unendlich-dimensionaler Darstellungen, die für uns aber nicht relevant sein wird.

Wir werden uns außerdem ausschließlich für die Fälle $\IK=\IC$ und ein bisschen $\IK=\IR$ interessieren und andere Körper außen vor lassen. (Typischerweise ist Darstellungstheorie über $\IC$ immer der einfachste Fall und Darstellungstheorie über anderen Körpern als $\IC$ ist mindestens genauso schwierig oder schwieriger)
\end{convention}

\begin{remark}
Eine Darstellung kann alternativ aufgefasst werden als Gruppenhomomorphismus $G\to GL(T)$: Jedem Gruppenelement $g\in G$ wird die (invertierbare!) lineare Abbildung $v\mapsto g\cdot v$ zugeordnet. Ist umgekehrt ein Gruppenhomomorphismus $\mathcal{D}: G\to GL(T)$ gegeben, so kann man $T$ als $G$-Darstellung auffassen, indem man $g\cdot v:=\mathcal{D}(g)(v)$ definiert.

\smallbreak
Diese beiden Sichtweisen sind völlig identisch und beliebig austauschbar. Je nach Situation kann es einfacher sein, einen Homomorphismus nach $GL(T)$ oder direkt die Operation der Gruppe auf $T$ zu definieren. Insbesondere können wir eindimensionale Darstellungen äquivalent auch als Gruppenhomomorphismen $G\to\IK\setminus{0}$ auffassen.

\medbreak
Hat man aus irgendeinem Grund eine Basis von $T$ gegeben, kann man bekanntlich lineare Abbildungen mit Matrizen identifizieren. Eine Darstellung ist in dieser Sichtweise dann eine Abbildung ${D}: G\to \IK^{n\times n}$, die jedem Gruppenelement $g\in G$ eine (invertierbare) Matrix $D(g)$ zuordnet, sodass $D(1)=1_{n\times n}$ und $D(g\cdot h)=D(g)\cdot D(h)$ erfüllt sind. Solch eine Abbildung nennt man dann auch \udot{Matrixdarstellung}.
\end{remark}

\begin{example}
\begin{itemize}
\item $GL(T)$ operiert auf $T$ via $g\cdot v := g(v)$, die sogenannte \udot{natürliche} oder \udot{kanonische} Darstellung.
\item Die symmetrische Gruppe $S_m$ operiert auf $\IK^m$ durch Vertauschen der Komponenten (die sogenannte \udot{Permutations-} oder \udot{kanonische Darstellung}):
\[\sigma \cdot (x_1,\ldots,x_m) := (x_{\sigma^{-1}(1)}, \ldots, x_{\sigma^{-1}(m)})\]
und auf der Tensorpotenz $T=V^{\otimes m}$ durch Vertauschen der Faktoren:
\[\sigma \cdot v_1\otimes\cdots\otimes v_m := v_{\sigma^{-1}(1)}\otimes \cdots \otimes v_{\sigma^{-1}(m)} \]
\item Jede beliebige Gruppe hat \udot{triviale Darstellungen}, nämlich für jeden beliebigen Vektorraum $T$:
\[g\cdot v := v\]
Spricht man von \emph{der} trivialen Darstellung meint man damit diejenige mit $T=\IK$.
\item Die symmetrische Gruppe hat eine nichttriviale, eindimensionale Darstellung mit $T=\IK$, das \udot{Signum} $\sgn: G\to\IK$:
\[\sgn(\sigma) := \begin{cases} -1 & \text{falls $\sigma$ ungerade viele Inversionen enthält} \\ +1 & \text{andernfalls}\end{cases}\]
\item Der Raum $T:=\IK[X_1, \ldots, X_n]$ der Polynome in $n$ Variablen mit Koeffizienten aus $\IK$ ist eine Darstellung für die Gruppe $G=GL_n(\IK)$ der invertierbaren $n\times n$-Matrizen via:
\[g\cdot p := p\left(\sum_{i=1}^n (g^{-1})_{1i} X_i, \sum_{i=1}^n (g^{-1})_{2i} X_i, \ldots, \sum_{i=1}^n (g^{-1})_{ni} X_i\right)\]
Für $\IK=\IR$ oder $\IC$ können wir $T$ auch als Raum der polynomiellen Abbildungen $\IK^n \to \IK$ auffassen. Dann schreibt sich dieselbe Definition etwas einfacher als
\[g \cdot p := X\mapsto p(g^{-1}(X))\]
\end{itemize}
(Aufgabe \ref{ex:invertieren_fuer_linksmoduln}: Beweise, dass das $^{-1}$, das ab und zu auftrat, wirklich notwendig ist)
\end{example}

\subsection{Wie kann ich mir eine Darstellung basteln?}
\begin{lemmadef}[Neue Darstellungen aus vorhandenen]
Sei $G$ eine Gruppe und $U,T$ Darstellungen von $G$ über einem festen Körper $\IK$. Die folgenden sind dann auch Darstellungen von $G$:
\begin{enumerate}
\item Die direkte Summe $U\oplus T := \Set{(u,t) | u\in U, t\in T}$ wird zu einer Darstellung durch
\[g\cdot(u,t) := (gu,gt)\]
\item Das Tensorprodukt $U\otimes T$ durch
\[g(u\otimes t) := (gu)\otimes (gt)\]
Das beinhaltet insbesondere alle Tensorpotenzen $U^{\otimes k}$.
\item Der Raum der $\IK$-linearen Abbildungen $\Hom_\IK(U,T)$ durch
\[g\cdot f := u\mapsto gf(g^{-1} u)\]
\item Der Dualraum $U^\ast:= \Hom_\IK(U,\IK)$ durch
\[g\cdot \phi := u\mapsto \phi(g^{-1} u)\]
(Aufgabe \ref{ex:invertieren_fuer_linksmoduln} Beweise, dass das $^{-1}$ wirklich notwendig ist)
\item Ist $H\leq G$ eine Untergruppe von $G$, dann operiert $H$ natürlich auch auf $T$. Die Darstellung von $H$, die wir so erhalten, nennt man \emph{Restriktion von $T$ auf $H$}.
\end{enumerate}
\end{lemmadef}

\begin{remark}
Man beachte, dass d. ein Spezialfall von c. ist, wobei man für $W$ die triviale Darstellung $\IK$ eingesetzt hat.
\end{remark}

\begin{example}[Zentrales Beispiel -- Die zwei- und dreidimensionalen Dreh- bzw. Isometriegruppen]
Wie bereits festgestellt, ist jeder Vektorraum $V$ eine $GL(V)$-Darstellung. Ist $V$ ein reeller Vektorraum, auf dem ein Skalarprodukt gegeben ist, so hat $GL(V)$ die Untergruppe der Isometrien $O(V)$, d.h. die Gruppe aller mit dem Skalarprodukt verträglichen Abbildungen:
\[O(V) := \Set{f\in GL(V) | \forall v,w\in V: \braket{f(v),f(w)} := \braket{v,w}}\]
In dieser Gruppe enthalten sind z.B. alle Rotationen, alle Spiegelungen sowie die Inversion $\iota: v\mapsto -v$ (die manchmal eine Drehung ist).

\smallbreak
Wir werden die Darstellungstheorie der Isometriegruppen von zwei- und dreidimensionalen reellen Vektorräumen entwickeln. Wir werden also $V$, $V^{\otimes m}$, den Raum der polynomiellen Abbildungen $V\to\IK$ uvm. als Darstellungen von $O(V)$ auffassen.

\smallbreak
Es ist außerdem manchmal von Vorteil, den dreidimensionalen Raum auch als Darstellung für die zweidimensionale Isometriegruppe auffassen, indem wir $O_2$ als Untergruppe von $O_3$ auffassen. Dies geschieht, indem wir eine Ebene $E\leq V$ festlegen und $O_2(E)$
\[g(u+u') := gu + u'\]
festlegen für alle $u\in E$ und $u'\in E^\perp$, d.h. die gewählte Ebene verhält sich wie die kanonische Darstellung von $O_2$, während jeder Vektor in der zu ihr orthogonalen Gerade fixiert wird. Da $V=E\oplus E^\perp$ gilt, haben wir damit die Operation für alle Vektoren aus $V$ festgelegt.

\medbreak
Dies ist sinnvoll in physikalischen Systemen mit einer festen (unendlich ausgedehnten) Ebene, wo zwar alles rotations- und spiegelsymmetrisch ist, aber nur solange diese Ebene erhalten bleibt.
\end{example}

\subsection{Weitere grundlegende Begriffe der Darstellungstheorie}
\begin{definition}[Unterdarstellungen]
Sei $T$ eine Darstellung von $G$. Ein Untervektorraum $U\leq T$ heißt \emph{$G$-invarianter Unterraum} oder \emph{Unterdarstellung von $T$}, falls $U$ unter der Gruppenoperation abgeschlossen ist, d.h.
\[\forall g\in G: g\cdot U = U\]
\end{definition}

\begin{example}
\begin{itemize}
\item $\set{0}$ und $T$ sind immer Unterdarstellungen von $T$.
\item Die Gerade $U_1=\set{\lambda(1,1,\ldots,1) | \lambda\in K}$ und die Hyperebene $U_2 := \set{(x_1,\ldots,x_m) | \sum_{i=1}^m x_i = 0}$ sind zwei Unterdarstellungen der Permutationsdarstellung von $G=S_m$ auf $T=\IK^m$. Falls $\IK\in\set{\IR,\IC}$ ist, dann gilt $T=U_1\oplus U_2$.
\end{itemize}
\end{example}

\begin{definition}[Irreduzible Darstellungen]
Eine Darstellung, die exakt zwei Unterdarstellungen hat (nämlich $\set{0}$ und sich selbst), heißt \emph{irreduzibel} oder \emph{einfach}. Eine nicht-einfache Darstellung heißt entsprechend auf \emph{reduzibel}.
\end{definition}

\begin{remark}
Man vergleiche mit der Definition einer Primzahl als natürliche Zahl, die genau zwei Teiler hat.
\end{remark}

\begin{remark}
Weil wir \enquote{exakt zwei} und nicht \enquote{höchstens zwei} fordern, ist der Nullvektorraum $\set{0}$ niemals eine irreduzible Darstellung.
\end{remark}

\begin{example}
\begin{itemize}
\item Aus Dimensionsgründen ist jede eindimensionale Darstellung automatisch irreduzibel.
\item $V$ ist eine irreduzible Darstellung von $GL(V)$ unabhängig vom Körper.
\item $\IK^n$ ist eine irreduzible Darstellung von $O_n$, sowohl für $\IK=\IR$ als auch $\IK=\IC$.
\item $\IK^3$ ist \emph{nicht} irreduzibel als Darstellung von $O_2$.
\end{itemize}
\end{example}

\begin{remark}
An den letzten beiden Beispielen erkennen wir, dass die Restriktion einer irreduziblen Darstellung auf eine Untergruppe selbst wieder irreduzibel sein kann, aber nicht muss: Wenn man von $GL_3$ zu $O_3$ einschränkt, bleibt $\IK^3$ irreduzibel; wenn wir von $O_3$ zu $O_2$ einschränken, bleibt es das nicht.

Auch die anderen Konstruktionsmöglichkeiten von neuen Darstellungen aus bekannten erhalten i.A. Irreduzibilität nicht:
\begin{itemize}
\item Die direkte Summe von zwei Darstellungen $\neq 0$ ist niemals irreduzibel, weil die beiden Summanden invariante Unterräume sind.
\item Das Tensorprodukt von Darstellungen ist immer reduzibel, wenn mindestens einer der Faktoren reduzibel ist, denn $(U_1\oplus U_2)\otimes W = (U_1\otimes W) \oplus (U_2\otimes W)$. Das Tensorprodukt von irreduziblen Darstellungen ist i.A. aber auch nicht irreduzibel, z.B. ist $T\otimes T$ niemals irreduzibel, wenn $T$ nicht zufällig eindimensional ist. (Wenn es eindimensional ist, ist $T\otimes T$ natürlich auch eindimensional und dementsprechend ausnahmsweise doch irreduzibel)

Es ist i.A. ein sehr schweres Problem, zu bestimmen, ob ein Tensorprodukt von zwei irreduziblen wieder irreduzibel ist und, wenn es das nicht ist, wie die irreduziblen Unterräume des Tensorprodukts genau aussehen. Im Falle $G=O_3$ ist diese Fragestellung unter dem Namen \enquote{Clebsch\footnote{Alfred Clebsch, 1833--1872, dt. Mathematiker}-Gordan\footnote{Paul Albert Gordan, 1832--1912, dt. Mathematiker}-Theorie} bekannt.
\end{itemize}

Einzige Ausnahme ist das Dualisieren:
\end{remark}

\begin{lemma}
Eine endlichdimensionale Darstellung $T$ ist irreduzibel genau dann, wenn $T^\ast$ irreduzibel ist.
\end{lemma}
% !TeX spellcheck = de_DE
% !TeX root = spherical_harmonics.tex
\begin{sheet}

\begin{problem}[title={Komische Minus Einsen: Invertieren für die Assoziativitätsbedingung}, difficulty=leicht]\label{ex:invertieren_fuer_linksmoduln}
\begin{subproblem}
In den Beispielen wurde definiert, dass die symmetrische Gruppe $Sym(m)$ durch
\[\sigma \cdot (v_1,\ldots,v_m) := (v_{\sigma^{-1}(1)}, \ldots, v_{\sigma^{-1}(m)})\]
auf $K^m$ auf der Tensorpotenz $V^{\otimes m}$ durch
\[\sigma \cdot v_1\otimes\cdots\otimes v_m := v_{\sigma^{-1}(1)}\otimes \cdots \otimes v_{\sigma^{-1}(m)} \]
operiert.
\end{subproblem}
\begin{subproblem}
Ebenfalls definiert wurde, dass bei einer gegebenen Darstellung $V$ der Dualraum $V^\ast := \Hom_K(V,K)$ zu einer Darstellung derselben Gruppe wird, indem man sie durch
\[g\cdot \phi := v\mapsto \phi(g^{-1}v)\]
operieren lässt.
\end{subproblem}
\begin{subproblem}
Ebenfalls definiert wurde, dass $G=GL_n(\IK)$ auf $V:=\IK[X_1, ..., X_n]$ dem Raum der polynomiellen Abbildungen durch
\[g\cdot p := v\mapsto p(g^{-1}X)\]
operieren lässt.
\end{subproblem}

Beweise, dass das Invertieren in diesen Definitionen nötig ist, damit die Assoziativitätsbedingung gilt.
\end{problem}

\begin{problem}[title={Darstellung der symmetrischen Gruppe mit Matrizen}]
Wie lautet die Matrixdarstellung zu der obigen Darstellung von $Sym(m)$ auf $K^m$ ?
\end{problem}

\begin{problem}[title={Dualisiere eine Matrixdarstellung}]
Wenn $\mathcal{D}: G\to K^{n\times n}$ eine Matrixdarstellung von $G$ auf $V$ bzgl. irgendeiner fest gewählten Basis $B\subseteq V$ ist, wie lautet die Matrixdarstellung $\mathcal{D}^\ast$ der dualen Darstellung von $G$ auf dem Dualraum $V^\ast$ bzgl. der dualen Basis $B^\ast$ ?
\end{problem}

\begin{problem}[title={Basiswechsel einer Darstellung}]
Wenn $\mathcal{D}$ und $\mathcal{D}'$ zwei Matrixdarstellungen derselben Darstellung von $G$ auf $V$ sind, d.h. $\mathcal{D}$ bzgl. einer Basis $B$ und $\mathcal{D}'$ bzgl. einer zweiten Basis $B'$ gedacht ist, und wenn $S$ die Basiswechselmatrix von $B$ nach $B'$ ist, was ist dann der Zusammenhang zwischen $\mathcal{D}$, $S$und $\mathcal{D}'$ ?
\end{problem}

\begin{problem}[title={Isomorphismus zwischen Tensorprodukt und linearen Abbildungen}]\label{ex:hom_tensor_isomorphismus}
\begin{subproblem}
Es seien $V$ und $W$ zwei endlich-dimensionale Darstellungen von $G$. Man beweise, dass die beiden Darstellungen $\Hom_K(V,W)$ und $V^\ast \otimes W$ via
\[\alpha: V^\ast \otimes W \to \Hom_K(V,W), \phi \otimes w \mapsto (v\mapsto \phi(v)w)\]
isomorph sind.
\end{subproblem}

Insbesondere erhalten wir im Spezialfall $V=W$ einen Isomorphismus zwischen $V^\ast\otimes V$ und $\Hom_K(V,V)$.

\begin{subproblem}
Für $V=W$, wie lautet der Tensor, der durch $\alpha$ auf $\id_V$ abgebildet wird?
\end{subproblem}

\begin{subproblem}
Zeige: Die Auswertungsabbildung $\varepsilon: V^\ast \otimes V \to K, \phi\otimes v\mapsto \phi(v)$ ist $G$-linear.
\end{subproblem}

\begin{subproblem}
$\varepsilon\circ\alpha^{-1}$ ist eine $G$-lineare Abbildung $\beta: \Hom_K(V,V) \to K$. Welche?
\end{subproblem}
\begin{subproblem}
	Matrizen und 2-Tensoren sind einander sehr ähnlich, so ähnlich sogar, dass sie zueinander isomorph sind. Gegeben ein $K$-Vektorraum $V$ der Dimension $n$. Genauer: Der Raum der Matrizen $K^{n\times n}$ und $V\otimes V^\ast$ sind isomorph zueinander. Wir wollen nun diesen Isomorphismus genauer betrachten.
	
	$V\otimes V$, $V\otimes V^\ast$, $V^\ast \otimes V^\ast$
\end{subproblem}
\end{problem}


\begin{problem}[title={Reduzibilität von $V\otimes V$}]
Zeige, dass $V\otimes V$ niemals irreduzibel ist, wenn $\dim(V) \neq 1$ ist.
\end{problem}

\begin{problem}[title={Irreduzibilität von $V^\ast\implies$ Irreduzibilität von $V$?}]
Zeige, dass $V$ irreduzibel ist, wenn $V^\ast$ es ist.
\end{problem}

\begin{problem}[title={Kanonische Darstellung von $SO_2$}]\label{ex:fundamentaldarstellung_von_so2}
\begin{subproblem}
Zeige, dass die kanonische Darstellung $V=\IR^2$ von $G=SO_2$ irreduzibel ist.
\end{subproblem}
\begin{subproblem}
Zeige, dass die komplexifizierte Darstellung $V_\IC=\IC^2$ von $G$ reduzibel ist.
\end{subproblem}
\begin{subproblem}
Finde die irreduziblen Unterdarstellungen von $V_\IC$.

Hinweis: Eigenräume.
\end{subproblem}
\end{problem}

\begin{problem}[title={Die kanonische Darstellung von $SO_3$ ist irreduzibel}]\label{ex:fundamentaldarstellung_von_so3}
Zeige, dass $\IK^3$ eine irreduzible Darstellung von $G=SO_3$ ist sowohl für $\IK=\IR$ als auch für $\IK=\IC$.
\end{problem}

\begin{problem}[title={Was geht bei $SO_2$ schief?}]
Spoiler: $SO_2$ ist kommutativ.

\begin{subproblem}
Zeige (oder erinnere dich) zunächst: Ist $\lambda\in\IC$ beliebig und sind $\alpha,\beta\in\End_\IC(V)$ zwei kommutierende Endomorphismen (d.h. $\alpha\circ\beta=\beta\circ\alpha$), dann ist der Eigenraum $\operatorname{Eig}_\lambda(\alpha)$ ein $\beta$-invarianter Untervektorraum.
\end{subproblem}
\begin{subproblem}
Zeige, dass alle endlich-dimensionalen, irreduziblen, komplexen Darstellungen einer kommutativen Gruppe $G$ eindimensional sind.
\end{subproblem}
\end{problem}

\begin{problem}[title={(Nicht-)Isomorphie von $V$ und $V^\ast$}]
Sei $V$ ein endlichdimensionaler $\IR$-Vektorraum. Zweck dieser Aufgabe ist es, genauer zu beleuchten, wieso $V$ und $V^\ast$ Grund verschiedene Vektorräume sind, obwohl sie als Vektorräume ja isomorph sind (d.h. die gleiche Dimension haben)
\begin{subproblem}
\textbf{Manchmal sind sie isomorph -- Satz von Riesz}.
Es sei ein Skalarprodukt $\braket{,}$ auf $V$ gegeben. Zeige, dass $V$ und $V^\ast$ als $O(V)$-Darstellungen via $V\to V^\ast, v\mapsto \braket{v,\cdot}$ isomorph sind.
\end{subproblem}
\begin{subproblem}[difficulty={schwerer als man denkt}]
\textbf{Manchmal aber auch nicht}.
Zeige, dass $V$ und $V^\ast$ \emph{nicht} isomorph sind als $GL(V)$-Darstellungen.

(Sobald wir Charaktertheorie haben, wird's einfach)
\end{subproblem}
\end{problem}

\begin{remark}
In allen Kontexten, in denen auch mit schiefen Koordinatensystemen gerechnet werden muss, \emph{muss} deshalb zwischen $V$ und $V^\ast$ sowie zwischen ko- und kontravarianten Tensoren unterschieden werden, da es keinen in diesem Kontext natürlichen Isomorphismus $V\isomorphic V^\ast$ gibt.

In allen Kontexten, in denen ausschließlich mit orthonormierten Koordinatensystemen gerechnet wird, existiert hingegen ein für diesen Kontext natürlichen Isomorphismus, sodass es sinnvoll ist, alle Tensoren gleich zu behandeln. Die Unnatürlichkeit des allgemeinen Falls wird so in der Wahl des Skalarprodukts versteckt.
\end{remark}

\begin{problem}[title={Quotienten}]
Das duale Konzept zu Unterdarstellungen sind Quotientendarstellungen. Zeige: Ist $V$ eine Darstellung von $G$ und $U\leq V$ ein $G$-invarianter Unterraum, dann ist auch der Quotient $V/U$ eine Darstellung von $G$ via $g\cdot\overline{v} := \overline{g\cdot v}$.
\end{problem}
	
\end{sheet}

\pagebreak
\section{Darstellungstheorie -- Level 0}
% !TeX root = spherical_harmonics.tex
% !TeX spellcheck = de_DE
\subsection{Symmetrische Tensoren: Die Verbindung zwischen Polynomen und Tensoren}

\begin{definition}\label{symmetrische_tensoren:def}
Ein Tensor $m$-ten Grades $t\in V^{\otimes m}$ heißt \emph{symmetrisch}, wenn er ein Fixpunkt der Operation der Permutationsgruppe $Sym(m)$ ist, d.h. wenn
\[\forall \sigma\in Sym(m): \sigma\cdot t = t\]
gilt. Der Raum der symmetrischen Tensoren wird \emph{symmetrische (Tensor-)Potenz} genannt und $Sym^m(V)$ geschrieben.
\end{definition}

\begin{example}
\begin{itemize}
\item $v\otimes v\otimes \cdots\otimes v$ ist symmetrisch und umgekehrt: Ein reiner Tensor $v_1\otimes v_2\otimes\cdots\otimes v_m$ ist genau dann symmetrisch, wenn alle $v_1,\ldots, v_m$ Vielfache eines Vektors $v$ sind.
\item $v\otimes w + w\otimes v$ ist symmetrisch. Man beachte, dass $v\otimes w + w\otimes v = (v+w)\otimes(v+w) - v\otimes v - w\otimes w$ ist. Man kann allgemein zeigen, dass $Sym^m(V) = \operatorname{span}\Set{v\otimes\cdots\otimes v | v\in V}$ ist.
\end{itemize}
\end{example}

\begin{lemma}[Mittelwerttrick]\label{symmetrische_tensoren:projektion}
Die Projektion auf den Unterraum der symmetrischen Tensoren gegeben durch
\[q(t) := \frac{1}{m!} \sum_{\sigma\in Sym(m)} \sigma\cdot t\]
und $O_n$-linear.
\end{lemma}

\begin{theorem}[Isomorphismus von Polynom(iellen Abbildung)en und symmetrischen Tensoren]
\begin{enumerate}
\item Es sei $e_1, e_2, \ldots, e_n$ eine Basis von $V$. Der Raum der Polynome ist isomorph zu $\bigoplus_{m=0}^\infty \Sym^m(V)$. Der Isomorphismus ist die Einschränkung der Abbildung $\bigoplus_{m=0}^\infty V^{\otimes m} \to \IK[x_1,\ldots,x_n]$
\[e_{i_1} \otimes e_{i_2} \otimes \cdots \otimes e_{i_m} \mapsto x_{i_1}\cdot x_{i_2} \cdots x_{i_m}\]
Die umgekehrte Richtung ist durch $x_1^{k_1} \cdots x_n^{k_n} \mapsto q(e_1^{\otimes k_1} \otimes e_2^{\otimes k_2} \otimes \cdots \otimes e_n^{\otimes k_n})$ gegeben.
\end{enumerate}
Diese Aussage ist nicht basis-frei und kann auch basis-frei gemacht werden, da es in $\IK[x_1, \ldots, x_n]$ ja $n$ ausgezeichnete Elemente vom Grad 1 gibt, eben die Unbekannten. Jeder Isomorphismus würde also auch eine Basis von $V$ auszeichnen.

Eine moralisch äquivalente, basisfreie Aussage ist aber die folgende:
\begin{enumerate}[resume]
\item Der Raum der polynomiellen Abbildungen $V\to\IK$ ist isomorph zu $\bigoplus_{m=0}^\infty \Sym^m(V)$. Der Isomorphismus ist die Einschränkung der Abbildung $\bigoplus_{m=0}^\infty V^{\otimes m} \to \IK[x_1,\ldots,x_n]$
\[v_{i_1} \otimes v_{i_2} \otimes \cdots \otimes v_{i_m} \mapsto (v\mapsto \braket{v_{i_1},v} \cdot \braket{v_{i_2},v} \cdots \braket{v_{i_m},v}) \]
\item In beiden Fällen gilt: Die Räume sind als Darstellungen der orthogonalen Gruppe isomorph; die angegebenen Isomorphismen sind $O_n$-linear.
\end{enumerate}
\end{theorem}
\begin{proof}
Da das Produkt von Polynomen kommutativ ist, werden $e_{i_1} \otimes e_{i_2} \otimes \cdots \otimes e_{i_m}$ und alle Permutationen $e_{i_{\sigma(1)}} \otimes e_{i_{\sigma(2)}} \otimes \cdots \otimes e_{i_{\sigma(m)}}$ auf denselben Wert abgebildet. Also wird der Mittelwert dieser $m!$ Tensoren ebenfalls auf denselben Wert abgebildet, d.h. $q(e_{i_1} \otimes e_{i_2} \otimes \cdots \otimes e_{i_m})$ geht auch auf $x_{i_1}\cdot x_{i_2} \cdots x_{i_m}$. Das zeigt, dass die beiden in a. definierten Abbildungen zueinander invers sind.

Indem man eine Orthonormalbasis für $e_1, \ldots, e_n$ einsetzt, sieht man, dass in b. tatsächlich derselbe Isomorphismus definiert wird, da $v\mapsto \braket{e_i, v}$ genau die $i$-te Koordinatenabbildung bzgl. dieser Basis ist, d.h. die Abbildung, die den Vektor mit den Koordinaten $(x_1, x_2, \ldots, x_n)$ auf $x_i$ schickt. An der Schreibweise in b. sieht man aber leichter als der in a., dass die Isomorphismen $O_n$-linear sind. 
\end{proof}
% !TeX root = spherical_harmonics.tex
% !TeX spellcheck = de_DE
\subsection{Natürliche Abbildungen}

\begin{remark}
Im Kontext unserer physikalischen Welt wollen wir Messungen von Quantitäten durchführen. Diese Quantitäten können unter anderem als Zahlen, Vektoren oder Tensoren verstanden werden, z. B. beschreibt der Spannungstensor die Zug-, Druck- und Scherspannungen in einem Festkörper.

Eine fundamentale Eigenschaft, die wir von allen messbaren Quantitäten erwarten, ist, dass das Ergebnis der Messung unabhängig von der Wahl des Koordinatensystem ist. Ein bestimmter Geschwindigkeitsvektor sollte immer in die gleiche Richtung zeigen, völlig unabhängig davon, ob seine Koordinaten nun $(3,4,0)$ oder $(0,0,5)$ sind. Gleiches gilt für seine Länge, oder die Determinante einer Matrix (=das Volumen eines gewissen räumlichen Körpers). Generell sind wir an den Eigenschaften interessiert, die charakteristisch sind für das untersuchte System, nicht für das Koordinatensystem, mit dem wir das System untersuchen.

Konkreter: Die Eigenschaften sollen \emph{unabhängig von der Drehung und Orientierung} des Koordinatensystems sein; die Basisvektoren zu rotieren, zu invertieren oder zu spiegeln sollte den gemessenen Wert einer Eigenschaft nicht verändern.

Gleiches gilt auch bei Interaktionen in der physikalischen Welt, die Wirkung eines bestimmten Zustandes auf zukünftige Zustände ist ebenfalls völlig unabhängig von dem Koordinatensystem, mit dem wir den Zustand und seine Wirkung beschreiben. Der resultierende Zustand muss also unabhängig von der Drehung/Spiegelung des Koordinatensystems sein. Dual dazu könnte man fordern: Dreht/spiegelt man den Input (und lässt dafür das Koordinatensystem gleich), so muss sich auch der Output genauso drehen/spiegeln.

Diese Idee lässt sich mathematisch präzisieren, sobald einem klar ist, dass jeder physikalische Prozess eine Abbildung von einem Zustand in einen anderen ist, die mit der Gruppe $O_3$ verträglich ist. Auch eine Messung ist z.B. eine Abbildung vom Zustand auf eine Zahl.
\end{remark}

\begin{definition}[Natürliche Abbildungen]
Es sei $G$ eine Gruppe, $X$ und $Y$ zwei Darstellungen von $G$ über $\IK$ und $\phi$ eine lineare Abbildung von $X$ nach $Y$. Wir nennen $\phi$ \emph{natürlich}\footnote{Es gibt eine allgemeinere, viel technischere Definition, was \enquote{natürlich} heißt, aber sie reduziert sich in den für uns wichtigen Fällen im Wesentlichen auf die hier gegebene Definition.}, wenn 
	\begin{equation*}
		\forall \rho\in G: \phi \circ \rho = \rho \circ \phi.
	\end{equation*}
\end{definition}

\begin{remark}
D.h. mit den Worten aus dem Darstellungstheorie-Kapitel: Sie muss $\IK G$-linear sein.
\end{remark}

\begin{remark}
In unserem Kurs behandeln wir nur lineare Abbildungen, wobei multi-lineare Abbildungen durch das Tensorprodukt linearisiert werden können. Wir könnten deshalb auch über natürliche polynomielle Abbildungen reden, wenn wir das wollten. $G$ ist für uns typischerweise $O_3$, wir betrachten also $\IK O_3$-lineare Abbildungen.
\end{remark}

\subsection{Vollständigkeit der Klassifikation der linearen Tensor-Abbildungen}
Wenden wir uns nun den Abbildungen zwischen Tensoren zu. Wir haben sie im ersten Kapitel bereits klassifiziert mit folgendem Versprechen:
\begin{theorem}[Brauer-Diagramme beschreiben alle natürlichen linearen Abbildungen zwischen Tensoren]
	\label{brauer:klassifikation}
	Die Menge aller Linearkombinationen von allen Brauer-Diagrammen ist gleich der Menge aller $O_n$-linearen Abbildungen zwischen Tensoren.
\end{theorem}
\begin{proof}[Beweis per Induktion]
Zunächst gilt es festzustellen, dass die drei Klassen von linearen Abbildungen zwischen Tensoren, die wir mittels Brauer-Diagrammen definiert haben (\ref{def:permutation}, \ref{def:spur} und \ref{def:casimireinfuegen}) tatsächlich natürliche Abbildungen sind (s. Aufgabe \ref{aufg:natAbbKlass}). Die Hintereinanderausführung und Linearkombination von linearen natürlichen Abbildungen ist selbst ebenfalls eine lineare natürliche Abbildung. 
 
\medbreak
In Aufgabe \ref{ex:hom_tensor_isomorphismus} haben wir festgestellt, dass jede lineare Abbildung $V^{\otimes m} \to V^{\otimes n}$ einem Element des Tensorraums $\left(V^{\otimes m}\right)^\ast \otimes V^{\otimes n}$ entspricht. Die Suche nach einer $O_3$-kompatiblen linearen Abbildung $V^{\otimes m} \to V^{\otimes n}$ entspricht also der Suche nach einem $O_3$ Fixpunkt in $\left(V^{\otimes m}\right)^\ast \otimes V^{\otimes n}$.
Dieser Schritt lässt sich an Hand eines farbigen Brauer-Diagramms verdeutlichen: Die Knoten der oberen Zeile werden nach unten vor die untere Zeile verschoben und alle Kanten werden beibehalten. Das Ergebnis lässt sich als ein Brauer-Diagramm lesen, welches nur aus Casimir-Elementen besteht und keinen Knoten in der oberen Zeile hat, also streng genommen als eine Abbildung $\IK \to \left(V^{\otimes m}\right)^\ast \otimes V^{\otimes n}$.
\begin{align*}
	\begin{tikzpicture}[baseline={([yshift=-.5ex]current bounding box.center)}]
		\foreach\x in {1,2,...,5}{
			\node[r] (s\x) at (\x,1){};
		}
		\foreach\x in {1,2,...,7}{
			\node[b] (h\x) at (\x,0){};
		}
		\draw[purple] (s1) to (h1);
		\draw[purple] (s2) to (h3);
		\draw[purple] (s3) to (h2);
		\draw[red] (s4) to [bend right=30] (s5);
		\draw[blue] (h4) to [bend left=30] (h5);
		\draw[blue] (h6) to [bend left=30] (h7);
	\end{tikzpicture} &
	\\ \\ \\
	\mapsto
	\begin{tikzpicture}[baseline={([yshift=-2ex]current bounding box.center)}]
		\foreach\x  in {1,2,...,5}{
			\node[r] (s\x) at (\x,0.5){};
		}
		\foreach\x [evaluate=\x as \y using {int(\x+6)}] in {1,2,...,7}{
			\node[b] (h\x) at (\y,0.5){};
		}
		\draw[purple] (s1) to [bend left=30] (h1);
		\draw[purple] (s2) to [bend left=30] (h3);
		\draw[purple] (s3) to [bend left=30] (h2);
		\draw[red] (s4) to [bend left=30] (s5);
		\draw[blue] (h4) to [bend left=30] (h5);
		\draw[blue] (h6) to [bend left=30] (h7);
	\end{tikzpicture} &
\end{align*}
Um unser gewohntes Brauer-Diagramm zu erhalten, wenden wir außerdem den Isomorphismus $v\mapsto \braket{v,\cdot}$ zwischen $V$ und $V^\ast$ an und überführen die lineare Abbildung $V^{\otimes m} \to V^{\otimes n}$ in ein Element des Tensorraums $V^{\otimes m}\otimes V^{\otimes n}=V^{\otimes m+n}$:
\begin{align*}
	\begin{tikzpicture}[baseline={([yshift=-.5ex]current bounding box.center)}]
		\foreach\x in {1,2,...,5}{
			\node[v] (s\x) at (\x,1){};
		}
		\foreach\x in {1,2,...,7}{
			\node[v] (h\x) at (\x,0){};
		}
		\draw[] (s1) to (h1);
		\draw[] (s2) to (h3);
		\draw[] (s3) to (h2);
		\draw[] (s4) to [bend right=30] (s5);
		\draw[] (h4) to [bend left=30] (h5);
		\draw[] (h6) to [bend left=30] (h7);
	\end{tikzpicture} &
	\\ \\ \\
	\mapsto
	\begin{tikzpicture}[baseline={([yshift=-2ex]current bounding box.center)}]
		\foreach\x  in {1,2,...,5}{
			\node[v] (s\x) at (\x,0.5){};
		}
		\foreach\x [evaluate=\x as \y using {int(\x+6)}] in {1,2,...,7}{
			\node[v] (h\x) at (\y,0.5){};
		}
		\draw[] (s1) to [bend left=30] (h1);
		\draw[] (s2) to [bend left=30] (h3);
		\draw[] (s3) to [bend left=30] (h2);
		\draw[] (s4) to [bend left=30] (s5);
		\draw[] (h4) to [bend left=30] (h5);
		\draw[] (h6) to [bend left=30] (h7);
	\end{tikzpicture} &
\end{align*}

Es ist noch zu klären, dass tatsächlich keine Abbildung übersehen wurde, z.B. weil sie nicht als Brauer-Diagramm darstellbar ist.
Für diese Suche wenden wir wieder den Mittelwert-Trick an, diesmal mit einer kompakten Gruppe anstelle einer endlichen, d.h. wir integrieren anstelle zu summieren. Das Maß $|O_3|$ können wir wählen und setzen es auf $1$, unsere Projektion auf den Raum der Fixpunkte sieht also folgendermaßen aus:
\begin{equation}
	\Psi_{O_3}^{m,n} : \left\lbrace\begin{array}{rcl}
				V^{\otimes m+n} &\to& V^{\otimes m+n}
		\\
			v &\mapsto& \int_{O_3} \rho(v) \dd \rho
		\end{array}\right.
\end{equation}
Wir könnten nun allgemeine Tensoren auf ihre Fixpunkte projizieren. Es ist als Übung sinnvoll, zunächst ein paar Beispiele explizit auszurechnen (siehe Aufgabe \ref{aufg:TensorfixProj}).

\medbreak
Den vollständigen Beweis führen wir nun per Induktion über $d:=\dim(V)$.

Induktionsbehauptung: 
Das Mittelwertintegral über $O_d$ dargestellt auf $V^{\otimes n}$ ergibt immer eine Linearkombination der Permutation von Casimir-Elementen $\Omega_{V}$ (mglw. ist diese auch $0$), wobei wir die Permutation von Casimir-Elementen auch mit einem entsprechenden einzeiligen Brauer-Diagramm darstellen können.
 
Induktionsanfang:
Dazu bestimmen wir zunächst den Fixpunktraum von $O_1$ dargestellt auf $\IK$, d.h. wir berechnen
\[
\int_{O_1} \rho (v^{\otimes n}) \dd \rho, v\in\IR
\]
Wir haben also für $d=1$ festgestellt, dass der Fixpunktraum mit geraden Potenzen von Casimir-Elementen aufgespannt wird, für einen bestimmten Tensorgrad $2n$ ist der Fixpunktraum eindimensional und wird von $v^{\otimes 2n}=\Omega^{\otimes n}$ aufgespannt.

\medbreak
Induktionsschritt: Für $d+1$ nutzen wir folgenden Zusammenhang, $ v_i\in\IR^{d+1}$ :
\begin{align*}
\int_{O_{d+1}} \rho (v_1\otimes\cdots \otimes v_n) \dd \rho &= \int_{O_{d}} \int_{O_{d+1}} \tau(\rho (v_1\otimes\cdots \otimes v_n)) \dd \rho \dd \tau \\
&= \int_{O_{d+1}} \int_{O_{d}} \tau(\rho (v_1\otimes\cdots \otimes v_n)) \dd \rho \dd \tau &\text{Satz v. Fubini-Tonelli}\\
&= \int_{O_{d+1}} \int_{O_{d}} \rho (v_1\otimes\cdots \otimes v_n) \dd \rho \dd \tau & \text{Substitution }\rho\to\tau^{-1}\rho \\
&= \int_{O_{d+1}} \int_{O_{d}} \rho(\tau (v_1\otimes\cdots \otimes v_n)) \dd \tau \dd \rho & \text{Substitution }\rho\to\rho\tau 
\end{align*}
wobei wir $O_d$ in $O_{d+1}$ auf bestimmte Weise einbetten: Jede Einbettung von $O_d$ in $O_{d+1}$ wird eine ausgezeichnete Richtung haben, deren Unterraum von der Untergruppe nicht verändert wird. Wir wählen uns die Einbettung so, dass $v_1$ von $O_d$ nicht verändert wird und zerlegen alle $v_i$ in zu $v_1$ senkrechte und parallele Anteile,
\[
v_1\otimes\cdots \otimes v_n = v_1 \otimes (v_2^\parallel + v_2^\perp)\otimes \cdots \otimes (v_n^\parallel + v_n^\perp) .
\]
wobei $v_i^\parallel=c_i v_1$ jeweils ein Vielfaches von $v_1$ ist. Damit ist
\begin{align*}
\tau(v_1\otimes\cdots \otimes v_n) &= \tau(v_1) \otimes (\tau(v_2^\parallel) + \tau(v_2^\perp))\otimes \cdots \otimes (\tau(v_n^\parallel) + \tau(v_n^\perp)) \\
&= v_1 \otimes (c_2 v_1 + \tau(v_2^\perp))\otimes \cdots \otimes (c_n v_1 + \tau(v_n^\perp)) .
\end{align*}
Durch Ausmultiplizieren bekommen wir eine Summe von reinen Tensoren, in denen jeder Faktor parallel oder senkrecht zu $v_1$ ist. Für jeden reinen Tensor können wir uns eine Permutation aussuchen, die alle senkrechten Faktoren nach rechts schiebt und alle parallelen Faktoren nach links. Die Skalare $c_i$ können in einer Konstante zusammengefasst werden, welche wir zur Einfachheit auf 1 setzen. Rechnen wir also o.B.d.A. mit dem Tensor
\[
 v_1^{\otimes l} \otimes \tau(v_{l+1}^\perp) \otimes \cdots \otimes \tau(v_n^\perp),
\] 
so erhalten wir
\begin{align*}
	&\int_{O_{d+1}} \int_{O_{d}} \rho\left(v_1^{\otimes l} \otimes \tau(v_{l+1}^\perp) \otimes \cdots \otimes \tau(v_n^\perp)\right) \dd \tau \dd \rho \\
	= &\int_{O_{d+1}} \rho(v_1)^{\otimes l}\otimes \rho\left(\int_{O_{d}}\tau(v_{l+1}^\perp) \otimes \cdots \otimes \tau(v_n^\perp) \dd \tau \right) \dd \rho
\end{align*}
Das innere Integral ergibt nach Induktionsvoraussetzung eine Linearkombination von Permutationen von Casimir-Elementen $\Omega_{\IK^d}$. Jedes einzelne Casimir-Element können wir nun schreiben als $\Omega_{\IK^d} = \Omega_{\IK^{d+1}} - v_1 \otimes v_1$, wobei wir o.b.d.A. annehmen können, dass $v_1$ normiert ist. Diese Summe multiplizieren wir erneut aus, sodass wir Permutationen von Tensoren der Form
\begin{align*}
	v_1^{\otimes l+2k} \otimes \underbrace{\Omega_{\IK^{d+1}} \otimes \cdots \otimes \Omega_{\IK^{d+1}}}_{\frac{n-l-2k}{2} \text{viele}}
\end{align*}
erhalten. Integrieren wir diesen Tensor über $O_{d+1}$: 
\begin{align*}
	&\int_{O_{d+1}} \rho(v_1)^{\otimes l+2k} \otimes \rho\left(\Omega_{\IK^{d+1}}\right) \otimes \cdots \otimes \rho\left(\Omega_{\IK^{d+1}}\right)\dd \rho \\
	=& \int_{O_{d+1}} \rho(v_1)^{\otimes l+2k} \dd \rho \otimes \Omega_{\IK^{d+1}} \otimes \cdots \otimes \Omega_{\IK^{d+1}},
\end{align*}
und laut Aufgaben Aufgaben \ref{aufg:TensorfixProj}\ref{aufg:TensorfixProjgeradeu} und \ref{aufg:TensorfixProj}\ref{aufg:TensorfixProjungeradeu} ist auch das verbliebene Integral eine Linearkombination Permutationen von Casimir-Elementen.
\end{proof}


\begin{sheet}
\begin{problem}[title={Die 4 Klassen von linearen Tensor-Abbildungen sind natürlich}]
	\label{aufg:natAbbKlass}
	Zeige, dass die vier Klassen von linearen Abbildungen zwischen Tensoren tatsächlich natürliche Abbildungen sind:
	\begin{subproblem}
		Skalarmultiplikation (\ref{def:skalarmultTensor})
	\end{subproblem}
	\begin{subproblem}
		Permutation (\ref{def:permutation})
	\end{subproblem}
	\begin{subproblem}
		Spur nehmen (\ref{def:spur})
	\end{subproblem}
	\begin{subproblem}
		Casimir-Element einfügen (\ref{def:casimireinfuegen})
	\end{subproblem}
\end{problem}

\begin{problem}[title={Bunte Casimir-Elemente}]
	In der grafischen Darstellung der Überführung einer linearen Abbildung $V^{\otimes m} \to V^{\otimes n}$ zu einem Element des Tensorraums $\left(V^{\otimes m}\right)^\ast \otimes V^{\otimes n}$ haben wir Casimir-Elemente mit drei verschiedenen Farben gezeichnet. Überzeuge dich, dass diese drei Farben nicht nur der Übersichtlichkeit dienen, sondern auch die verschiedenen Arten von Casimir-Elementen deutlich machen. Wie ist es gerechtfertigt, dass die Permutation oder Identität ebenfalls auf ein Casimir-Element abgebildet wird? Wo versteckt sich die Permutation?
	
	Wir haben \emph{farbige Brauer-Diagramme} ohne Definition eingeführt, hole dies nach und finde eine sinnvolle Definition.
\end{problem}

\begin{problem}[title={Tensoren auf $O_3$-Fixpunkte projizieren}]
	\label{aufg:TensorfixProj}
	Projiziere die folgenden Tensoren auf ihre $O_3$ Fixpunkte, indem du das entsprechende Integral über $O_3$ löst:
	\begin{subproblem}
		1-Tensor $v\in\IR$.
	\end{subproblem}
	\begin{subproblem}
		$u\otimes u$ mit $u \in \IR$.
	\end{subproblem}
	\begin{subproblem}
		$u\otimes v$ mit $u\perp v \in \IR$.
	\end{subproblem}
	\begin{subproblem}
		$u\otimes (u+v)$ mit $u\perp v \in \IR$.
	\end{subproblem}
	\begin{subproblem}
		$u\otimes v \otimes w$ mit $u,v,w \in \IR$.
	\end{subproblem}
	\begin{subproblem}
		\label{aufg:TensorfixProjgeradeu}
		$u^{\otimes n}$ mit $u \in \IR$ und $n$ gerade. Hinweis: Polynome und symmetrische Tensoren.
	\end{subproblem}
	\begin{subproblem}
		\label{aufg:TensorfixProjungeradeu}
		$u^{\otimes n}$ mit $u \in \IR$ und $n$ ungerade.
	\end{subproblem}
\end{problem}
\end{sheet}


\pagebreak
\section{Darstellungstheorie -- Level 1}
% !TeX root = spherical_harmonics.tex
% !TeX spellcheck = de_DE
\begin{remark}
Wir wollen nun zwei Sätze beweisen und diskutieren, die die ersten beiden wirklich nichttrivialen Resultate und zugleich das Rückgrat der Darstellungstheorie bilden.

Der erste Satz ist das Lemma von Schur.
\end{remark}

\subsection{Lemma von Schur}
\begin{theorem}[Lemma von Schur\footnote{Issai Schur (1875--1941), russisch-deutsch-israelischer Mathematiker}]\label{darstellungen:schur}
Sei $T$ eine endlich-dimensionale, irreduzible Darstellung von $G$ über dem Körper $\IK$.
\begin{enumerate}
\item Jede $\IK G$-lineare Abbildung $T\to S$ ist entweder Null oder injektiv.
\item Jede $\IK G$-lineare Abbildung $U\to T$ ist entweder Null oder surjektiv.
\item Ist $T'$ eine weitere irreduzible Darstellung von $G$, so unterscheiden sich alle von Null verschiedenen Abbildungen $T\to T'$ (von denen es keine geben muss) um einen Isomorphismus $T\to T$.
\item Ist $\IK=\IC$, so ist $\End_{\IK G}(T) = \IC\id_T := \set{\lambda\id_T | \lambda\in\IC}$. Insbesondere: In der Situation von c. gibt es also bis auf konstante Vielfache genau eine $\IC G$-lineare Abbildung $T\to T'$.
\end{enumerate}
\end{theorem}

\begin{remark}
Man beachte die starke Einschränkung, die die Struktur einer $G$-Darstellung bewirkt: Während es $\IK$-lineare Abbildungen wie Sand am Meer gibt und insbesondere $\End_\IK(T)$ streng monoton mit der Dimension von $T$ anwächst, hat $\End_{\IK G}(T)$ extrem wenige, für $\IK=\IC$ sogar nu einen Freiheitsgrad, solange $T$ irreduzibel ist.

Das ist letztlich die Erklärung dafür, wieso wir relativ wenige \enquote{natürliche} Abbildungen zwischen Tensorräumen gefunden haben.
\end{remark}

\begin{proof}
a. folgt aus der Beobachtung, dass $\ker(f)$ eine Unterdarstellung von $T$ ist. b. folgt daraus, dass $\im(f)$ eine Unterdarstellung von $T$ ist. Da $T$ irreduzibel ist, gibt es also nur zwei Möglichkeiten: Null oder ganz $T$.

\smallbreak
c. Sind $f, g: T\to T'$ zwei von Null verschiedene Homomorphismen, dann folgt aus a. und b., dass sie bijektiv sind. Entsprechend ist $\tau:=f^{-1}\circ g$ ein Isomorphismus $T\to T$, d.h. $g=f\circ\tau$ unterscheidet sich von $f$ durch Multiplikation mit einem Isomorphismus.

\smallbreak
d. Ist nun $f\in\End_{\IC G}(T)$ beliebig, so gibt es einen Eigenwert $\lambda\in\IC$ von $f$. Hier benutzen wir, dass $T$ ein endlichdimensionaler $\IC$-Vektorraum ist. Also ist $f-\lambda\id_T$ ein Endomorphismus, der einen nichttrivialen Kern hat. Der Kern ist, wie wir bereits festgestellt haben, eine Unterdarstellung von $T$. Die einzige nichttriviale Unterdarstellung ist $T$ selbst. Also muss $f-\lambda\id = 0$ sein, d.h. $f$ ist ein Vielfaches der Identität.
\end{proof}

\begin{remark}
In d. ist es ganz wesentlich, dass wir mit dem Körper der komplexen Zahlen arbeiten. Es ist in der allgemeinen Situation immer noch korrekt, dass $E:=\End_{\IK G}(V)$ ausschließlich aus Null und invertierbaren Abbildungen besteht, dass $E$ also ein Schiefkörper ist (d.h. so wie ein Körper, nur dass Kommutativität der Multiplikation nicht gelten muss), aber ohne die Eigenschaft, algebraisch abgeschlossen zu sein, kann es (und wird es) Darstellungen geben, für die $E$ nicht $\IK$ selbst, sondern ein größerer Schiefkörper, ist.

\medbreak
Beispielsweise gibt es über $\IK=\IR$ drei endlich-dimensionale Schiefkörper: $\IR$ selbst, $\IC$ sowie die Quaternionen $\mathbb{H}$. Entsprechend finden sich auch Gruppen $G$, die irreduzible Darstellungen $T$ mit $\dim_\IR \End_{\IR G}(T)=2$ oder $=4$ besitzen.

Die Gruppen $G=O_3$ und $G=O_2$, die uns vor allem interessieren, haben interessanterweise keine solche Darstellungen. Die Gruppe der zweidimensionalen Drehungen $G=SO_2$ zeigt aber tatsächlich dieses Phänomen; sogar haben \emph{alle} bis auf eine irreduzible reelle Darstellung von $SO_2$ einen zweidimensionalen Endomorphismenraum.
\end{remark}

\subsection{Satz von Maschke und Test auf Irreduzibilität}
\begin{theorem}[Satz von Maschke]\label{darstellungen:maschke}
Ist $G$ eine kompakte Gruppe und $T$ eine stetige, endlich-dimensionale Darstellung von $G$, dann ist $T$ \emph{halbeinfach}, d.h. $T$ ist die direkte Summe von einigen seiner einfachen Unterdarstellungen:
\begin{enumerate}
\item Es gibt irreduzible Unterdarstellungen $T_i$ mit $T=T_1\oplus T_2\oplus\cdots\oplus T_k$.
\end{enumerate}

Speziell für unseren Fall $\IK\in\set{\IR,\IC}$ ist es sinnvoll, von Skalarprodukten auf $T$ zu sprechen. Dann gelten folgende zusätzliche Aussagen:
\begin{enumerate}[resume]
\item Es gibt ein \emph{$G$-invariantes} Skalarprodukt, d.h. eines, dass $\forall g\in G, v,w\in T: \braket{gv,gw} = \braket{v,w}$ erfüllt. Äquivalent formuliert: Die Darstellung $\mathcal{D}: G\to GL(T)$ ist in Wirklichkeit eine Darstellung $G\to O(T)$ (bzw. $U(T)$ im komplexen Fall).
\item Ist ein $G$-invariante Skalarprodukt gegeben, kann man die Zerlegung in a. sogar so wählen, dass die Summanden paarweise senkrecht aufeinander stehen. Unterdarstellungen, die nicht zueinander isomorph sind, sind sogar bzgl. \emph{jedem} $G$-invarianten Skalarprodukt senkrecht aufeinander.
\end{enumerate}
\end{theorem}
\begin{proof}
a. gilt allgemein für alle Körper $\IK$, aber wir zeigen nur den reellen/komplexen Fall. Dazu beginnen wir mit b.: Zunächst gibt es überhaupt ein Skalarprodukt $[,]: T\times T\to\IK$. Mit Hilfe des Mittelwerttricks können wir daraus ein $G$-invariantes Skalarprodukt zaubern:
\[\braket{v,w} := \int_G [gv,gw] \dd g\]
Für jedes feste $g$ ist $(v,w) \mapsto [gv,gw]$ selbst ein Skalarprodukt. Der Mittelwert $\braket{,}$ ist also auch ein Skalarprodukt, weil es ein Grenzwert von (reellen, nicht-negativen) Linearkombinationen von Skalarprodukten ist. Nach Konstruktion ist es auch $G$-invariant.

\smallbreak
Daraus folgt nun a., denn wenn wir eine Unterdarstellung $U\leq T$ haben, dann ist das orthogonale Komplement (bzgl. $\braket{,}$) auch eine Unterdarstellung von $T$ und wir haben eine Zerlegung $T=U\oplus U^\perp$. Jetzt zerlegen wir $U$ und $U^\perp$ so lange weiter, bis wir nicht weiter zerlegen können (es gibt ein Ende, weil $T$ endlich-dimensional ist), weil es keine nichttrivialen, echten Untermoduln mehr gibt, d.h. bis wir bei irreduziblen Summanden angekommen sind.

\smallbreak
c. folgt aus dem Lemma von Schur: Wenn ein beliebiges $G$-invariantes Skalarprodukt und $U,U'\leq T$ zwei irreduzible Unterdarstellungen sind, dann können wir die orthogonale Projektion bzgl. $\braket{,}$ auf $U$ betrachten und auf $U'$ einschränken. Damit erhalten wir eine $\IK G$-lineare Abbildung $U'\to U$. Wenn $U$ und $U'$ nicht isomorph sind, dann kann es keine bijektive $\IK G$-lineare Abbildung zwischen ihnen geben und laut dem Lemma von Schur ist die einzige andere Möglichkeit die Nullabbldung, d.h. $U$ und $U'$ sind senkrecht bzgl. $\braket{,}$.
\end{proof}

\begin{remark}
Der Satz von Maschke ist falsch für beliebige Gruppen und beliebige Darstellungen. Selbst sehr einfache, nichtkompakte Gruppen erfüllen ihn nicht, selbst für sehr einfach gestrickte Darstellungen. Schon $G=\IZ$ ist ein Gegenbeispiel.

Und selbst, wenn die Gruppe kompakt ist, müssen unstetige oder unendlich-dimensionale Darstellungen ihn nicht erfüllen.
\end{remark}

\begin{corollary}[Umkehrung des Satzes von Schur]\label{darstellungen:schur_umkehrung}
Ist $T$ eine endlich-dimensionale, stetige Darstellung der kompakten Gruppe $G$ über $\IK\in\set{\IR,\IC}$, dann gilt: $\dim_\IK \End_{\IK G}(T) = 1 \implies T$ ist irreduzibel.
\end{corollary}
\begin{proof}
Die Voraussetzungen erlauben es uns, auf $T$ den Satz von Maschke anzuwenden, d.h. $T=T_1\oplus T_2\oplus ...$ zu zerlegen. Wenn $T$ nicht selbst irreduzibel wäre, dann gäbe es mindestens zwei solche Summanden und die beiden Projektionen $(v_1,v_2, ...) \mapsto (v_1,0,\ldots)$ bzw. $(v_1,v_2,\ldots)\mapsto (0,v_2,\ldots)$ wären zwei lineare unabhängige $\IK G$-lineare Abbildungen $T\to T$ und somit ein Widerspruch zur Annahme $\dim_\IK \End_{\IK G}(T)=1$.
\end{proof}

\begin{remark}
Der Satz von Maschke sagt uns, dass wir jede Darstellung bis auf Isomorphie festlegen können, indem wir angeben, wie oft welche irreduzible Darstellung als direkter Summand in $T$ vorkommt. Ein fundamentales Problem des Studiums einer (oder aller) Darstellungen ist deshalb erstens die Frage, welche irreduziblen Darstellungen die jeweilige Gruppe überhaupt besitzt. Je nach Gruppe ist das ein mehr oder weniger schwieriges Problem. Und zweitens stellt sich die Frage, wie man die Vielfachheiten der irreduziblen Summanden in einer gegebenen Darstellung bestimmt.

Beide Fragen stellen sich als erstaunlich zugänglich heraus, wenn man Charaktertheorie kennt.
\end{remark}
% !TeX root = spherical_harmonics.tex
% !TeX spellcheck = de_DE
\subsection{Einige irreduzible Darstellungen von Isometriegruppen}

\begin{remark}
Wir sind immer noch daran interessiert, die Darstellungen von $O_2$ und $O_3$ zu verstehen, und, wie wir eben gelernt haben, ist ein Wesentlicher Teil dieses Vorhabens die Aufgabe, die irreduziblen Darstellungen dieser Gruppen zu finden.

In Aufgabe \ref{ex:fundamentaldarstellung_von_so3} haben wir gelernt, dass die kanonische Darstellung von $O_3$ irreduzibel ist. Wir wissen, dass jede Gruppe eine triviale Darstellung hat, die auch immer irreduzibel ist. Aber welche gibt es noch? In Aufgabe \ref{ex:keine_2D_darstellung_von_so3} wurde gezeigt, dass es keine ein- oder zwei-dimensionalen Kandidaten gibt.
\end{remark}

\begin{definition}[Spurfreie Tensoren]
Die Tensoren im Kern der Spur $\tr_{i,j} : Sym^m(V) \to Sym^{m-2}(V)$ werden \emph{spurfreie} Tensoren genannt. Der Raum der symmetrischen, spurfreien Tensoren bezeichnen wir mit $STF^m$.
\end{definition}

\begin{lemma}
Die spurfreien symmetrischen Tensoren sind eine $O_n$-Unterdarstellung von $V^{\otimes m}$.
\end{lemma}

\begin{lemma}
Der Isomorphismus zwischen symmetrischen Tensoren und polynomiellen Abbildungen übersetzt...
\begin{enumerate}
\item ...die Spur in den Laplace-Operator
\item ...die spurfreien symmetrischen Tensoren in harmonische Polynome.
\end{enumerate}
\end{lemma}



\begin{theorem}[Zerlegung der symmetrischen Tensoren in irreduzible Unterdarstellungen]
\begin{enumerate}
\item Die Darstellungen $STF^m(V)$ sind paarweise nicht isomorph und irreduzibel.
\item Mit der Projektion $q$ auf die symmetrischen Tensoren gilt:
\[Sym^m(V) \isomorphic STF^m(V) \oplus STF^{m-2}(V) \oplus STF^{m-4}(V) \oplus \cdots\]
\end{enumerate}
\end{theorem}
\begin{proof}
Die Spur ist eine $O_n$-lineare Abbildung $\tr: Sym^m(V) \to Sym^{m-2}(V)$, deren Kern per Definition $STF^m$ ist. Das orthogonale Komplement des Kerns ist eine $O_n$-Darstellung, die zum Bild $\im(\tr)=Sym^{m-2}(V)$ isomorph ist (die Spur ist surjektiv laut \todo{Aufgabe definieren}). Dies benutzen wir, um einen induktiven Beweis aufzuziehen.

Zum einen folgt daraus, dass $\dim(STF^m) = \dim(Sym^m(V)) - \dim(Sym^{m-2}(V)) = \binom{m+n+1}{n-1} - \binom{m+n-1}{n-1}$ ist, was streng monoton mit $m$ wächst. Also sind diese Darstellungen auf jeden Fall paarweise nicht isomorph. Die Frage ist, wieso sie irreduzibel sind und wieso sich $Sym^m$ so zerlegt.

\medbreak
Wir beweisen beide Behauptungen mit einer gemeinsamen Induktion. Für $m\leq 1$ ist nicht zu zeigen: $STF^0 = Sym^0(V) = V^{\otimes 0} = \IK$ ist die triviale Darstellung, also irreduzibel. $STF^1 = Sym^1(V) = V$ ist uns auch bereits als irreduzibel bekannt. Wir nehmen also an, wir wüssten bereits, dass alle $STF^k$ mit $k<m$ irreduzibel sind.

Also enthält dieses orthogonale Komplement Unterdarstellungen der Form $Sym^k(V)$ mit $k=m-2, m-4, ...$, jeweils genau einmal.





\end{proof}
% !TeX root = spherical_harmonics.tex
% !TeX spellcheck = de_DE
\begin{sheet}

\begin{problem}[title={Satz von Schur über Körpern $\neq\IC$}]
In \ref{ex:fundamentaldarstellung_von_so2} wurde gezeigt, dass die kanonische Darstellung $V=\IR^2$ von $G=SO_2$ irreduzibel ist. Zeige, dass $\End_{\IR G}(V)$ 2-dimensional ist.
\end{problem}

\begin{problem}[title={Der Satz von Maschke ist falsch für nichtkompakte Gruppen}, difficulty={leichter als es aussieht}]
Zeige, dass nicht jede endlich-dimensionale Darstellung von $G=\IZ$ in irreduzible Unterdarstellungen zerlegt werden kann.

Hinweis: Das ist eine leicht getarnte Aussage, die bereits aus der LA-I-Vorlesung bekannt ist.
\end{problem}


\begin{problem}
Es sei $V = V_1^{\oplus m_1} \oplus V_2^{m_2} \oplus \cdots \oplus V_k^{m_k}$ eine Darstellung von $G$, wobei die $V_i$ paarweise nicht-isomorphe, irreduzible Unterdarstellungen seien. Zeige, dass
\[\dim \End_{\IK G}(V) = m_1^2 + m_2^2 + \ldots + m_k^2\]
gilt.
\end{problem}

\begin{problem}\label{ex:keine_2D_darstellung_von_so3}
Zeige, dass die einzigen reellen oder komplexen Darstellungen von $G=SO_3$ von Dimension $<3$ trivial sind.

Hinweis: Verwende, dass laut Satz von Maschke die Darstellung $\Delta: G\to SO(V)$ gewählt werden kann.
\end{problem}

\end{sheet}

\pagebreak
\section{Darstellungstheorie -- Level 2: Charaktertheorie}
% !TeX root = spherical_harmonics.tex
% !TeX spellcheck = de_DE
\begin{definition}[Charaktere]
Es sei $G$ eine Gruppe und $\mathcal{D}: G\to GL(T)$ eine endlichdimensionale Darstellung von $G$ über $\IK$. Der \emph{Charakter von $T$} ist die Abbildung
\[\chi_T: G\to \IK, g\mapsto \tr(\mathcal{D}(g))\]
Ist $D: G\to GL_n(\IK)$ eine Matrixdarstellung von $\mathcal{D}$ bzgl. irgendeiner Basis, so ist $\chi_T(g)$ also auch gleich der Spur der Matrix $D(g)$.

Ist $T$ eine irreduzible Darstellung, so nennt man $\chi_T$ \emph{irreduziblen Charakter}.
\end{definition}

\begin{remark}
Man beachte, dass die Spur einer Matrix unabhängig von der Basiswahl ist. Man erinnere sich, dass allgemein
\[\forall A,B\in \IK^{n\times n}: \tr(AB) = \tr(BA)\]
gilt, für invertierbare $A$ also auch $\tr(ABA^{-1}) = \tr(B)$.

Oder basisfrei formuliert: Für alle $\IK$-linearen Abbildungen $\phi: X \leftrightarrows Y : \psi$ zwischen endlichdimensionalen $\IK$-Vektorräumen gilt $\tr(\phi\circ\psi) = \tr(\psi\circ\phi)$, für invertierbare $\phi$ also auch $\tr(\phi\circ\psi\circ\phi^{-1}) = \tr(\psi)$.
\end{remark}

\begin{lemma}
Charaktere erfüllen:
\begin{enumerate}
\item Charaktere unterscheiden nicht zwischen isomorphen Darstellungen: $T\isomorphic T' \implies \chi_T = \chi_{T'}$.
\item Charaktere unterscheiden nicht zwischen konjugierten Elementen: Sind $g,g'\in G$ konjugiert, d.h. existiert ein $h\in G$ mit $g'=hgh^{-1}$, so gilt $\chi_T(g)=\chi_T(g')$.
\end{enumerate}
\end{lemma}
\begin{proof}
Beide Aussagen folgen aus dieser Eigenschaft der Spur: Ist $\phi: T\to T'$ ein Isomorphismus, dann ist $\mathcal{D}'(g) = \phi\circ\mathcal{D}(g)\circ\phi^{-1}$ und somit $\tr(\mathcal{D}(g)) = \tr(\mathcal{D}'(g))$. Sind $g,g'$ konjugiert, so ist $\mathcal{D}(g') = \mathcal{D}(h)\mathcal{D}(g)\mathcal{D}(h)^{-1}$ und somit $\tr(\mathcal{D}(g)) = \tr(\mathcal{D}(g'))$.
\end{proof}

\begin{remark}
Es scheint, als ob Charaktere viel Information über ihre Darstellung verlieren würden, wenn sie die ganze Abbildung $\mathcal{D}(g)$ auf eine einzelne Zahl reduzieren und dann auch noch sehr viele Gruppenelemente dieselbe Zahl zugeordnet bekommen. Im allgemeinsten Fall ist das auch so.

Das große Wunder ist aber, dass im für uns interessanten Fall einer kompakten Gruppe und stetiger Darstellungen, nicht zu viel Information verloren geht, sondern Charaktere immer noch hinreichend viel enthalten, um ihre Darstellung vollständig bis auf Isomorphie zu beschreiben, d.h. in unserem Fall gilt die Umkehrung von a.
\end{remark}

\begin{theorem}[1.Orthogonalitätsrelation]
Es sei $G$ eine kompakte Gruppe und $\IK\in\set{\IR,\IC}$. Auf der Menge der stetigen Abbildungen $G\to\IC$ definieren wir die folgende Bilinearform:
\[\braket{\phi,\psi} = \int_G \phi(g)\psi(g^{-1}) \dd g\]
Speziell für Charaktere von stetigen Darstellungen gilt:
\[\braket{\chi_T, \chi_{T'}} = \dim_\IK \Hom_{\IK G}(T,T')\]
Insbesondere gilt für irreduzible Charaktere:
\[\braket{\chi_T, \chi_{T'}} = \begin{cases} 0 & \text{falls }T\not\isomorphic T' \\ \dim_\IK \End_{\IK G}(T) & \text{falls }T\isomorphic T'\end{cases}\]
\end{theorem}
\begin{proof}
Wir betrachten die beiden Darstellungen $\mathcal{D}$ und $\mathcal{D}'$ von $T$ bzw. $T'$ sowie die Darstellung $X:=\Hom_\IK(T,T')$. Zur Erinnerung: $G$ operiert auf $X$ via $g\phi := \mathcal{D}(g)\circ\phi\circ \mathcal{D}'(g)^{-1}$, d.h. $g\phi$ ist die Abbildung $t\mapsto g\cdot\phi(g^{-1}\cdot t)$. Anhand dieser Definition ist klar, dass der Raum der $G$-Fixpunkte von $X$ genau der Raum der $\IK G$-linearen Abbildungen ist.

\smallbreak
Wir wenden also wieder einmal den Mittelwerttrick an: Die Projektion auf den Fixpunktraum ist gegeben durch
\[pr(\phi) := \int_G g\circ\phi\circ g^{-1} \dd g\]
Wenn wir nun die Dimension des Fixpunktraums bestimmen wollen, können wir die Spur der Projektion ausrechnen. Weil (konstante) $\IK$-lineare Abbildungen mit dem Integral vertauschen, erhalten wir also
\[\dim_\IK \Hom_{\IK G}(T,T') = \tr(pr) = \tr\left(\phi \mapsto \int_G g\circ\phi\circ g^{-1} \dd g \right) = \int_G \tr(\phi \mapsto g\circ \phi\circ g^{-1}) \dd g\]
Es ist eine Standardaufgabe aus der linearen Algebra, zu beweisen, dass die Abbildung $\Hom_\IK(T,T') \to \Hom_\IK(T,T'), \phi \mapsto \alpha\circ\phi\circ\beta$ genau die Spur $\tr(\alpha)\tr(\beta)$ hat. Also ist der Integrand gleich $\chi_T(g)\chi_{T'}(g^{-1})$ und das zeigt die Behauptung.
\end{proof}

\begin{remark}[Diverse Charakterformeln]
Für sehr viele Standardkonstruktionen ist es möglich, explizit anzugeben, wie sie sich mit Charakteren vertragen:
\begin{enumerate}
\item $\chi_{T\oplus T'} = \chi_T + \chi_{T'}$
\item $\chi_{T\otimes T'} = \chi_T \cdot \chi_{T'}$
\item $\chi_{T^\ast} = \overline{\chi_T}$
\item Es gibt explizite, aber relativ aufwändige kombinatorische Formeln, die z.B. die Charaktere der symmetrischen Potenzen $\Sym^m(T)$ durch $\chi_T$ ausdrücken.
\item uvm.
\end{enumerate}
\end{remark}

\begin{corollary}
Ist $G$ eine kompakte Gruppe und $\IK\in\set{\IR,\IC}$, so gilt für stetige, endlich-dimensionale Darstellungen:
\[\chi_T = \chi_{T'} \iff T \isomorphic T'\]
\end{corollary}
\begin{proof}
Die $\impliedby$ Richtung ist bekannt. Wir zeigen also die andere. Aufgrund des Satzes von Maschke genügt es, zu zeigen, dass $T$ und $T'$ isomorphe irreduzible Darstellungen in den gleichen Vielfachheiten enthalten.

Wenn also $T \isomorphic T_1 ^{\oplus n_1} \oplus \cdots T_k^{\oplus n_k}$ und $T' \isomorphic T_1^{\oplus n_1'} \oplus \cdots \oplus T_k^{n_k'}$ mit paarweise nicht-isomorphen, irreduziblen Darstellungen $T_i$ ist, dann wollen wir $n_i=n_i'$ zeigen. Die Charaktere erfüllen
\[\chi_T = n_1 \chi_{T_1} + \ldots + n_k\chi_{T_k} \quad\text{bzw.}\quad \chi_{T'} = n_1' \chi_{T_1} + \ldots + n_k' \chi_{T_k}\] 
Die Orthogonalitätsrelation zeigt, dass die $\chi_{T_i}$ paarweise orthogonal bzgl. der dort definierten Bilinearform sind. Also gilt
\[n_i = \frac{\braket{\chi_T, \chi_{T_i}}}{\dim_\IK \End_{\IK G}(T_i)} = \frac{\braket{\chi_{T'}, \chi_{T_i}}}{\dim_\IK \End_{\IK G}(T_i)} = n_i'\]
und das wollten wir zeigen.
\end{proof}

\pagebreak
\section{Darstellungstheorie \texorpdfstring{von $O_1$}{der Isometriegruppe der reellen Geraden}}
% !TeX root = spherical_harmonics.tex
% !TeX spellcheck = de_DE
%\begin{sheet}

\setcounter{sheet}{\thesection}
\setcounter{problem}{0}

\begin{problem}[title={Darstellungstheorie von $O_1$}]
Beweise alles, was man über Darstellungstheorie der Gruppe mit zwei Elementen $G=O_1=\set{\pm 1}$ über $\IK\in\set{\IR,\IC}$ wissen muss:

\begin{subproblem}
Es gibt genau zwei irreduzible Darstellungen, beide eindimensional, nämlich die triviale $G\to\IK, g\mapsto 1$ und die nichttriviale $G\to\IK, \pm 1\mapsto \pm 1$.
\end{subproblem}
\begin{subproblem}
Man kann jede Darstellung von $G$ über $\IK$ (selbst unendlich-dimensionale) explizit in irreduzible Bestandteile zerlegen, indem man eine Eigenbasis von $\mathcal{D}(-1)$ findet. Gib eine explizite, basisfreie Formel für die Projektionen auf die beiden Eigenräume an.
\end{subproblem}

Hinweis: Eine Darstellung von $G$ ist der Raum der Funktionen $\IR \to \IC$, auf der $-1$ durch $f\mapsto \check{f}$ operiert, wobei $\check{f}(x) := f(-x)$ ist. Die beiden Eigenräume sind dir als \enquote{gerade} bzw. \enquote{ungerade} Funktionen bereits bekannt. Eine andere Darstellung ist der Raum der quadratischen Matrizen $\IK^{k\times k}$, auf dem $-1$ durch $A\mapsto A^T$ operiert. Die beiden Eigenräume sind dir als symmetrische bzw. antisymmetrische Matrizen bekannt. Lasse dich von diesen beiden Beispielen inspirieren.
\end{problem}
%\end{sheet}

\pagebreak
\section{Darstellungstheorie \texorpdfstring{von $O_2$}{der Isometriegruppe der Ebene}}
% !TeX root = spherical_harmonics.tex
% !TeX spellcheck = de_DE
\begin{remark}
Ein kurioser Fakt über $O_2$ und $SO_2$ ist, dass man diese Gruppe durch Matrizen basisfrei beschreiben kann:
\[O_2 = \underbrace{\Set{\begin{pmatrix} \cos(\alpha) & -\sin(\alpha) \\ \sin(\alpha) & \cos(\alpha)\end{pmatrix} | \alpha\in\IR}}_{=SO_2} \cup \Set{\begin{pmatrix} \cos(\alpha) & \sin(\alpha) \\ \sin(\alpha) & -\cos(\alpha)\end{pmatrix} | \alpha\in\IR}\]
Etwas genauer: Eine ebene Drehung hat bzgl. \emph{jeder} Orthonormalbasis der Ebene eine Matrix der Form $\rho_\alpha=\begin{psmallmatrix} \cos(\alpha)&-\sin(\alpha)\\\sin(\alpha)&\cos(\alpha)\end{psmallmatrix}$; es kann sich der Winkel $\alpha$ in $-\alpha$ ändern, wenn der Basiswechsel nicht orientierungserhaltend ist, aber das war's auch schon.

Eine Spiegelung ist immer von der Form $\begin{psmallmatrix}\cos(\alpha)&\sin(\alpha)\\\sin(\alpha)&-\cos(\alpha)\end{psmallmatrix}$; dabei ist der Winkel $\alpha$ aber von der Basiswahl abhängig, denn in diesem Fall legt $\alpha$ fest, wie die Gerade, an der gespiegelt wird, im Vergleich zur Basis orientiert ist. (Man beachte aber, dass für uns wichtige Informationen wie die Spur der Matrix unabhängig von $\alpha$ sind)
\end{remark}

\begin{theorem}[Darstellungstheorie von kommutativen Gruppen]
Es sei $G$ eine kompakte, \emph{kommutative} Gruppe. Dann ist jede stetige, irreduzible, endlich-dimensionale Darstellung über $\IC$ eindimensional.
\end{theorem}
\begin{proof}
Im Satz von Maschke \ref{darstellungen:maschke} haben wir gezeigt, dass man die Darstellung $\mathcal{D} : G \to GL(T)$ so wählen kann, dass sie ein Skalarprodukt invariant lässt, also jedes $\mathcal{D}(g)$ eine \emph{unitäre} Abbildung $T\to T$ ist.

Unitäre Abbildungen sind diagonalisierbar (siehe Aufgabe~\ref{ex:normale_matrizen}.\ref{ex:unitaere_matrizen_diagonalisierbar}). Da die Gruppe $G$ kommutativ ist, kommutieren die Abbildungen $\mathcal{D}(g)$ natürlich auch miteinander. Es ist ein Fakt der linearen Algebra (und keine schwere Übungsaufgabe, siehe Aufgabe~\ref{ex:simultan_diagonalisieren}), dass man kommutierende, diagonalisierbare Abbildungen auch \emph{simultan} diagonalisieren kann, d.h. man kann eine Basis $b_1, \ldots, b_n$ von $T$ aus \emph{gemeinsamen} Eigenvektoren finden, sodass also
\[\forall g: g\cdot b_i = \lambda_i(g) b_i\]
gilt für geeignete Funktionen $\lambda_i: G\to\IC$. Insbesondere sind die eindimensionalen Unterräume $\IC b_i$ Unterdarstellungen von $T$. Alle mindestens zwei-dimensionalen Darstellungen sind mit anderen Worten reduzibel.
\end{proof}

\begin{remark}
Es gilt auch die Umkehrung: Besitzt eine kompakte Gruppe nur eindimensionale irreduzible Darstellungen, dann ist sie kommutativ.
\end{remark}

\begin{remark}
Fourieranalyse basiert wesentlich auf dieser Aussage. Die verschiedenen Formen der Fourieranalyse haben alle das gleiche zugrunde liegende Prinzip: Die Zerlegung von (meist unendlich-dimensionalen) Darstellungen einer kommutativen Gruppe in ihre eindimensionalen, irreduziblen Bestandteile. Die geschieht, indem man eine Basis so wählt, dass aus diesen eindimensionalen Untervektorräumen je ein Vektor Teil der Basis ist. Der Unterschied zwischen den verschiedenen Arten von Fouriertransformation ist, welches die dahinter stehende Gruppe ist:

\begin{itemize}
\item Die diskrete Fouriertransformation (inkl. FFT) ist die Darstellungstheorie der zyklischen Gruppen $C_n = \set{\exp(\frac{2\pi k i}{n}) | k=0,\ldots,n-1}$.
\item Die Fouriertransformation periodischer Funktionen (=Fourierreihe) ist die Darstellungstheorie der Kreisgruppe $S^1 = \set{\exp(i\alpha) | \alpha\in[0,2\pi)}$ (welche zu $SO_2$ isomorph ist)
\item Die kontinuierliche Fouriertransformation ist die Darstellungstheorie der reellen Gerade $\IR$. ($\IR$ ist nicht kompakt, aber lokalkompakt. Für solche Gruppen gibt es auch ein Haar-Integral)
\end{itemize}

Für eine beliebige, (lokal)kompakte kommutative Gruppe spricht man auch von Pontrjagin-Dualität nach $\El$. $\Ssae$. $\Pae\ooo\en\tae\rae\ja\gae\iii\en$
\footnote{Lev S. Pontrjagin, $\El\jae\wae$ $\Ssae\jae\emm\jo\en\ooo\wae\iii\tschae$ $\Pae\ooo\en\tae\rae\ja\gae\iii\en$ (1908-1988), russ. Mathematiker}\footnote{Wir nennen ja keine Namen, aber Andrea bestand darauf, dass man unbedingt das kyrillische Original bräuchte, um den Namen korrekt aussprechen zu können.}.
\end{remark}

\begin{theorem}[Darstellungstheorie von $SO_2$ über $\IK=\IC$]
Jede stetige, endlich-dimensionale, irreduzible Darstellung von $G=SO_2$ über $\IC$ ist zu genau einer der Darstellungen
\[\mathcal{D}_k : \begin{psmallmatrix}\cos(\alpha)&-\sin(\alpha)\\\sin(\alpha)&\cos(\alpha)\end{psmallmatrix} \mapsto e^{ik\alpha}\]
für eine ganze Zahl $k$ isomorph und diese Darstellungen sind paarweise nicht-isomorph.
\end{theorem}
\begin{proof}
Dass die Darstellungen irreduzibel sind, ist klar, da sie eindimensional sind. Dass sie stetig sind, sieht man ihnen an. Dass sie paarweise nicht isomorph sind, kann man direkt überprüfen, oder man benutzt abstrakt, dass ihre Charaktere verschieden sind.

Der nichttriviale Teil ist also zu zeigen, dass jede endlich-dimensionale, stetige, irreduzible Darstellung in dieser Liste vorkommt. $SO_2$ ist kommutativ, also muss jede irreduzible Darstellung ein-dimensional sein, d.h. es handelt sich um eine stetige Abbildung $\mathcal{D}: G\to\IC$. Betrachte die Rotationen um $\alpha:=\frac{2\pi}{n}$. Dies sind Gruppenelemente mit $\rho_\alpha^n=1$. Also muss auch $\mathcal{D}(\rho_\alpha)^n=1$ gelten, d.h. $\mathcal{D}(\rho_\alpha) = \exp(\frac{2\pi i k}{n})$ für ein geeignetes $k\in\IZ$. Ohne Stetigkeit könnte dieses $k$ von $n$ abhängig sein. Wir zeigen, dass es unabhängig von $n$ ist, wenn $\mathcal{D}$ stetig ist.

Es gilt natürlich auch $\mathcal{D}(\rho_\alpha^m) = (\exp(\frac{2\pi i k}{n}))^m = \exp(\frac{2\pi i\cdot  km}{n})$, d.h. die $n$ Elemente $1=\rho_\alpha^0, \rho_\alpha^1, \ldots, \rho_\alpha^{n-1}$ erfüllen die Gleichung, die wir haben wollen, bereits. Welches $k$ erhalten wir, wenn wir von $n$ zu $2n$ wechseln? Die Drehung um $\alpha/2$ erfüllt $\rho_{\alpha/2}^2=\rho_\alpha$, d.h. $\mathcal{D}(\rho_{\alpha/2})$ ist eine der beiden Quadratwurzeln von $\mathcal{D}(\rho_\alpha)=\exp(\frac{2\pi i k}{n})$, also entweder $\exp(\frac{2\pi i k}{2n})$ oder $\exp(\frac{2\pi i(k+n)}{2n})$. Wenn $n$ hinreichend groß ist, sind $\rho_\alpha$ und $\rho_{\alpha/2}$ aber nah beieinander, und die Stetigkeitsbedingung verlangt, dass auch $\mathcal{D}(\rho_\alpha)$ und $\mathcal{D}(\rho_{\alpha/2})$ beide nah beieinander sind, was $k+n$ ausschließt, weil das auf der entgegengesetzten Seite des Einheitskreises liegt.

Per Induktion erfüllen also alle Drehungen um $\frac{2\pi}{2^l n}$ die Gleichung, die wir haben wollen. Jede Drehung kann durch diese beliebig genau approximiert werden. Aus der Stetigkeit folgt also, dass alle Drehungen die Gleichung erfüllen.
\end{proof}

\begin{remark}
Wenn man den komplexen Fall einmal hat, kann man den reellen Fall daraus folgern.
\end{remark}

\begin{corollary}[Darstellungstheorie von $SO_2$ über $\IK=\IR$]
Jede stetige, endlich-dimensionale, irreduzible Darstellung von $G=SO_2$ über $\IR$ ist entweder die triviale Darstellung oder zu genau einer der Darstellungen
\[\mathcal{D}_k : \begin{psmallmatrix}\cos(\alpha)&-\sin(\alpha)\\\sin(\alpha)&\cos(\alpha)\end{psmallmatrix} \mapsto \begin{psmallmatrix}\cos(k\alpha)&-\sin(k\alpha)\\\sin(k\alpha)&\cos(k\alpha)\end{psmallmatrix}\]
für eine natürliche Zahl $k>0$ isomorph und diese Darstellungen sind paarweise nicht-isomorph.
\end{corollary}

\begin{remark}
Wenn man nun die Spiegelungen hinzunimmt, ist die Gruppe nicht länger kommutativ. Eine wichtige Beobachtung ist, dass in $O_2$ die Drehung um $\alpha$ und die Drehung um $-\alpha$ zueinander konjugiert sind. Wenn wir Darstellungen also in Termen von Charakteren klassifizieren wollten, dann halten wir fest, dass Charaktere von $O_2$ nicht zwischen diesen beiden unterscheidet.

Andererseits können wir natürlich jede $O_2$-Darstellung auf $SO_2$ einschränken und in irreduzible Bestandteile bzgl. $SO_2$ zerlegen, deren Charaktere wir jetzt kennen. Der Wert $\chi_T(\rho_\alpha)$ muss also eine Summe von Exponentialfunktionen sein. Welche solche Summen unterscheiden nicht zwischen $\alpha$ und $-\alpha$ ? $2\cos(\alpha) = e^{i\alpha k} + e^{-i\alpha k}$ und Kombinationen davon.
\end{remark}

\begin{theorem}[Darstellungstheorie von $O_2$]
Jede stetige, endlich-dimensionale, irreduzible Darstellung von $O_2$ über $\IK\in\set{\IR,\IC}$ ist isomorph zu genau einer Darstellung der folgenden Liste:
\begin{enumerate}
\item Die triviale Darstellung $O_2\to\IC, g\mapsto 1$
\item Die Determinante $\det: O_2\to\IC$, d.h. $g\mapsto\begin{cases}+1&g\in SO_2 \\ -1&g\notin SO_2\end{cases}$.
\item Eine der Darstellungen
\[\mathcal{D}_k : \begin{pmatrix}\cos(\alpha)&\mp\sin(\alpha)\\\sin(\alpha)&\pm\cos(\alpha)\end{pmatrix} \mapsto \begin{pmatrix}\cos(k\alpha)&\mp\sin(k\alpha)\\\sin(k\alpha)&\pm\cos(k\alpha)\end{pmatrix}\]
für eine natürliche Zahl $k>0$ mit dazugehörigem Charakter $\chi_k(g) = \begin{cases} 2\cos(k\alpha) & g=\rho_\alpha \\ 0 & g\notin SO_2\end{cases}$.
\end{enumerate}
$\mathcal{D}_k$ ist isomorph zur Darstellung $STF^k$.
\end{theorem}
\begin{proof}
Man kann von Hand nachprüfen, dass diese Definitionen tatsächlich Darstellungen sind. Die beiden eindimensionalen sind automatisch auch irreduzibel.

Wir zeigen die Irreduzibilität von $\mathcal{D}_k$, indem wir die Bilinearform auf Charakteren anwenden und $\braket{\chi_k,\chi}$ für einen beliebigen Charakter $\chi$ von $O_2$ ausrechnen:
\begin{align*}
\braket{\chi_k,\chi}_{O_2} &= \frac{1}{2}\int_{SO_2} \chi_k(g)\chi_k(g^{-1}) \dd g + \frac{1}{2}\int_{SO_2} \underbrace{\chi_k(hg)}_{=0}\chi((hg)^{-1})\dd g &\text{für}\,h\in O_2\setminus SO_2\\
&= \frac{1}{2}\int_{SO_2} \chi_k(g)\chi(g^{-1}) \dd \alpha \\
&= \frac{1}{2} \braket{ (\chi_k)_{|SO_2} , \chi_{|SO_2}}_{SO_2} \\
\intertext{Speziell für $\chi=\chi_k$ selbst ist das}
&= \frac{1}{2} \braket{e^{ik\alpha} + e^{-ik\alpha}, e^{ik\alpha} + e^{-ik\alpha}}_{SO_2} \\
&=\frac{1}{2}(1^2 + 1^2)=1 &\text{Orthogonalitätsrelation}
\end{align*}
Also sind die $\chi_k$ tatsächlich irreduzible Charaktere von $O_2$ nach \ref{darstellungen:schur_umkehrung}.

\medbreak
Unsere Behauptung ist u.A., dass es keine weiteren, uns noch unbekannten irreduziblen Charaktere von $O_2$ gibt. Wir nehmen uns also einen irreduziblen Charakter $\chi$, der nicht einer der $\chi_k$ ist und auch nicht trivial ist, und zeigen, dass er gleich der Determinante sein muss.

\smallbreak
Wir haben schon festgestellt, dass in $O_2$ die Charakterwerte eingeschränkt auf $SO_2$ immer die Form $\chi(\rho_\alpha) = \sum_{k\in\IZ} a_k\cdot  e^{ik\alpha}$ mit $a_k=a_{-k}\in\IN$ haben müssen. Wenn wir nun ausnutzen, dass $\braket{\chi_k,\chi}_{O_2}=0$ sein müsste aufgrund der Orthogonalitätsrelation, dann folgern wir $a_k=0$ für $k\neq 0$, d.h. $\chi(\rho_\alpha)=a_0$ ist konstant auf $SO_2$. Das Komplement von $SO_2$ besteht ausschließlich aus Spiegelungen und die sind alle zueinander konjugiert, sodass $\chi$ auch auf dem Komplement $O_2\setminus SO_2$ konstant sein muss. Die Orthogonalitätsrelation $\braket{1,\chi}_{O_2}=0$ impliziert dann, dass die Konstante auf dem Komplement genau $-a_0$ sein muss, d.h. $\chi = a_0\cdot \det$. Wegen der Irreduzibilität muss $a_0=1$ sein.

\medbreak
Damit ist also insbesondere zeigt, dass die Darstellungen $STF^k$, die wir schon vor der Charaktertheorie als irreduzibel erkannt hatten, irgendwo in der Liste vorkommen müssen. Wir nutzen aus, dass $STF^k$ zum Raum der homogenen, harmonischen Polynome in zwei Variablen vom Grad $k$ isomorph ist.

Für Polynome in zwei Variablen kann man zufällig explizit die harmonischen hinschreiben: Es sind genau die Real- und Imaginärteile der Polynome in einer komplexen Variable. In Grad $k$ haben wir also $\Re((x+iy)^k)$ und $\Im((x+iy)^k)$ als Basis des Raums der harmonischen Polynome vom Grad $k$. Auf $\IR^2$, also der komplexen Ebene operiert $SO_2$ durch einfache Drehung, also Multiplikation mit $e^{i\alpha}$. Im Polynomraum wird also insbesondere $z^k$ auf $e^{ik\alpha}z^k$ abgebildet, d.h. auf den Grad-$k$-Polynomen operiert $SO_2$ durch $k$-fache Drehung, also muss es die irreduzible Darstellung $\mathcal{D}_k$ sein.
\end{proof}

\pagebreak
\section{Darstellungstheorie \texorpdfstring{von $O_3$}{der Isometriegruppe des dreidimensionalen Raums}}
% !TeX root = spherical_harmonics.tex
% !TeX spellcheck = de_DE

\begin{remark}
Jedes Element von $SO_3$ ist bekanntlich eine Drehung $\rho_{\alpha,v}$ um eine Drehachse $v\in S^2$ und einen Drehwinkel $\alpha$. Wir erinnern noch einmal, dass $\rho_{\alpha,v} =\rho_{-\alpha,-v}$ ist.

\smallbreak
Besonders wichtig für uns ist, dass alle Drehungen um den gleichen Drehwinkel zueinander konjugiert sind. Insbesondere sind also alle Charaktere aller Darstellungen von $SO_3$ nur von $\alpha$, nicht von $v$ abhängige, (gerade) Funktionen.
\end{remark}

\begin{theorem}[Darstellungstheorie von $SO_3$]
Jede stetige, endlich-dimensionale, irreduzible Darstellung von $G=SO_3$ über $\IK\in\set{\IR,\IC}$ ist zu $STF^k$ für ein $k\in\IN$ isomorph. Der Charakter von $STF^k$ ist genau $\chi_k(\rho_{\alpha,v}) = \sum_{j=-k}^k e^{ij\alpha} = 1+\sum_{j=1}^k 2\cos(j\alpha)$. 
\end{theorem}
\begin{proof}
Wir zeigen zunächst, dass der Charakter der Darstellung $STF^k$ genau die angegebene Form hat. Wie wir eben festgestellt haben, ist es ausreichend, eine feste Drehachse zu betrachten und für jeden Drehwinkel um diese Achse den Charakterwert auszurechnen.

Wir benutzen außerdem, dass $STF^k$ zum Raum der harmonischen, homogenen Polynome vom Grad $k$ isomorph ist. Dieser Isomorphismus ist basis-abhängig und wir wählen die Basis so, dass die Unbekannte $z$ unsere feste Drehachse ist. Wir schreiben ein beliebiges homogenes Polynom dann als $p = \sum_{a=0}^k p_a(x,y)z^a$, wobei $p_a$ ein homogenes Polynom vom Grad $k-a$ ist. Dass $p$ harmonisch ist, ist äquivalent zu
\begin{align*}
0 &= \Delta p \\
\iff \partial_z^2 p &= -(\partial_x^2+\partial_y^2)p \\
\iff \sum_{a\geq 2} p_a(x,y) a(a-1)z^{a-2} &= -\sum_a (\partial_x^2+\partial_y^2)p_a z^a \\
\iff \forall a: p_{a+2}(x,y) &= \frac{-1}{(a+1)(a+2)}(\partial_x^2+\partial_y^2)p_a(x,y)
\end{align*}
Mit anderen Worten können wir $p$ eindeutig festlegen, indem wir $p_0$ und $p_1$ beliebig festlegen und diese Rekursion anwenden. Die Rekursion liefert uns einen expliziten Isomorphismus der Vektorräume $P^{k}\oplus P^{k-1}$ mit $P_h^k$. Außerdem ist die Rekursion offenbar $SO_2$-linear, weil der 2D-Laplace-Operator $SO_2$-linear ist. Das zeigt uns, dass das sogar ein Isomorphismus von Darstellungen ist. Wir erhalten also die Aussage, dass die Restriktion von $STF^k$ auf $SO_2$ isomorph ist zu $Sym^k(\IR^2)\oplus Sym^{k-1}(\IR^2)$ und wie sich die symmetrischen Tensoren in irreduzible zerlegen, wissen wir bereits.

Insbesondere können wir also die Charaktere explizit ausrechnen: Es ist
\begin{align*}
    \chi_{STF^k}(\rho_{\alpha,v}) &= \chi_{Sym^k(\IR^2)}(\rho_\alpha) + \chi_{Sym^{k-1}(\IR^2)}(\rho_\alpha) \\
    &= (2\cos(k\alpha)+2\cos((k-2)\alpha)+\ldots)+(2\cos((k-1)\alpha)+2\cos((k-3)\alpha) +\ldots) \\
    &= 1+\sum_{j=1}^k \cos(j\alpha)
\end{align*}
wie behauptet.

\medbreak
Wieso sind das nun alle irreduziblen Charaktere? Wir haben bereits festgestellt, dass jeder Charakter $\chi: SO_3\to\IK$ nur vom Drehwinkel $\alpha$ abhängig ist und eine gerade Funktion von $\alpha$ ist. Als Darstellung von $SO_2$ betrachtet, muss es also die Form $a_0 + \sum_{k>0} a_k\cdot 2\cos(k\alpha)$ mit $a_k\in\IN$ haben. Die Funktionen $2\cos(k\alpha)$ bilden eine Basis des Vektorraum solcher Funktionen. Die Funktionen $\sum_{j=0}^k \cos(j\alpha)$ also auch. Es kann also keinen weiteren irreduziblen Charakter geben, denn der müsste im selben Vektorraum liegen, aber orthogonal zu allen diesen Basiselementen sein.
\end{proof}

\begin{remark}
Man beachte, dass beide Basen im letzten Abschnitt zwar Orthonormalbasen sind, aber bzgl. verschiedener Skalarprodukte (Haar-Integral von $SO_2$ vs. $SO_3$).
\end{remark}


\pagebreak
\section{Warum Spherical Harmonics?}
% !TeX root = spherical_harmonics.tex
% !TeX spellcheck = de_DE
Auch wenn wir eingangs in unserem Kurs so sehr über Basen geschimpft haben\footnote{Wir wollen keine Namen nennen, aber Johannes war besonders hartnäckig...}, so ist uns natürlich bewusst, dass gerade für effiziente Berechnungen eine Basis praktisch immer gewählt wird. Welche Basis man wählt, ist dabei von großer Bedeutung: Andrea hat in ihrer Doktor-Arbeit eine bestimmte Berechnung, welche z.B. zur Modellierung von Luftbewegung millionenfach wiederholt genutzt werden soll, um den Faktor 1000 verschnellern können, und das nur durch geschickte Basiswahl. 

\begin{remark}
	Meistens ist eine Basis geschickt gewählt, wenn sie orthogonal ist. Nun gibt es aber viele Skalarprodukte, die man wählen kann, die durchaus auch unterschiedliches Verständnis von Orthogonalität festlegen.
\end{remark}
Schauen wir uns einmal einige Beispiele für mögliche Skalarprodukte auf dem Raum der Polynome an:
\begin{example}
	Sei $B$ die Kugel in $\IR^3$ mit Radius 1 und $S^2$ die Oberfläche dieser Kugel. Dann sind mit
	\begin{align*}
		\braket{f,g}_B :&= \int_B f g \dd B & \braket{f,g}_{S^2} :&= \int_{S^2} f g \dd {S^2}
	\end{align*}
	zwei Skalarprodukte auf dem Raum der Polynome gegeben.
\end{example}

\begin{example}
	Sei $Z$ ein Zyliner in $\IR^3$, z.B. mit Radius und Höhe 1. Dann ist mit 
	\begin{align*}
		\braket{f,g}_Z :&= \int_Z f g \dd Z
	\end{align*}
	ein Skalarprodukt auf dem Raum der Polynome gegeben.
\end{example}

\begin{example}
	Sei $K$ ein Kegelstumpf in $\IR^3$, z.B. mit oberen Radius und Höhe 1, unterem Radius $\frac{1}{2}$. Dann ist mit 
	\begin{align*}
		\braket{f,g}_K :&= \int_K f g \dd K
	\end{align*}
	ein Skalarprodukt auf dem Raum der Polynome gegeben.
\end{example}

\begin{example}
	Sei $Q$ ein Quader in $\IR^3$, z.B. mit den Kantenlängen 1, 2, und 3. Dann ist mit 
	\begin{align*}
		\braket{f,g}_Q :&= \int_Q f g \dd Q
	\end{align*}
	ein Skalarprodukt auf dem Raum der Polynome gegeben.
\end{example}

\begin{centralquestion}
Wäre es nicht praktisch, eine Basis zu finden, die für möglichst viele Skalarprodukte orthogonal ist? Wie würdet ihr mit dem neu erworbenen Wissen nach solch einer Basis suchen? Diskutiert über eure Ansätze und versucht einen Algorithmus zu entwickeln, der eine möglichst allgemein einsetzbare Basis konstruiert.
\end{centralquestion}

\pagebreak
\begin{remark}
Am besten wählt man eine Basis, die möglichst gut an das gegebene Problem angepasst ist, doch was genau heißt das? Für die Auswahl an Problemen in diesem Kurs haben wir festgelegt, dass es sich um Probleme in unserer physikalischen Welt handeln soll. Wir haben uns bereits angeschaut, wie sich diese Einschränkung mathematisch äußert: Alles was wir messen wollen, muss natürlich, also $O_3$-verträglich sein. Wir haben uns auch schon mit der Darstellungstheorie von $O_3$ und einigen ihrer Untergruppen beschäftigt und festgestellt, dass $O_3$ den Raum der Polynome (oder äquivalent: den Raum der symmetrischen Tensoren) in irreduzible Unterräume zerlegt. 
\end{remark}
\begin{remark}
Die irreduziblen Unterräume sind besonders: Wir haben bereits sehr gut verstanden, dass sie immer (also für jedes $O_3$-verträgliche Skalarprodukt) senkrecht zueinander stehen und dass lineare Abbildungen zwischen zwei irreduziblen Unterräumen entweder 0 oder ein Vielfaches der Identität sein können. Wenn man also eine lineare Wirkung berechnen möchte, erspart man sich durch die Zerlegung in irreduzible Unterräume einiges an Arbeit: Alles, was zu bestimmen ist, sind die Konstanten, mit der die Identität zwischen zwei zueinander isomorphen irreduziblen Unterräumen koppeln.
\end{remark}
\begin{remark}
Im übrigen: Für bilineare Abbildungen ist es etwas komplizierter, aber die grundsätzliche Aussage bleibt bestehen: Alle Abbildungen sind entweder 0 oder Vielfache einer Identität (auch dann wenn die Identität etwas komplizierter zu berechnen ist).
\end{remark}
\begin{remark}
 All dies legt nahe, dass eine Basiswahl, die an die irreduziblen Unterräume angepasst ist, praktisch immer eine gute Idee ist. Jetzt sind diese irreduziblen Unterräume aber allermeistens nicht ein-dimensional, wie sucht man sich nun eine möglichst kanonische Basis aus? 
\end{remark}
\begin{remark}
	\label{rem:einbettung_in_physik}
 Hinzu kommt, dass die irreduziblen Unterräume zwar für ganz $O_3$ irreduzibel sind, aber es gibt einige Probleme in der Physik, die eher einer Einbettung von $O_2$ in $O_3$ gleichen, z.B. (Teilchen-) Kollision an einer Wand, eine ausgezeichnete Richtung haben, z.B. die Bewegung von geladenen Teilchen durch einen Kondensator, oder rotierende Systeme beschreiben (Einbettung von $SO_2$ in $O_3$.
\end{remark}

 \begin{remark}
 	Idealerweise ist unsere Basis auch für solche Fälle ausgelegt. Es stellt sich heraus, dass beide Probleme gleichzeitig gelöst werden können.
 \end{remark}
 \begin{definition}[Konstruktion der $\Gae\jae\el\soft\fae\aaa\en\dae$-$\Zae\jae\tae\el\iii\en$-$\Bae\aaa\sae\iue$ (Gelfand\footnote{Israel Gelfand, $\I\sae\rae\aaa\iii\el\soft$ $\Em\ooo\iii\ssae\jae\jae\wae\iii\tschae$ $\Gae\jae\el\soft\fae\aaa\en\dae$ (1913-2009), sovietischer Mathematiker}-Zetlin\footnote{Michael Zetlin, $\Em\iii\xa\aaa\iii\el$ $\El\soft\wae\ooo\wae\iii\tschae$ $\Zae\jae\tae\el\iii\en$ (1924-1966), russ. Mathematiker und Physiker}-Basis]
 	\label{def:konstruktion_gz_basis}
	Sei $G$ eine Gruppe mit halb-einfacher Darstellung auf einem Vektorraum $V$ und $U^i_0$ einer der irreduziblen Unterräume von $V$. Für diesen Unterraum wird dessen \emph{Gelfand-Zetlin-Basis} wie folgt konstruiert:
	\begin{enumerate}[label={\arabic*.)}]
		\item Setze das Level $l=0, H_0 = G$
		\item Bestimme für jeden Unterraum $U_l^i$: \label{def:gz_it_anfang}
		\begin{enumerate}
			\item Berechne $\dim{U_l^i}$
			\item Ist $U_l^i$ 1-dimensional? Dann wähle einen Vektor aus $U_l^i$ als einen der Basisvektoren von $U^1_0$.
			\item Ist $U_l^i$ $>1$-dimensional? Dann merke dir $U_l^i$ für den nächsten Schritt.
		\end{enumerate}
		\item Wähle geschickt eine möglichst große Untergruppe $H_{l+1}$ von $H_{l}$, sodass jeder Unterraum $U_l^i$ mit $\dim{U_l^i}>1$ in für $H_{l+1}$ irreduzible Unterräume  zerfällt. Dabei müssen alle Unterräume, die von einem gemeinsamen $U_l^i$ stammen, paarweise nicht isomorph sein (Klappt dies nicht, ist entweder eine andere Untergruppe zu finden, oder es gibt keine Gelfand-Zetlin-Basis). \label{def:gz_it_ugschritt}
		\item Nummeriere die Unterräume des nächst-höheren Levels: Nenne alle für $H_{l+1}$ irreduziblen Unterräume $U_{l+1}^{1}, \cdots, U_{l+1}^{k}$, inklusive der bereits für $H_l$ 1-dimensionalen Unterräume.
		\item Erhöhe das Level um 1: $l++$. \label{def:gz_it_ende}
		\item Wiederhole die Schritte \ref{def:gz_it_anfang} bis \ref{def:gz_it_ende}, bis alle Unterräume vom nächst-höheren Level 1-dimensional sind.
		\item $l_{\text{max}} = l$
		\end{enumerate}
\end{definition}
\begin{remark}
	Formal betrachtet, brauchen die 1-dimensionalen Untergruppen nicht für Schritt \ref{def:gz_it_ugschritt} ausgeschlossen werden. Sie werden für jede Untergruppe 1-dimensional bleiben und sich nicht weiter ändern, entsprechend braucht man sie für die weiteren Schritte auch nicht zu berücksichtigen.
\end{remark}
\begin{definition}[$\Gae\jae\el\soft\fae\aaa\en\dae$-$\Zae\jae\tae\el\iii\en$-$\Dae\jae\rae\jae\wae\ooo$ (Gelfand-Zetlin-Baum)]
	Mit der Konstruktion nach \ref{def:konstruktion_gz_basis} bauen wir einen Baum mit $l_{\text{max}}$ Leveln. Der Stammknoten ist $U_0^i$, die Blätter sind alle gefundenen 1-dimensionale Unterräume. Die Knoten zwischen dem Stammknoten und den Blättern entsprechen den irreduziblen Unterräumen der jeweiligen Level, die Kanten verbinden einen irreduziblen Unterraum $U$ eines Levels mit allen irreduziblen Unterräumen des nächst-höheren Levels, die aus $U$ im Schritt \ref{def:gz_it_ugschritt} herauskommen. Eindimensionale Unterräume, die bereits in kleineren Leveln auftreten, werden hier als Knoten auf jedem höheren Level wiederholt bis zum maximalen Level.
	Diesen Baum nennen wir \emph{Gelfand-Zetlin-Baum}. Er kann z.B. wie folgt aussehen:
	\begin{align*}
		\begin{tikzpicture}[baseline={([yshift=-2ex]current bounding box.center)}]
			\foreach\x  in {1,2,...,11}{
				\node (u3\x) at (\x,-4.5){$U_3^{\x}$};
			}
%			\foreach\x  in {1,2,3.5,6,8,9.5,11}{
%				\node (u2\x) at (\x,-3){$U_2^{\x}$};
%			}
			\foreach\x [evaluate=\x as \y using {\x*10/7}] in {1,2,...,7}{
				\node (u2\x) at (\y,-3){$U_2^{\x}$};
			}
%			\foreach\x  in {1,2.75,6,9.5}{
%				\node (u1\x) at (\x,-1.5){$U_1^{\x}$};
%			}
			\foreach\x [evaluate=\x as \y using {\x*10/5}] in {1,2,...,4}{
				\node (u1\x) at (\y,-1.5){$U_1^{\x}$};
			}
			\node (u0) at (4.7875,0){$U_0^i$};
			\foreach\x in {1,2,...,4}{
				\draw[] (u0) to (u1\x);
			}
			\draw[] (u11) to (u21);
			\draw[] (u12) to (u22);
			\draw[] (u12) to (u23);
			\draw[] (u13) to (u24);
			\draw[] (u14) to (u25);
			\draw[] (u14) to (u26);
			\draw[] (u14) to (u27);
			\draw[] (u21) to (u31);
			\draw[] (u22) to (u32);
			\draw[] (u23) to (u33);
			\draw[] (u23) to (u34);
			\draw[] (u24) to (u35);
			\draw[] (u24) to (u36);
			\draw[] (u24) to (u37);
			\draw[] (u25) to (u38);
			\draw[] (u26) to (u39);
			\draw[] (u26) to (u310);
			\draw[] (u27) to (u311);
			\foreach\x [evaluate=\x as \y using {-\x*1.5}] in {0,1,...,3}{
				\node (l\x) at (-2,\y){$l=\x$};
				\node (h\x) at (-1,\y){$H_{\x}$};
			}
		\end{tikzpicture}
	\end{align*}
	In einem Gelfand-Zetlin-Baum sind alle Darstellungen paarweise ungleich. Es kann aber auftreten, dass manche Darstellungen isomorph zueinander sind (wobei die Kinder eines Knotens niemals isomorph zueinander sein dürfen).
\end{definition}
\begin{remark}
		Die Wahl der Untergruppen-Kette $H_1\cdots H_{l_{\text{max}}}$ ist hier spielentscheidend und keinesfalls eindeutig. Aus diesem Grund gehört zu einer vollständigen Anweisung \enquote{Wähle eine Gelfand-Zetlin-Basis für $G$ auf $V$} auch die Angabe der Untergruppenkette $H_1\cdots H_{l_{\text{max}}}$.
\end{remark}
\begin{remark}
	Auch wenn die Konstruktion einer Gelfand-Zetlin-Basis klappt, hat man nicht unbedingt etwas gewonnen. Wenn man dazu besonders exotische Untergruppen benutzen muss, bringt einem die Gelfand-Zetlin-Basis in der Anwendung nicht besonders viel. Eigentlich möchte man sich erst für eine Untergruppenkette $H_0\cdots H_{l_{\text{max}}}$ entscheiden und dann schauen, ob die Konstruktion funktioniert.
\end{remark}
\begin{definition}[$\Gae\jae\el\soft\fae\aaa\en\dae$-$\Zae\jae\tae\el\iii\en$-$\Bae\aaa\sae\aaa$]
	\label{def:gz_basis}
	Sei $G$ eine Gruppe mit Darstellung auf einem Vektorraum $V$ und $G$ zerlege $V$ in seine irreduziblen Unterräume $U^1_0,\cdots U^k_0$. Wähle außerdem eine Untergruppenkette $H_1 \gneq\cdots\gneq H_{l_{\text{max}}}$. Für jeden dieser Unterräume wird nun einzeln eine Basis $\mathcal{B}$ konstruiert nach \ref{def:konstruktion_gz_basis}. $\mathcal{B}$ heißt dann \emph{Gelfand-Zetlin-Basis von der Darstellung V bezüglich} $H_1,\cdots H_{l_{\text{max}}}$. 
\end{definition}

\begin{remark}
	Wie in \ref{rem:einbettung_in_physik} bereits angemerkt, sind wir insbesondere an der Darstellung von $O_3$ auf dem Polynomraum und der Einbettung von $O_2$, $SO_2$ und $O_1$ interessiert. Praktischerweise gilt $O_3\gneq O_2\gneq SO_2 \gneq O_1$ und somit bilden die Gruppen $O_3, O_2,SO_2,O_1$ eine Untergruppenkette.
\end{remark}

\begin{maintheorem}[Spherical Harmonics sind eine Gelfand-Zetlin-Basis]
	Die Darstellung von $O_3$ auf dem Polynomraum hat eine Gelfand-Zetlin-Basis bezüglich Untergruppenkette $O_2,SO_2,O_1$. Die Elemente dieser Gelfand-Zetlin-Basis heißen \emph{Spherical Harmonics} oder auch \emph{Kugelflächenfunktionen}.
\end{maintheorem}
% !TeX root = spherical_harmonics.tex
% !TeX spellcheck = de_DE
\begin{sheet}
	\begin{problem}[title={Basiswechsel zwischen verschiedenen Gelfand-Zetlin-Basen}]
			Man kann zwischen den für die vier Fälle aus \ref{cq:fortsetzung} gefundenen Basen wechseln. Gib hierfür explizite Formeln an. Was sagt dies über die relative Lage der Basisvektoren zueinander aus?
	\end{problem}

	\begin{problem}[title={Gelfand-Zetlin-Wald}]
		Wo ein Baum ist, gibt es meist mehrere und wenn hinreichend viele aufeinander hocken, hat man einen Wald. Da ein Gelfand-Zetlin-Baum mit einer irreduziblen Darstellung anfängt, liegt die Idee nahe, einen Gelfand-Zetlin-Wald für halbeinfache Darstellungen zu definieren.
		
		Hierbei gilt zu klären: Wie geht man mit mehrfach auftretenden isomorphen oder gleichen irreduziblen Unterdarstellungen um? Ist es sinnvoll, diese alle einzeln als Knoten aufzuzählen?
		
		Finde eine sinnvolle Definition des Gelfand-Zetlin-Waldes.
		
		Hinweise:
		\begin{itemize}
		 \item Man kann einen \emph{Verzweigungsgraph} definieren. Dies ist ein Gelfand-Zetlin-Baum, der isomorphe Unterräume als einen Knoten miteinander identifiziert. Er enthält genauso viel Information, wie ein Gelfand-Zetlin-Baum.
		\item Kann man kann es sich durch geschickte Einschränkung auf bestimmte halbeinfache Darstellungen relativ einfach machen?
		\item Nach welchen Kriterien kann man die irreduziblen Unterdarstellungen voneinander trennen?
		\end{itemize}
	\end{problem}


\end{sheet}
\end{document}

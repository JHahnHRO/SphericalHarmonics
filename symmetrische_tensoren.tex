
\subsection{Symmetrische Tensoren: Die Verbindung zwischen Polynomen und Tensoren}

\begin{definition}\label{symmetrische_tensoren:def}
Ein Tensor $m$-ten Grades $t\in V^{\otimes m}$ heißt \emph{symmetrisch}, wenn er ein Fixpunkt der Operation der Permutationsgruppe $Sym(m)$ ist, d.h. wenn
\[\forall \sigma\in Sym(m): \sigma\cdot t = t\]
gilt. Der Raum der symmetrischen Tensoren wird \emph{symmetrische (Tensor-)Potenz} genannt und $Sym^m(V)$ geschrieben.
\end{definition}

\begin{example}
\begin{itemize}
\item $v\otimes v\otimes \cdots\otimes v$ ist symmetrisch und umgekehrt: Ein reiner Tensor $v_1\otimes v_2\otimes\cdots\otimes v_m$ ist genau dann symmetrisch, wenn alle $v_1,\ldots, v_m$ Vielfache eines Vektors $v$ sind.
\item $v\otimes w + w\otimes v$ ist symmetrisch.
\end{itemize}
\end{example}

\begin{lemma}
Die Projektion auf den Unterraum der symmetrischen 
$q(t) := \frac{1}{m!} \sum_{\sigma\in Sym(m)} \sigma\cdot t$
\end{lemma}


% !TeX root = spherical_harmonics.tex
% !TeX spellcheck = de_DE
\subsection{Symmetrische Tensoren: Die Verbindung zwischen Polynomen und Tensoren}

\begin{definition}\label{symmetrische_tensoren:def}
Ein Tensor $m$-ten Grades $t\in V^{\otimes m}$ heißt \emph{symmetrisch}, wenn er ein Fixpunkt der Operation der Permutationsgruppe $S_m$ ist, d.h. wenn
\[\forall \sigma\in S_m: \sigma\cdot t = t\]
gilt. Der Raum der symmetrischen Tensoren wird \emph{symmetrische (Tensor-)Potenz} genannt und $\Sym^m(V)$ geschrieben.
\end{definition}

\begin{example}
\begin{itemize}
\item $v\otimes v\otimes \cdots\otimes v$ ist symmetrisch und umgekehrt: Ein reiner Tensor $v_1\otimes v_2\otimes\cdots\otimes v_m$ ist genau dann symmetrisch, wenn alle $v_1,\ldots, v_m$ Vielfache eines Vektors $v$ sind.
\item $v\otimes w + w\otimes v$ ist symmetrisch. Man beachte, dass $v\otimes w + w\otimes v = (v+w)\otimes(v+w) - v\otimes v - w\otimes w$ ist. Man kann allgemein zeigen, dass $Sym^m(V) = \operatorname{span}\Set{v\otimes\cdots\otimes v | v\in V}$ ist.
\end{itemize}
\end{example}

\begin{lemma}[Mittelwerttrick]\label{symmetrische_tensoren:projektion}
Die Projektion auf den Unterraum der symmetrischen Tensoren gegeben durch
\[q(t) := \frac{1}{\abs{S_m}} \sum_{\sigma\in S_m} \sigma\cdot t\]
und $O_n$-linear. Zur Erinnerung: $\abs{S_m} = m!$.
\end{lemma}
Die obige Idee, den Mittelwert über eine Gruppe zu bilden und daraus eine Projektion zu basteln, werden wir häufiger verwenden. Definieren wir uns also allgemeiner:
\begin{definition}[Mittelwerttrick für endliche Gruppen]
	Der Mittelwert der Wirkung einer endlichen Gruppe $G$ auf einen Vektor $v$ des Darstellungsraumes berechnet sich wie folgt:
	\[
		\frac{1}{\abs{G}} \sum_{g\in G} g(v)
	\]
	Dies ist genau die Projektion von $V$ auf den Unterraum der $G$-Fixpunktraum.
\end{definition}

\begin{remark}
Da wir uns vorwiegend mit $O_3$ beschäftigen, reicht diese Definition allerdings nicht aus. Wie uns bestimmt bereits aufgefallen ist, ist $O_3$ nicht endlich, was uns auf den ersten Blick ein paar Schwierigkeiten für die obige Operation einhandelt. Zum Glück zählt $O_3$ zu den kompakten Gruppen, sodass wir die Summe infinitesimalisieren können, ohne über ein unendliches Volumen zu integrieren.
\end{remark}

\begin{definition}[Mittelwerttrick für kompakte Gruppen \& Integration über kompakte Gruppen]
	Sei $G$ eine kompakte Gruppe mit Darstellung auf einem Vektorraum $V$. Dann können wir den Mittelwerttrick auffassen als Integral über die Wirkung von $G$ auf einen Vektor $v\in V$:
	\[
		\frac{1}{\abs{G}} \int_{G} g(v) \dd g
	\]
	Auch dies ist die Projektion von $V$ auf den Unterraum der $G$-Fixpunkte.
	
	\medbreak
	Wir werden nicht präzise machen, wie das Integral tatsächlich definiert ist. Man kann aber zeigen, dass für alle kompakten Gruppen $G$ ein Integral $\int_G$ existiert (das sogenannte Haar\footnote{Alfréd Haar (1885--1933), ungarischer Mathematiker}-Integral), das stetige Funktionen von $G$ in einen endlich-dimensionalen $\IR$- oder $\IC$-Vektorraum $X$ als Input nimmt, einen Wert aus $X$ als Output hat und u.A. folgende Eigenschaften hat: 
	\begin{enumerate}
		\item Das Volumen von $G$, also $\abs{G} = \int_G 1 \dd g$ kann auf einen beliebigen konstanten Faktor $\neq0$ festgelegt werden. Wir setzen der Einfachheit halber immer $\abs{G}=1$.
		\item Das Integral ist rechts- und linksinvariant, es gilt also für alle $\phi: G\to X$ und jedes feste Gruppenelement $h\in G$:
		\[\int_{G} \phi(gh) \dd g = \int_{G} \phi(g) \dd g  = \int_{G} \phi(hg) \dd g\]
		(Im üblichen Sinne: Das eine Integral existiert genau dann, wenn das andere existiert, und dann haben beide den gleichen Wert)
		\item Invertieren ist volumenerhaltend, d.h. es gilt für alle $\phi: G\to X$:
		\[\int_G \phi(g) \dd g = \int_G \phi(g^{-1}) \dd g\]
		\item Konstante (also nicht vom Gruppenelement abhängige) lineare Abbildungen $f: X\to X'$ können am Integral vorbeigezogen werden, es gilt also für alle $\phi: G\to X$:
		\[\int_G f(\phi(g)) \dd g = f\left(\int_G \phi(g) \dd g\right)\]
	\end{enumerate}
	
	Konkrete Beispiele für Haar-Integrale:
	
	\begin{itemize}
	\item Für endliche Gruppen ist das Haar-Integral einfach der diskrete Mittelwert: $\int_G \phi(g) \dd g = \frac{1}{\abs{G}} \sum_{g\in G} \phi(g)$.
	\item Für die zweidimensionale Drehgruppe $G=SO_2$ ist das Haarintegral durch Integration über alle Drehwinkel $\alpha\in[0,2\pi]$ gegeben (ggf. normiert mit $\frac{1}{2\pi}$, damit das Gesamtvolumen 1 wird)
	\item Für die Drehgruppe $SO_3$ ist das Haar-Integral gegeben durch ein Doppelintegral der Form $\int_{S^2} \int_{[0,2\pi]} \ldots \dd \alpha \dd v$: Es wird über die Einheitssphäre $S^2$ integriert, das sind alle Drehachsen, und über $[0,2\pi]$, das sind alle möglichen Drehwinkel. Da eine Drehung um Winkel $\alpha$ um eine Achse $v$ dasselbe ist wie eine Drehung um $-\alpha$ um die Achse $-v$, haben wir dabei jede Drehung doppelt erwischt, aber das kann man ausgleichen, indem man dem Normierungsfaktor eine zusätzliche $\frac{1}{2}$ spendiert.
	\item Die Isometriegruppen $O_2$ und $O_3$ bestehen aus der Untergruppe der Drehungen $SO_n$ und ihrem Komplement, die beide jeweils die Hälfte des Gesamtvolumens haben. Das Komplement hat jeweils die Form $h\cdot SO_n = \set{hg | g\in SO_n}$ für ein beliebiges, fest gewähltes Gruppenelement $h\in O_n \setminus SO_n$. Über das Komplement zu integrieren funktioniert also genauso wie über $SO_n$ selbst zu integrieren. Man kann integriert über $O_n$, indem man die Integrale über beide Hälften summiert.
	\end{itemize}
	
	Äquivalent zur Existenz des Haar-Integrals ist die Existenz einer \enquote{uniformen} Wahrscheinlichkeitsverteilung auf $G$. Man kann also alternativ das Integral auch mit einer Monte-Carlo-Methode definieren als \enquote{Wähle unabhängig und gleichverteilt $N$ zufällige Gruppenelemente, bilde den Mittelwert $\frac{1}{N} \sum_{i=1}^N \phi(g_i)$ und lasse $N\to\infty$ gehen}.
\end{definition}

\begin{theorem}[Isomorphismus von Polynom(iellen Abbildung)en und symmetrischen Tensoren]
\begin{enumerate}
\item Es sei $e_1, e_2, \ldots, e_n$ eine Basis von $V$. Der Raum der Polynome ist isomorph zu $\bigoplus_{m=0}^\infty \Sym^m(V)$. Der Isomorphismus ist die Einschränkung der Abbildung
\[\left\lbrace\begin{array}{rcl}
\bigoplus_{m=0}^\infty V^{\otimes m} &\to& \IK[x_1,\ldots,x_n] \\
e_{i_1} \otimes e_{i_2} \otimes \cdots \otimes e_{i_m} &\mapsto& x_{i_1}\cdot x_{i_2} \cdots x_{i_m}
\end{array}\right.\]
Die umgekehrte Richtung ist durch $x_1^{k_1} \cdots x_n^{k_n} \mapsto q(e_1^{\otimes k_1} \otimes e_2^{\otimes k_2} \otimes \cdots \otimes e_n^{\otimes k_n})$ gegeben.
\end{enumerate}
\smallbreak
Diese Aussage ist nicht basis-frei und kann auch nicht basis-frei gemacht werden, da es in $\IK[x_1, \ldots, x_n]$ ja $n$ ausgezeichnete Elemente vom Grad 1 gibt, eben die Unbekannten. Jeder Isomorphismus würde also auch eine Basis von $V$ auszeichnen.
\smallbreak
Eine moralisch äquivalente, basisfreie Aussage ist aber die folgende:
\begin{enumerate}[resume]
\item Der Raum der polynomiellen Abbildungen $V\to\IK$ ist isomorph zu $\bigoplus_{m=0}^\infty \Sym^m(V)$. Der Isomorphismus ist die Einschränkung der Abbildung
\[\left\lbrace\begin{array}{rcl}
\bigoplus_{m=0}^\infty V^{\otimes m} &\to& \IK[V] \\
v_{i_1} \otimes v_{i_2} \otimes \cdots \otimes v_{i_m} &\mapsto& (v\mapsto \braket{v_{i_1},v} \cdot \braket{v_{i_2},v} \cdots \braket{v_{i_m},v})
\end{array}\right. \]
\item In beiden Fällen gilt: Die Räume sind als Darstellungen der orthogonalen Gruppe isomorph; die angegebenen Isomorphismen sind $O_n$-linear.
\end{enumerate}
\end{theorem}
\begin{proof}
Da das Produkt von Polynomen kommutativ ist, werden $e_{i_1} \otimes e_{i_2} \otimes \cdots \otimes e_{i_m}$ und alle Permutationen $e_{i_{\sigma(1)}} \otimes e_{i_{\sigma(2)}} \otimes \cdots \otimes e_{i_{\sigma(m)}}$ auf denselben Wert abgebildet. Also wird der Mittelwert dieser $m!$ Tensoren ebenfalls auf denselben Wert abgebildet, d.h. $q(e_{i_1} \otimes e_{i_2} \otimes \cdots \otimes e_{i_m})$ geht auch auf $x_{i_1}\cdot x_{i_2} \cdots x_{i_m}$. Das zeigt, dass die beiden in a. definierten Abbildungen zueinander invers sind.

Indem man eine Orthonormalbasis für $e_1, \ldots, e_n$ einsetzt, sieht man, dass in b. tatsächlich derselbe Isomorphismus definiert wird, da $v\mapsto \braket{e_i, v}$ genau die $i$-te Koordinatenabbildung bzgl. dieser Basis ist, d.h. die Abbildung, die den Vektor mit den Koordinaten $(x_1, x_2, \ldots, x_n)$ auf $x_i$ schickt. An der Schreibweise in b. sieht man aber leichter als der in a., dass die Isomorphismen $O_n$-linear sind. 
\end{proof}

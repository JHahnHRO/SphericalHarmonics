% !TeX root = spherical_harmonics.tex
% !TeX spellcheck = de_DE
\subsection{Symmetrische Tensoren: Die Verbindung zwischen Polynomen und Tensoren}

\begin{definition}\label{symmetrische_tensoren:def}
Ein Tensor $m$-ten Grades $t\in V^{\otimes m}$ heißt \emph{symmetrisch}, wenn er ein Fixpunkt der Operation der Permutationsgruppe $S_m$ ist, d.h. wenn
\[\forall \sigma\in S_m: \sigma\cdot t = t\]
gilt. Der Raum der symmetrischen Tensoren wird \emph{symmetrische (Tensor-)Potenz} genannt und $\Sym^m(V)$ geschrieben.
\end{definition}

\begin{example}
\begin{itemize}
\item $v\otimes v\otimes \cdots\otimes v$ ist symmetrisch und umgekehrt: Ein reiner Tensor $v_1\otimes v_2\otimes\cdots\otimes v_m$ ist genau dann symmetrisch, wenn alle $v_1,\ldots, v_m$ Vielfache eines Vektors $v$ sind.
\item $v\otimes w + w\otimes v$ ist symmetrisch. Man beachte, dass $v\otimes w + w\otimes v = (v+w)\otimes(v+w) - v\otimes v - w\otimes w$ ist. Man kann allgemein zeigen, dass $Sym^m(V) = \operatorname{span}\Set{v\otimes\cdots\otimes v | v\in V}$ ist.
\end{itemize}
\end{example}

\begin{lemma}[Mittelwerttrick]\label{symmetrische_tensoren:projektion}
Die Projektion auf den Unterraum der symmetrischen Tensoren gegeben durch
\[q(t) := \frac{1}{m!} \sum_{\sigma\in S_m} \sigma\cdot t\]
und $O_n$-linear. Zur Erinnerung: $|S_m| = m!$.
\end{lemma}

Die obige Idee, den Mittelwert über eine Gruppe zu bilden und daraus eine Projektion zu basteln, werden wir häufiger verwenden. Definieren wir uns also allgemeiner:
\begin{definition}[Mittelwerttrick für endliche Gruppen]
	Der Mittelwert der Wirkung einer endlichen Gruppe $G$ auf einen Vektor $v$ des Darstellungsraumes berechnet sich wie folgt:
	\[
		\frac{1}{|G|} \sum_{g\in G} g(v)
	\]
\end{definition}

Da wir uns vorwiegend mit $O_3$ beschäftigen, reicht diese Definition allerdings nicht aus. Wie uns bestimmt bereits aufgefallen ist, ist $O_3$ nicht endlich, was uns auf den ersten Blick ein paar Schwierigkeiten für die obige Operation einhandelt. Zum Glück zählt $O_3$ zu den kompakten Gruppen, sodass wir die Summe infinitesimalisieren können, ohne über ein unendliches Volumen zu integrieren.
\begin{definition}[Mittelwerttrick für kompakte Gruppen \& Integration über eine kompakte Gruppe]
	Sei $G$ eine kompakte Gruppe mit Darstellung auf einem Vektorraum $V$. Dann können wir den Mittelwerttrick auffassen als Integral über die Wirkung von $G$ auf einen Vektor $v\in V$:
	\[
		\frac{1}{|G|} \int_{G} g(v) \dd g
	\]
	Wir werden nicht präzise machen, wie das Integral tatsächlich definiert ist. Man kann aber zeigen, dass für alle kompakten Gruppen $G$ ein Integral $\int_G$ existiert (das sogenannte Haar\footnote{Alfréd Haar (1885--1933), ungarischer Mathematiker}-Integral), das stetige Funktionen von $G$ in einen endlich-dimensionalen $\IR$- oder $\IC$-Vektorraum $X$ als Input nimmt, einen Wert aus $X$ als Output hat und u.A. folgende Eigenschaften hat: 
	\begin{enumerate}
		\item Das Volumen von $G$, also $\abs{G} = \int_G 1 \dd g$ kann auf einen beliebigen konstanten Faktor $\neq0$ festgelegt werden. Wir setzen immer $\abs{G}=1$.
		\item Das Integral ist rechts- und linksinvariant, es gilt also für jede integrierbare Funktion $\phi: G\to X$ und jedes feste Gruppenelement $h\in G$:
			\[
			\int_{G} \phi(gh) \dd g = \int_{G} \phi(g) \dd g  = \int_{G} \phi(hg) \dd g 
			\]
	\end{enumerate}
	Für endliche Gruppen ist das Haar-Integral einfach der diskrete Mittelwert: $\int_G \phi(g) \dd g = \frac{1}{\abs{G}} \sum_{g\in G} \phi(g)$.
\end{definition}

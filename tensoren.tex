% !TeX root = spherical_harmonics.tex
% !TeX spellcheck = de_DE

\subsection{Wiederholung: Vektorräume}
% !TeX root = spherical_harmonics.tex
Wir kennen uns bereits mit Vektorräumen aus. Typische Beispiele sind $\IR^3$ oder der Raum der Polynome von einem bestimmten Grad $n$. Da uns in diesem Kurs noch weitere Vektorräume begegnen werden und zum Vergleich mit anderen Mathematischen Objekten (z.B. Algebren), ist hier ihre Definition zusammengefasst.


\begin{definition}[Vektorräume und lineare Abbildungen]\label{vektorraeume:def}
Sei $K$ ein Körper. Ein \udot{$K$-Vektorraum} $(V,+,\cdot)$ besteht aus
\begin{itemize}
	\item einer Menge $V$,
	\item einer Abbildung $+: V \times V \to V, (v,w) \mapsto v+w$, genannt \emph{(Vektor)addition} und
	\item einer Abbildung $\cdot: K \times V \to V, (\lambda,v) \mapsto \lambda\cdot v$, genannt \emph{Skalarmultiplikation}.
\end{itemize}
die die Vektorraum-Axiome in Tabelle \ref{vektorraeume:def_table} erfüllen. Alles, was Element eines Vektorraums ist, kann \emph{Vektor} genannt werden.

\begin{table}[!ht]
	\setlength\extrarowheight{10pt} % for a bit of visual "breathing space"
	\begin{tabularx}{\textwidth}{p{7cm} X}
		
		\toprule
		\textbf{Vektorraum-Axiom}                            & \textbf{Bedeutung} \\
		\midrule
        $(V,+)$ ist eine abelsche Gruppe                     & Addition verhält sich wie erwartet \\
		\hspace{1cm}Assoziativität                           & $\forall u,v,w\in V: u+(v+w) = (u+v)+w$  \\
		\hspace{1cm}Kommutativität                           & $\forall u,v\in V: u+v=v+u$ \\
		\hspace{1cm}Neutrales Element bzw. \enquote{Null}    & $\exists 0\in V\forall v\in V: v+0=0+v$  \\
		\hspace{1cm}Inverse bzw. \enquote{negative} Elemente & $\forall v \in V\exists w \in V: v+w=0$ \\
        Eigenschaften der Skalarmultiplikation \\
		\hspace{1cm}Assoziativität                           & $\forall a,b\in K, v\in V: a\cdot(b\cdot v) = (a\cdot b)\cdot v$ \\
		\hspace{1cm}Normierung bzw. Nichttrivialität         & $\forall v \in V: 1 \cdot v = v$, wobei $1$ das Einselement des Körpers bezeichnet \\
        Verträglichkeit von Addition und Skalarmultiplikation \\
		\hspace{1cm}Distributivität bzgl. $(V,+)$            & $\forall a\in K, u,v\in V: a\cdot(u + v) = a\cdot u + a\cdot v$ \\
		\hspace{1cm}Distributivität bzgl. $(K,+)$            & $\forall a,b\in K, v\in V: (a + b)\cdot v = a\cdot v + b\cdot v$ \\
        
        \textbf{Axiome linearer Abbildungen}                  & \textbf{Bedeutung} \\
        \midrule
        Additivität & $\forall v_1,v_2\in V: f(v_1+v_2) = f(v_1)+f(v_2)$ \\
        Homogenität & $\forall \lambda\in K, v\in V: f(\lambda\cdot v) = \lambda \cdot f(v)$ \\
        \bottomrule
	\end{tabularx}
	\caption{Definierende Eigenschaften von Vektorräumen und linearen Abbildungen}
    \label{vektorraeume:def_table}
\end{table}

Sind $V,W$ zwei $K$-Vektorräume und $f: V\to W$ eine Abbildung, so heißt f \emph{(K-)lineare Abbildung} oder \emph{(Vektorraum-)Homomorphismus}, falls die beiden Axiome in Tabelle \ref{vektorraeume:def_table} erfüllt sind.
\end{definition}

\begin{remark}
Wir sind praktisch ausschließlich an $\IR$- und $\IC$-Vektorräumen interessiert in diesem Kurs.
\end{remark}

\begin{definition}[Kern \& Bild]
Ist $f: V\to W$ eine $K$-lineare Abbildung, so ist
\[\ker(f) := \Set{v\in V | f(v)=0}\]
der \emph{Kern von $f$} und
\[\im(f) := \Set{win W | \exists v\in V: f(v)=w}\]
das \emph{Bild von $f$}.
\end{definition}

\subsection{Tensoren}

\begin{remark}
Es gibt i.A. keine Multiplikation zweier Vektoren in irgendeinem Sinne. Wir können immer Skalare mit Vektoren multiplizieren, aber nicht Vektoren mit Vektoren. Nichts desto trotz ist es \emph{manchmal} doch so, dass ein sinnvolles Konzept von Multiplikation existiert für spezielle Vektorräume, z.B.
\begin{enumerate}
	\item Der Vektorraum der Polynome $\IR[X]$ hat die übliche Multiplikation,
	\item Der Vektorraum der Funktionen $X\to\IC$ für einen festen Definitionsbereich $X$ hat die punktweise Multiplikation,
	\item Die Hintereinanderausführung von linearen Abbildungen $\Hom_K(V,W) \times \Hom_K(U,V) \to \Hom_K(U,W), (f,g) \mapsto f\circ g$ ist eine Abbildung, die sich in vielerlei Hinsicht auch wie eine Multiplikation verhält
	\item 
\end{enumerate}

\end{remark}

\begin{definition}[Bilineare \& multilineare Abbildungen]
Sind $V,W,X$ drei $K$-Vektorräume, so heißt eine Abbildung $b: V\times W\to X$ \emph{bilinear}, falls sie die folgenden beiden Distributivgesetze erfüllt:
\begin{description}
	\item $\forall v,v'\in V, w\in W, \lambda,\lambda'\in K: b(\lambda v+ \lambda' v', w) = \lambda b(v,w) + \lambda' b(v',w)$
	\item $\forall v\in V,  w,w'\in W, \lambda,\lambda'\in K: b(v,\lambda w+\lambda'w') = \lambda b(v,w) + \lambda' b(v,w')$
\end{description}
Zusammengefasst: Die Funktion ist separat linear, wenn man nur den ersten oder nur den zweiten Input variiert und den anderen festhält.

Für Abbildungen $V_1\times V_2\times V_3\to X$, die von drei Inputvektoren abhängig sind, kann man entsprechend definieren, dass eine Abbildung \emph{trilinear} heißt, wenn sie die drei Distributivgesetze erfüllt. Für vier, fünf, ... $k$ Input-Vektoren spricht man von dann von \emph{$k$-fach linearen} Abbildungen.
\end{definition}

\begin{example}
Die drei genannten \enquote{Multiplikationen} sind bilinear. Es gibt viele weitere, äußerst nützliche Beispiele.
\begin{enumerate}[resume]
	\item Richtungsableitungen sind bilinear in der Richtung und der Funktion:
	
	 Sei $X\subseteq\IR^n$ ein geeigneter Definitionsbereich (z.B. eine offene Menge) und $x_0\in X$ ein Punkt. Sei außerdem $f:X \to\IC$ eine differenzierbare Funktion. Dann existieren insbesondere alle Richtungsableitungen $(\partial_vf)(x_0) = \lim_{t\to 0} \frac{f(x_0)-f(x_0+tv)}{t}$. Diese Operation ist bilinear in $v$ und $f$.
     
     \item Allgemeiner Differentialoperatoren:
     
     Für mehrfach differenzierbare Funktionen kann man natürlich auch mehrere Ableitungsschritte hintereinander ausführen. Auf diese Weise erhält man multilineare Abbildungen: $k$-faches Ableiten in $k$ Richtungen, also der Differentialoperator $\partial_{v_1} \partial_{v_2} \cdots \partial_{v_k}$ ist $k$-fach linear in den Vektoren $v_1, ..., v_k$ als Inputs. Die Anwendung auf eine Funktion ist entsprechend $(k+1)$-fach linear in den $k$ Richtungsvektoren und der Funktion als Inputs.
\end{enumerate}
\end{example}

The above definition has already hinted at an operation, that a tensor space may lack: A bilinear multiplication of two vectors $\cdot : V \times V \to V$. We can formally define the properties of such a binary operation:
\begin{definition}[Bilinear multiplication in a vector space]
	Let $F$ be a field, and let $V$ be a vector space over $F$. Denote with $\cdot$ an additional binary operation, $\cdot : V \times V \to V$ (i.e. if $\vc{v}$ and $\vc{w}$ are any two elements of $V$, $\vc{v}\cdot \vc{w}$ is the product of $\vc{v}$ and $\vc{w}$ and also lies in $V$). If this operation fullfills all properties in the following table for all elements $\vc{u},\vc{v},\vc{w}\in V$, $a,b,\in F$, then we call it a \emphName{bilinear multiplication} of two vectors. 
%	\begin{table}[!ht]
%		\setlength\extrarowheight{10pt} % for a bit of visual "breathing space"
%		\begin{tabularx}{\textwidth}{X X}
%			\toprule
%			\textbf{Axiom}             & \textbf{Meaning}                                                           \\ 
%			\midrule
%			Right distributivity       & $(\vc{u} +\vc{v})\cdot\vc{w} = \vc{u}\cdot\vc{w} + \vc{v}\cdot\vc{w}$      \\
%			Left distributivity        & $\vc{w}\cdot(\vc{u} +\vc{v}) = \vc{w}\cdot\vc{u}\cdot + \vc{w}\cdot\vc{v}$ \\
%			Compatibility with scalars & $(a\vc{w})\cdot(b\vc{u}) = (ab)(\vc{w}\cdot\vc{u})$   
%			\\              
%			(Possibly) Associativity of multiplication  & $\vc{v}\cdot(\vc{w}\cdot\vc{u}) = (\vc{v} \cdot \vc{w}) \cdot \vc{u}$     
%			\\
%			\bottomrule
%		\end{tabularx}
%		\caption{Properties of bilinear multiplication}
%	\end{table}
\end{definition}
Note, that commutativity is a not required property of a bilinear multiplication (hence the distinction between right and left distributivity). Depending on the situation, the multiplication may or may not have to fullfill the requirement of associativity.

Typical examples for vector spaces, that are equipped with a bilinear multiplication, include: $\IR^3$ with the cross product, $\IR^{n\times n}$ with the matrix product. A vector space that is typically not equipped with a bilinear multiplication is the space of polynomials up to degree $n$, since multiplying two polynomials can result in a polynomial of higher degree than $n$. However, we can extend this vector space to the space of all polynomials $P$. $P$ is easily equipped with a bilinear multiplication (defined as you would expect the multiplication of two polynomials to work), which even happens to be commutative.

The concept of a vector space equipped with a bilinear multiplication has its own name:
\begin{definition}[Algebra]
	Let $F$ be a field, and let $V$ be a vector space over $F$. If $V$ can be equipped with a bilinear multiplication, then $V$ is an \emphName{algebra} over $F$. 
\end{definition}
An algebra over $F$ is sometimes also called an $F$-algebra, and $F$ is called the base field of $V$. The binary multiplication is often simply referred to as multiplication in $V$. 

\subsubsection{The tensor product}
Let us now look at a new kind of multiplication $\otimes$ between two vectors $\vc{u},\vc{v} \in \IR^3$: The tensor product. There are many ways of motivating and defining it, but for our purposes we shall be satisfied by starting with a bilinear, associative, non-commutative multiplication of two vector spaces $\otimes: U \times V \to U \otimes V$ whose result is called a tensor and written as $\vc{U}\otimes\vc{V}$ (spoken "U tensor V").
\[\otimes: U \times V \to U \otimes V,\vc{u}\times\vc{v} \mapsto \vc{u}\otimes\vc{v}\]

$U\otimes V$ is in itself a vector space (the so called tensor space) of dimensionality 
\[\dim{U \otimes V} = \dim{U} \cdot \dim{V}.\]
Choosing basis sets for $U$ and $V$, $B_U = \{\vc{b}_i^u| i=1, \cdots, \dim{U}\}$ and $B_V=\{\vc{b}_i^v| i=1, \cdots, \dim{V}\}$ respectively, allows to immediately choose a basis for $U\otimes V$: 
\[\{\vc{b}_i^u\otimes \vc{b}_j^v| i=1, \cdots, \dim{U, j=1,\cdots,\dim{V}}\}.\]
Similar to denoting the coordinates of a vector $v\in V$ e.g. as $v_i$, we can express the coordinates of a tensor with respect to the above basis as $t_{ij}$.


Given a tensor 
\[t=\vc{u}\otimes\vc{v},\] 
one may multiply it with yet another vector $\vc{w}$, resulting in a product with more terms (i.e. vectors)
\[f=\vc{u}\otimes\vc{v}\otimes\vc{w}\] 
and so on. This is similar to the multiplication of polynomials. Indeed, just like polynomials have degrees, tensors can be of different order. The order of a tensor is determined by the number of vectors that are multiplied with each other, the tensor $t$ is of order 2, the tensor $f$ is of order 3. This nomenclature applies to tensor spaces as well, $V\otimes W$ is a tensor space of second order. We are practically exclusively interested in tensor spaces of k-th order of only one vector space $V$, i.e. 
\[\underbrace{V\otimes V \otimes \cdots \otimes V}_{k \text{ times}},\]
which can be denoted as $V^{\otimes k}$. For our purposes we choose $V=\IR^n$ with mostly $n=3$.

\begin{remark}[Tensor algebra]
	Much like the space of all polynomials is an algebra, the vector space of the sum of all tensor products of a vector space $V$,
	\[\mathcal{T}: \bigoplus_{i=0}^{\infty} V^{\otimes i},\]
	is equipped with the multiplication $\otimes$ and hence is an algebra.
\end{remark}

\begin{definition}[Elementary Tensor]
	An elementary tensor is a tensor that can be expressed as a single product of vectors, e.g. $v_1\otimes v_2 $, (as opposed to a sum of products, e.g. $v_1\otimes v_2 - v_2 \otimes v_1$). In the following definitions we will often use elementary tensors, the linear continuation for sums of elementary tensors is always implied.
\end{definition}

\begin{definition}[Casimir Element or $\delta_{ij}$-Tensor]
	Let $\Omega_V$ denote the symmetric, isotropic 2-tensor of the vector space $V$ of dimensionality $k$. With any complete orthonormal basis set $\{\vc{b}_1,\cdots,\vc{b}_k\}$ for $V$, $\Omega_V$ can be expressed as
	\begin{equation}
		\Omega_V = \sum_{i=1}^k \vc{b}_i \otimes \vc{b}_i
	\end{equation}
	Since we will mostly work with $V=\IR^3$ with some orthonormal basis set $\{\vc{b}_1,\vc{b}_2,\vc{b}_3\}$, the corresponding $\Omega_V$ shall be written as $\Omega$,
	\[\Omega := \Omega_{\IR^3} = \sum_{i=1}^3 \vc{b}_i \otimes \vc{b}_i.\]
\end{definition}
\begin{lemma}
	The Casimir element does not depend on the choice of the orthonormal basis for $V$, i.e. it is indeed isotropic.
\end{lemma}
\begin{proof}
	Take $B:=\left\{e_1, \cdots e_d\right\}$, $\tilde{B}:=\left\{\tilde{e_1}, \cdots \tilde{e_d}\right\}$ to be two orthonormal basis sets and the basis transformation $\phi:V\to V, v \mapsto \sum_{i=1}^{d}\braket{v,\tilde{e_i}}\tilde{e_i}$. Then $s_{ij}=\braket{e_j,\tilde{e_i}}$ are the entries of an orthogonal matrix and we have
	\begin{align*}
		\Omega_V &= \sum_{i=1}^d e_i \otimes e_i\\
		\implies \phi\left(\Omega_V\right) &= \phi\left(\sum_{i=1}^d e_i \otimes e_i\right)\\
		&= \sum_{i=1}^d \phi\left(e_i\right) \otimes \phi\left(e_i\right)\\
		&=\sum_{j=1}^d \left(\sum_{i=1}^d \braket{e_j,\tilde{e_i}}\tilde{e_i}\right) \otimes \left(\sum_{i=1}^d \braket{e_j,\tilde{e_i}}\tilde{e_i}\right) \\
		&=\sum_{i=1}^d\sum_{i'=1}^d\underbrace{\left(\sum_{j=1}^d \braket{e_j,\tilde{e_i}}\cdot\braket{e_j,\tilde{e_i}}\right)}_{=(SS^T)_{ii'}=1_{ii'}=\delta_{ii'}} \tilde{e_i}\otimes \tilde{e_i'} \\
		&=\sum_{i=1}^d\tilde{e_i}\otimes \tilde{e_i} = \Omega_V
	\end{align*}
\end{proof}

\subsection{The notion of meaningfull linear operations with tensors - light introduction to invariant theory}
In the context of our physical world, we can use tensors to measure quantities. A possibly familiar example is stress within a material or gas, which is measured with a tensor of second order. 
The fundamental property we expect from all quantities that we measure, is that they are independent of the orientation of the orthonormal coordinate system - rotating, inverting or mirroring the basis vectors should not change the value of the properties we are interested in (though this may change the numbers with which we express that quantity).
A velocity vector for example should always point in the same direction, regardless of wether its coordinates are $(3,4,0)^T$ or $(0,0,5)^T$. The same goes for its length or the determinant of a matrix.

%The first question we need to answer is: How does a length preserving change of the coordinate system such as a rotation act on a tensor? 
\begin{definition}[Orthogonal Group $O^3$]
	Let $O^3$ be the set of all orthogonal matrices \[O^3:=\{r\in\IR^{3\times 3}: |\det{r}| = 1 \}.\] 
	All of the operations described above can be described with a corresponding matrix in $O^3$. Furthermore, equipped with the typical matrix-matrix multiplication, $O^3$ fullfills all requirements of a group - hence the name \emphName{orthogonal group}. $O^3$ has infinitely many elements, but is still compact - allowing for the notion of taking an average over the whole group.
\end{definition}

\begin{definition}[Action of $O^3$ on the tensor space]
	An element $r\in O^3$ acts on a vector $v\in\IR^3$ via the matrix-vector product $r\cdot v$, resulting in the rotation or mirroring of the given vector. This is extended to tensors - An element $r\in O^3$ acts on a tensor in $\mathcal{T}$ through the matrix-vector multiplication of each factor of that tensor:
	\begin{align*}
		r(\cdot):\mathcal{T} &\to \mathcal{T}\\
		v_1 \otimes \cdots \otimes v_n&\mapsto r(v_1 \otimes \cdots \otimes v_n) :=r\cdot v_1 \otimes \cdots \otimes r\cdot v_n.
	\end{align*}
\end{definition}

\begin{definition}[Orthogonal Group $O^2_{\vc{n}}$ as subgroup of $O^3$]
	Let $\vc{n}\in \IR^3$. $O^2_{\vc{n}}$ is the subgroup of $O^3$, that acts like the identity on $\vc{n}$,  \[O^2_{\vc{n}}:=\{r\in O^3 | r(\vc{n}) = \vc{n} \}.\] 
	$O^2_{\vc{n}}$  is the embedding of $\IR^2$ and $O^2:= \{r\in\IR^{2\times 2}: |\det r| = 1\}$ as a plane $E$ into $\IR^3$ with $E\perp\vc{n}$.
\end{definition}


\begin{definition}[$O^3$-Invariant linear map of the tensor space]
	A linear map $f: \mathcal{T}\to \mathcal{T}$ is called \emphName{compatible (or invariant) with respect to $O^3$}, iff for all actions $r\in O^3$  and for any tensor $\ten{T}\in\mathcal{T}$ the relationship
	\[f \circ r (\ten{T}) = r \circ f (\ten{T})\]
	holds. This precisely expresses the above notion of independency (or invariancy) of a quantity or tensor of the choice of coordinate system.
\end{definition}
Finding $O^3$-invariant linear maps and being sure, that one has found all possibilities, requires some very interesting and complicated piece of mathematics known as \emphName{invariant theory}. This goes far beyond the scope of this thesis, we will restrict ourselves to introducing and applying the results. The interested reader may start by inquiring for the \emphName{Brauer algebra}. 

\subsubsection{Operations on one tensor}\label{chap:tensorOperations}
Below is a list of all types of $O^3$-invariant linear maps between tensors, defined with the elementary tensor $v_1 \otimes \cdots \otimes v_n$. All combinations of them is the full set of of $O^3$-invariant linear maps between tensors:
\begin{description}
	\item[Multiplication with a scalar $s\in\IR$.] This does not change the degree of the tensor.
	
	\item[Permutation of the factors of a tensor.] We have seen this in tensors of second degree, where we called it transpose in correspondence to matrices. Formally, $\sigma \in S_n$ acts on a tensor $v_1 \otimes \cdots \otimes v_n$:
	\[\sigma(\cdot): \mathcal{T}\to \mathcal{T}, v_1 \otimes \cdots \otimes v_n \mapsto v_{\sigma^{-1}(1)} \otimes \cdots \otimes v_{\sigma^{-1}(n)}\]
	This does not change the degree of the tensor or any of the vectors $v_i$, it only reorders them. When using index notation, this operation is represented by a permutation of indices.
	
	\item[Taking the trace $tr$ between two factors of a tensor.] We can pick two vectors of a tensor and replace them with their scalar product:
	\[tr_{i;j} (\cdot): \mathcal{T} \to \mathcal{T}, v_1 \otimes \cdots \otimes v_i \otimes \cdots \otimes v_j \otimes \cdots \otimes v_n \mapsto \braket{v_i,v_j} v_1 \otimes \cdots \otimes v_{i-1}\otimes v_{i+1} \otimes \cdots \otimes v_{j-1} \otimes v_{j+1} \otimes \cdots \otimes v_n\]
	This reduces the degree of the tensor by $2$. When using index notation, this operation is represented by the sum over two identical indices, which is often abbreviated with Einstein's sum convention.
	
	\item[Tensor multiplication with the Casimir element $\Omega$.]
	\[\ten{\Omega} (\cdot):  \mathcal{T}\to \mathcal{T}, v_1 \otimes \cdots \otimes v_n \mapsto \Omega \otimes v_{1} \otimes \cdots \otimes v_{n} \]
	This increases the degree of the tensor by $2$. When using index notation, this operation is represented by the multiplication with $\delta_{ij}$.
\end{description}

\subsubsection{Operations between two tensors}
Given two tensors $u,v \in \mathcal{T}$, the question of finding invariant linear maps with respect to $O^3$ $\mathcal{T} \times \mathcal{T} \to \mathcal{T}$ is easily reduced to the already described case of operations of one tensor. We just have to identify all possible maps $\mathcal{T} \otimes \mathcal{T} \to \mathcal{T} = \mathcal{T} \to \mathcal{T}$ due to the isomorphism $\mathcal{T} \times \mathcal{T} \to \mathcal{T} \cong \mathcal{T} \otimes \mathcal{T} \to \mathcal{T}$ for bilinear and linear maps. To be more specific: Given two tensors $u,v$, take the tensor product $u\otimes v$ and now aply any combination of the operations described for one tensor. Nevertheless, some operations are very common and worth mentioning.

\begin{definition}[Contraction between two tensors]
	Taking one or multiple traces between two tensors is given a special name: Contraction.
	\begin{align*}
		tr_{i;j} (\cdot,\cdot): \mathcal{T} &\times \mathcal{T} \to \mathcal{T},
		\\
		v_1 \otimes \cdots \otimes v_i \otimes \cdots\otimes v_n &\times u_1 \otimes \cdots \otimes u_i \otimes u_j \otimes \cdots \otimes u_n 
		\\
		\mapsto \braket{v_i,u_j} v_1 \otimes \cdots \otimes v_{i-1}\otimes v_{i+1} \otimes \cdots \otimes v_n &\otimes u_1 \otimes u_{j-1} \otimes u_{j+1} \otimes \cdots \otimes u_n
	\end{align*}
\end{definition}
\begin{definition}[Standard scalar product]
	The standard scalar product $\braket{\cdot,\cdot}$ between two tensors of same order is defined as the (ordered) full contraction between the two tensors:
	\begin{align*}
		\braket{\cdot,\cdot} &: \  V^{\otimes n} \times V^{\otimes n} \to \IR,\\
		\braket{u_1 \otimes \cdots \otimes u_n,v_1 \otimes \cdots \otimes v_n} &\mapsto tr_{1,\cdots,n;1,\cdots,n} (u_1 \otimes \cdots \otimes u_n,v_1 \otimes \cdots \otimes v_n) \\
		&= \braket{u_1,v_1}\cdot\braket{u_2,v_2} \cdots \braket{u_n,v_n}.
	\end{align*}
\end{definition}
\subsection{Symmetric tensor space}
\begin{definition}[Symmetric tensors]
	A tensor $v$ of $n$-th order is called symmetric, iff for all permutations $\sigma \in S_n$ holds:
	\[\sigma(v) = v.\]
\end{definition}
The space spanned by all symmetric tensors of degree $n$, $\Sym{n}$, is an invariant subspace of the tensor space due to the completely decoupled nature of the actions of $S_n$ and $O^3$ on a tensor. We can take any tensor and project it onto this subspace by taking the average of all possible permutations:
\begin{align*}
	q: {\IR^3}^{\otimes n} &\to \Sym{n}\\
	q\left(v_1 \otimes \cdots  \otimes v_n\right) &\mapsto \frac{1}{n!} \sum_{\sigma \in S_n} v_{\sigma(1)} \otimes \cdots \otimes v_{\sigma(n)}
\end{align*}
and linear extension. In index notation this is denoted by round brackets surrounding the indices. $\Sym{n}$ is of dimension $\frac{1}{2}(n+1)(n+2)$.

\begin{note}[]
	It is easy to show for the standard scalar product on $\IR^{\otimes n}$, that for all $v,u \in \IR^{\otimes n}$ the following identity holds:
	\begin{equation}
		\braket{q(v),q(u)} = \braket{v,q(u)} = \braket{q(v),u}.
		\label{eq:scalarqprod}
	\end{equation}
\end{note}
\begin{definition}[Trace and contraction of symmetric tensors]
	Since for symmetric tensors all the positions $i,j (i\neq j)$ of the two vectors that are used to take the trace $tr_{i;j}$ (contraction) are equivalent to each other, we can use a notation that only shows the relevant information: The \emphProp{number} of traces (contractions) taken of one tensor (between two tensors). \\
	$k$-fold trace of a symmetric tensor:
	\begin{align*}
		tr^{k} (\cdot): \Sym{n} &\to \Sym{n-2k},\\
		tr^{k} (s) & \mapsto tr_{1,3,5\cdots, 2k-1;2,4,6,\cdots,2k} (s)
	\end{align*}
	$k$-fold contraction between two symmetric tensors:
	\begin{align*}
		tr^{k} (\cdot,\cdot): \Sym{n}\times\Sym{m} &\to \Sym{n-k}\otimes\Sym{m-k},\\
		tr^{k} (s,t) & \mapsto tr_{1,2,3\cdots, k;1,2,3\cdots,k} (s,t)
	\end{align*}		
\end{definition}

Due to the polynomial nature of the ansatz for the distribution function, we will almost exclusively work with symmetric tensor spaces (reference chapter \ref{chap:isomorphismPolyTensor} to see why). Casually spoken, $O^3$ imposes a specific structure on $\Sym{n}$, which brings us to the next section.

\subsection{Level 1 invariant theory: Irreducable and invariant subspaces of the symmetric tensor space}
\begin{definition}[Invariant subspaces] A subspace $U$ of any vector space $V$ is called \emphName{invariant} with respect to the action of a group $G$, or $G$-invariant, iff for all $r\in G$ and for all $u\in U$, the action of $r$ on $u$ still is in $U$, $r(u)\in U$ or equivilantly: $\forall r\in G: r(U)=U$.	
\end{definition}

\begin{example}[$O^2_{\vc{n}}$-invariant subspaces of $\IR^3$]
	We pick $\IR^3$ as vector space and the subgroup $O^2_{\vc{n}} \subset O^3$ along with a direction $n\in\IR^3$, for which by definition $\forall r \in O^2_{\vc{n}}: r(n)=n$.
	Then the subspace $\spn{\vc{n}}= \{\alpha \vc{n}|\alpha\in\IR\}$ is a one-dimensional invariant subspace, since 
	\[\forall r \in O^2_{\vc{n}}, \forall \alpha\in\IR: r(\alpha \vc{n}) = \alpha r(\vc{n}) = \alpha\vc{n}.\]
	Given an $O^3$-compatible scalar product, we can define the plane $E\subset\IR^3$ orthogonal to $\vc{n}$ and decompose $\IR^3$ into \[\IR^3 = \spn{\vc{n}}\bigoplus E.\] It is easily shown that $E$ is also an invariant, 2-dimensional subspace under the action of $O^2_{\vc{n}}$.
	
	% a general vector $\vc{v}\in\IR^3$ into \[\vc{v}=\underbrace{\vc{v}-\braket{\vc{v},\vc{n}}\vc{n}}_{\in E} + \braket{\vc{v},\vc{n}}\vc{n} =: \vc{v}_{E} + \vc{v}_{\vc{n}}\]
\end{example}

\begin{example}[$O^3$-invariant subspaces of the tensor space]
	These are subspaces $U\subset\mathcal{T}$ that fullfill $\forall r\in O^3: r(U)=U$. One such subspace is the above described space of symmetric tensors, which is of relatively large dimensionality for high $n$. Another example of an invariant subspace would be $\spn{\Omega}\subset\Sym{2}$ with a dimensionality of $1$.
	One may wonder, if $\Sym{n}$ can be further decomposed into "smallest", invariant subspaces, and if such a decomposition is unique.
\end{example}
\begin{definition}[Irreducable (and invariant) subspaces] 
	A subspace $U$ of any vector space $V$ is called \emphName{irreducable} with respect to the action of a group $G$, or \emphName{$G$-irreducable}, iff it is invariant and there is no invariant subspace $W\subset U$ with the properties $W\neq U$ and $W\neq 0$. If a subspace $U$ is invariant and $\dim U=1$, then it is also irreducable, however an irreducable subspace can be of dimension larger than $1$.
	
	Two irreducable subspaces $U_1, U_2, U_1 \neq U_2$ of a vector space $V$ are orthogonal to each other with respect to any scalar product of $V$ that is compatible with the action of the group $G$. We can project from the full vector space onto an irreducable subspace. Due to the universal orthogonality, such a projection is always unique up to a constant (or a zero).
	
\end{definition}
\begin{example}[$O^2_{\vc{n}}$-irreducable subspaces of $\IR^3$]
	We already know that $E$ and $\spn{\vc{n}}$ are invariant subspaces. Since $\dim{\spn{\vc{n}}}=1$, $\spn{\vc{n}}$ is also an irreducable subspace.
	For the 2-dimensional $E$ we cannot find a lower-dimensional invariant subspace, since \[\exists r\in O^2_{\vc{n}}: \forall \vc{v}\neq 0 \in E, \forall \alpha\in\IR: r(\vc{v})\neq \alpha \vc{v}.\] This means that $E$ is irreducable as well and that we have found all $O^2_{\vc{n}}$-irreducable subspaces of $\IR^3$.
\end{example}
\begin{example}[$O^2_{\vc{n}}$-irreducable subspaces of $\Sym{n}$]
	We note:
	\[\Sym{n} = q\left(\left(\IR^3\right)^{\otimes n}\right).\]
	Insert the splitting of $\IR^3$ into $E$ and $\spn{\vc{n}}$:
	\begin{align*}
		\Sym{n} &= q\left(\left(E\bigoplus\spn{\vc{n}}\right)^{\otimes n}\right) \\
		&=\bigoplus_{i=0}^n q\left(E^{\otimes i}\otimes \left(\spn{\vc{n}}\right)^{\otimes n-i}\right)
	\end{align*}
	The orthogonality between $E$ and $\spn{\vc{n}}$ leads to an orthogonality between each of these tensor spaces, the invariancy of each $q\left(E^{\otimes i}\otimes \left(\spn{\vc{n}}\right)^{\otimes n-i}\right)$ follows from the invariancy of the "building blocks" $E$ and $\spn{\vc{n}}$. However, for general $n,i$ these subspaces are not irreducable. For instance we find that the invariant tensor $q(\Omega_E^{\otimes j})$ spans a 1-dimensional (and therefore irreducable) subspace  of $q(E^{\otimes 2j})$. In general, since $q\left(E^{\otimes i}\otimes \left(\spn{\vc{n}}\right)^{\otimes n-i}\right)\cong q(E^{\otimes i})\cong\sym{i}$, finding all irreducable subspaces of $\sym{i}$ immediately leads to all irreducable subspaces of $q\left(E^{\otimes i}\otimes \left(\spn{\vc{n}}\right)^{\otimes n-i}\right)$. 
\end{example}
\begin{remark}[Invariant tensors]
	Sometimes one may work with a tensor $\ten{T}$ that is invariant with respect to a group $G$, i.e. $\forall r\in G: r(T) = T$. For example:
	\begin{itemize}
		\item $G=S_n$. Then all tensors $\ten{T}\in\Sym{n}$ are invariant
		\item $G=O^3$. Then all tensors $\ten{T}\in(\IR^3)^{\otimes 2i}$ that are a linear combination of permutations $\Omega^{\otimes i}$ are invariant. If in addition $\ten{T}\in\Sym{2i}$: $\ten{T}=\alpha q\left(\Omega^{\otimes i}\right), \alpha\in\IR$.
		\item $G=O^2_{\vc{n}}$ and $\ten{T}\in\Sym{n}$. Not all tensors in $q\left(E^{\otimes m}\right)$ are invariant, in fact the only invariant tensors are $\alpha q\left(\Omega_E^{\otimes i}\right), \alpha\in\IR\implies m=2i$. All tensors in $\spn{\vc{n}}$ are invariant, so an $O^2_{\vc{n}}$-invariant, symmetric tensor must come from the subspace spanned by $\left\{q\left(\Omega_E^{\otimes i}\otimes \vc{n}^{\otimes n-2i}\right)\right\}$, i.e.
		\begin{equation}
			\ten{T}=\sum_{i=0}^{\frac{n}{2}} \alpha_i q\left(\Omega_E^{\otimes i}\vc{n}^{\otimes n-2i}\right), \alpha_i \in \IR.
			\label{eq:O2invariantTensors}
		\end{equation}
	\end{itemize}
	Notice, that not all tensors from an invariant subspace are themselves invariant. Even that a tensor lies in an irreducable subspace is no guarantee that the tensor itself is invariant. 
\end{remark}

\begin{example}[$O^3$-irreducable subspaces of $\Sym{n}$]
	It turns out, that $\Sym{n}$ can be expressed as the direct sum of $\left\lfloor\frac{n}{2}\right\rfloor+1$ $O^3$-irreducable subspaces (i.e. $\Sym{n}$ is \emphName{semi-simple}). Most of these subspaces are of dimensionality greater than $1$. Furthermore, these irreducable subspaces are \emphProp{pairwise orthogonal} with respect to \emphProp{any} scalar product of the $\Sym{n}$ that is compatible with $O^3$ (which is a rather redundant requirement, since we wouldn't want to choose a scalar product that is not compatible with $O^3$ anyway).
	
	For every order $n$, we find that the largest irreducable subspace of $\Sym{n}$ consists of trace-free tensors.
	\begin{definition}[Symmetric trace-free tensors]
		A symmetric tensor $v$ of order $n>1$ is called symmetric trace-free, iff $tr_{1,2}(v)=0$. Tensors of order 1 are always symmetric tracefree.
		The space spanned by all symmetric trace-free tensors of order $n$ is denoted as $\STF{n}$ and has dimension $2n+1$. On $\STF{n}$ all $O^3$-compatible scalar products of the tensor space are equivalent to each other (they give the same result up to a constant factor). That has the effect that an orthogonal basis of $\STF{n}$ chosen with respect to one scalar product is orthogonal with respect to any other scalar product (as long as both are $O^3$-compatible). In index notation, the projection onto the space of symmetric trace-free tensor is denoted as square brackets around the indices.
	\end{definition}
	The other irreducable subspaces are related to symmetric tracefree tensors:
	\begin{definition}[Symmetric traceable tensors]
		$\ST{n,a}$ is the space of symmetric tensors of order $n+2a$, that result in a trace-free tensor (of order $n$) after taking $i$ traces. Equivilantly: $\ST{n,a}$ is the space of symmetric tensors of order $n+2a$, that can be expressed as the result of taking trace-free tensors (of order $n$), and applying the linear map
		\[q\circ \underbrace{\ten{\Omega}\circ \cdots \circ \ten{\Omega}}_{i \text{ times}}.\]
		This means that there is a bijective map between $\ST{n,i}$ and $\STF{n}$, i.e. $\ST{n,i}$ and $\STF{n}$ are isomorph to each other.
	\end{definition}
	
	\begin{lemma}[Decomposition of $\Sym{n}$ into its $O^3$ irreducable subspaces]
		Each $\STF{i}$ is a building block for subspaces of $\Sym{n}, i\leq n$,
		\[\Sym{n} = \bigoplus_{i=0}^{\lfloor\frac{n}{2}\rfloor} q\left(\Omega^{\otimes i}\otimes \STF{n-2i}\right)=\bigoplus_{i=0}^{\lfloor\frac{n}{2}\rfloor}\ST{n-2i,i}\]
		
	\end{lemma}
	
	When building an orthogonal basis for $\Sym{n}$ for a physical problem (i.e. $O^3$ acts on $\Sym{n}$), the natural choice is the composition of all orthonormal basis sets from each of the subspaces. 
	
\end{example}

\subsection{Level 2 invariant theory: Linear and Bilinear Maps between Irreducable, invariant subspaces of the symmetric tensor space}
\begin{lemma}[Compatible linear Maps between irreducable subspaces (Schur's Lemma)]
	Let $U_1$ and $U_2$ be two invariant, irreducable subspaces of a vector space. A compatible linear map $\phi: U_1 \to U_2$ is only non-zero, iff $U_1$ and $U_2$ are isomorph to each other. All non-zero linear maps $U_1 \to U_2$ are the same up to a constant.
\end{lemma}
Applied to the space of all symmetric tensors, we find non-zero, $O^3$-compatible linear maps only of the form $\ST{n,a}\to\ST{n,b}, n,a,b \in \IN_0$.
\begin{lemma}[$O^3$-compatible, bilinear maps between symmetric trace free tensors]
	An $O^3$-compatible bilinear map $\phi^{n\tilde{n}\hat{n}}: \STF{\tilde{n}}\times\STF{\hat{n}}\to\STF{n}$ can only be non-zero, iff the following conditions apply:
	\begin{align*}
		\tilde{n}+\hat{n} + n \text{ even}\\
		|\tilde{n}-\hat{n}|\leq n\leq \tilde{n}+\hat{n}
	\end{align*}
	Then $\phi^{n\tilde{n}\hat{n}}$ is unique (up to a constant).
\end{lemma}
\begin{proof}
	$\phi^{n\tilde{n}\hat{n}}$ is isomorph to an $O^3$-compatible linear map $\varphi^{n\tilde{n}\hat{n}}: \STF{\tilde{n}}\otimes\STF{\hat{n}}\to\STF{n}$. $\varphi$ can only be non-zero, iff there exists a subspace of $\STF{\tilde{n}}\otimes\STF{\hat{n}}$, that is isomorph to $\STF{n}$, so we are looking for a projection of $\STF{\tilde{n}}\otimes\STF{\hat{n}}$ onto $\ST{n,i}$ with condition \[n+2i=\tilde{n}+\hat{n} \iff \tilde{n}+\hat{n} + n \text{ even}.\]
	Such a projection is always linear, unique, and compatible with $O^3$. Due to the isomorphism between $\STF{n}$ and $\STF{n,i}$, this makes $\varphi$ unique up to a constant, if it exists. 
	
	Let $s$ be any tensor of degree $n$, and $t$ be the resulting tensor from the projection of $t$ onto $\STF{n}$. Projecting a tensor of the form $\Omega \otimes s$ (or some permutation) onto $\STF{n+2}$ is equivalent to projecting $t$ onto $\STF{n+2}$. Since the only linear map from $\STF{n}$ to $\STF{n+2}$ is the zero-map, the projection of $\Omega \otimes s$ onto $\STF{n+2}$ will always result in $0$. This imposes the condition
	\[n\leq \tilde{n}+\hat{n}\]
	onto $n, \tilde{n},$ and $\hat{n}$.
	
	In chapter \ref{chap:tensorOperations} we listed all types of linear, $O^3$-compatible maps, $\varphi^{n\tilde{n}\hat{n}}$ must be a combination of these. The only map that decreases the order of a given tensor is taking the trace of that tensor. For a tensor of the specific form $\STF{\tilde{n}}\otimes\STF{\hat{n}}$, taking the trace within the first $\tilde{n}$ or the last $\hat{n}$ factors will always be zero. The only traces that yield a non-zero result are those taken between the first $\tilde{n}$ and the last $\hat{n}$ factors, i.e. $k$-fold contractions between the two arguments of $\phi$. Since the two arguments are each symmetric, all $k$-fold contractions between them are equivalent. The number of traces $k$ with a non-zero result we can take of a tensor from the space $\STF{\tilde{n}}\otimes\STF{\hat{n}}$ is limited by $\tilde{n}$ and $\hat{n}$, $k\leq \min{\tilde{n},\hat{n}}$. Since $n=\tilde{n}+\hat{n}-2k$, that imposes the condition on $n$:
	\[n \geq \tilde{n}+\hat{n}-2\min(\tilde{n},\hat{n}) = \max(\tilde{n}-\hat{n},\hat{n}-\tilde{n}) = |\tilde{n}-\hat{n}|.\]
	
	%5For all $n$ fullfilling the above conditions, $\ST{n,i}$ with $i=\frac{\tilde{n}+\hat{n}-n}{2}$ is one of the invariant, irreducable subspaces of $\STF{\tilde{n}}\otimes\STF{\hat{n}}$, i.e. $\varphi$ is a projection (up to a constant) onto $\ST{n,i}$ (or 0), making $$.
\end{proof}
\begin{lemma}[$O^3$-compatible, bilinear maps between symmetric tensors]\label{lem:bilinearMapDecomp}
	Every $O^3$-compatible bilinear map $\psi: \Sym{\tilde{m}}\times\Sym{\hat{m}}\to\Sym{m}$ with $m,\tilde{m},\hat{m}\in\IN_0$  can be expressed as the sum of $O^3$-compatible bilinear maps between all combinations of invariant, irreducable subspaces of $\Sym{\tilde{m}}$, $\Sym{\hat{m}}$, and $\Sym{m}$. With
	\begin{align*}
		m&=n+2a, & \tilde{m}&=\tilde{n}+2b, & \hat{m} &= \hat{n} +2c & 	\tilde{n},b,\hat{n},c,n,a &\in \IN_0
	\end{align*}
	we can write
	\begin{align*}
		\psi: &\Sym{\tilde{m}}\times\Sym{\hat{m}}\to\Sym{m}
		\\
		& = \bigoplus_{b=0}^{\lfloor\frac{\tilde{m}}{2}\rfloor}\ST{\tilde{n},b} \times \bigoplus_{c=0}^{\lfloor\frac{\hat{m}}{2}\rfloor}\ST{\hat{n},c} \to \bigoplus_{a=0}^{\lfloor\frac{m}{2}\rfloor}\ST{n,a}
		\\
		&=
		\bigoplus_{b=0}^{\lfloor\frac{\tilde{m}}{2}\rfloor} \bigoplus_{c=0}^{\lfloor\frac{\hat{m}}{2}\rfloor}
		\bigoplus_{a=0}^{\lfloor\frac{m}{2}\rfloor}
		\ST{\tilde{n},b} \times \ST{\hat{n},c} \to \ST{n,a}\\
		&=:\bigoplus_{b=0}^{\lfloor\frac{\tilde{m}}{2}\rfloor} \bigoplus_{c=0}^{\lfloor\frac{\hat{m}}{2}\rfloor}
		\bigoplus_{a=0}^{\lfloor\frac{m}{2}\rfloor} \psi_{abc}^{n\tilde{n}\hat{n}}
	\end{align*}
	Each of the $\psi_{abc}^{n\tilde{n}\hat{n}}$ are unique up to a constant and only non-zero, iff the following conditions apply:
	\begin{align*}
		\tilde{n}+\hat{n} + n \text{ even}\\
		|\tilde{n}-\hat{n}|\leq n\leq \tilde{n}+\hat{n}.
	\end{align*}
	Given a set of all maps $\psi_{abc}^{n\tilde{n}\hat{n}}$, two $\psi$ can only differ in the constants multiplied with those maps. Furthermore, due to the isomorphism between $\STF{n}$ and $\ST{n,i}$, all $\psi_{abc}^{n\tilde{n}\hat{n}}$  are isomorph to a map $Q^{n\tilde{n}\hat{n}}: \STF{\tilde{n}}\otimes\STF{\hat{n}}\to\STF{n}$, which is unique up to a constant. 
	Given a set of all maps $Q^{n\tilde{n}\hat{n}}$, two tensors of the form $U:=q\left(\Omega^{\otimes b}\otimes u\right)$ and $V:=q\left(\Omega^{\otimes b}\otimes v\right)$ with $u\in\STF{\tilde{n}},v\in\STF{\hat{n}}$, we can express any $\psi_{abc}^{n\tilde{n}\hat{n}}(U,V)$ as
	
	%	Given a set of all maps $Q^{n\tilde{n}\hat{n}}$, we can express any $\psi_{abc}^{n\tilde{n}\hat{n}}$ as
	%	\begin{align*}
	%	\psi_{abc}^{n\tilde{n}\hat{n}}(u,v) = S_{abc}^{n\tilde{n}\hat{n}} q\circ\underbrace{\ten{\Omega}\circ\cdots\circ\ten{\Omega}}_{a \text{ times}}\circ Q^{n\tilde{n}\hat{n}}(tr^b(u),tr^c(v)), S_{abc}^{n\tilde{n}\hat{n}} \in \IR.
	%	\end{align*}
	%	Given two tensors of the form $q\left(\Omega^{\otimes b}\otimes u\right)$ and $q\left(\Omega^{\otimes b}\otimes v\right)$ with $u\in\STF{\tilde{n}},v\in\STF{\hat{n}}$, this simplifies to:
	
	\begin{align}
		\psi_{abc}^{n\tilde{n}\hat{n}}\left(q\left(\Omega^{\otimes b}\otimes u\right),q\left(\Omega^{\otimes b}\otimes v\right)\right) = S_{abc}^{n\tilde{n}\hat{n}} \cdot q\left(\Omega^{\otimes a}\otimes Q^{n\tilde{n}\hat{n}}(u,v)\right).
		\label{eq:bilinearMapDecomp}
	\end{align}
	
	\textbf{The only degree of freedom for $O^3$-compatible, bilinear maps $\psi$ between symmetric tensors lies in the choice for the constants $S_{abc}^{n\tilde{n}\hat{n}}$.}
\end{lemma}

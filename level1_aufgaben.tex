% !TeX root = spherical_harmonics.tex
% !TeX spellcheck = de_DE
\begin{sheet}

\begin{problem}[title={Satz von Schur über Körpern $\neq\IC$}]
In Aufgabe~\ref{ex:fundamentaldarstellung_von_so2} wurde gezeigt, dass die kanonische Darstellung $V=\IR^2$ von $G=SO_2$ irreduzibel ist. Zeige, dass $\End_{\IR G}(V)$ 2-dimensional ist.
\end{problem}

\begin{problem}[title={Der Satz von Maschke ist falsch für nichtkompakte Gruppen}, difficulty={leichter als es aussieht}]
Zeige, dass nicht jede endlich-dimensionale Darstellung von $G=\IZ$ in irreduzible Unterdarstellungen zerlegt werden kann.

Hinweis: Das ist eine leicht getarnte Aussage, die bereits aus der LA-I-Vorlesung bekannt ist.
\end{problem}

\begin{problem}[title={Isomorphismus zwischen irreduziblen Darstellungen von Polynomen und Tensoren}]
Wir wissen, dass sich der Raum der symmetrischen Tensoren in irreduzible zerlegt wie folgt:
\[\Sym^m(V) \isomorphic \STF^m(V) \oplus \STF^{m-2}(V) \oplus \STF^{m-4}(V) \oplus \cdots\]
Und wir wissen, dass $\Sym^m(V)$ zum Raum der homogenen Polynome vom Grad $m$ isomorph ist. Also zerlegt sich diese Darstellung ganz genauso. Wir wollen den Isomorphismus explizit beschreiben.
\begin{subproblem}
Erinnerung: Der Polynom-Tensor-Isomorphismus identifiziert $\STF^m(V)$ mit dem Unterraum der \emph{harmonischen}, homogenen Polynome vom Grad $m$.

Finde Unterdarstellungen vom Polynomraum vom Grad $m$, die zum Raum der harmonischen Polynome vom Grad $m-2k$ isomorph sind.
\end{subproblem}
\begin{subproblem}
Benutze den Polynom-Tensor-Isomorphismus in der anderen Richtung um die entsprechenden Unterdarstellung von $\Sym^m(V)$ zu finden, die zu $\STF^{m-2k}(V)$ isomorph sind.
\end{subproblem}
\end{problem}


\begin{problem}[title={Isomorphismus zwischen irreduziblen Darstellungen von Polynomen und Tensoren II}]
Mit Hilfe des Isomorphismus zwischen $\Sym^l(V)$ und dem Raum der homogenen, polynomiellen Abbildungen $V\to\IK$ vom Grad $l$ können wir mehr Strukturen als nur Unterdarstellungen von einer in die andere Sprache übersetzen.
\begin{subproblem}
Zeige, dass der Isomorphismus die Spur $\tr: \Sym^l(V) \to \Sym^{l-2}(V)$ in $\frac{1}{l(l-1)}\Delta$ übersetzt.
\end{subproblem}
\begin{subproblem}
Zeige, dass der Isomorphismus das Standardskalarprodukt (also das von $V^{\otimes l}$ eingeschränkt auf den Unterraum $\Sym^l(V)$) übersetzt in $\braket{p,q} = p(\partial)q$. Wobei $p(\partial)$ den Differentialoperator meint, der entsteht, wenn man jedes $x$ in $p$ durch $\partial_x$, jedes $y$ durch $\partial_y$ und jedes $z$ durch $\partial_z$ ersetzt.
\end{subproblem}
\end{problem}

\end{sheet}
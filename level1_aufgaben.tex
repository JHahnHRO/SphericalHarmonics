% !TeX root = spherical_harmonics.tex
% !TeX spellcheck = de_DE
\begin{sheet}

\begin{problem}[title={Satz von Schur über Körpern $\neq\IC$}]
In \ref{ex:fundamentaldarstellung_von_so2} wurde gezeigt, dass die kanonische Darstellung $V=\IR^2$ von $G=SO_2$ irreduzibel ist. Zeige, dass $\End_{\IR G}(V)$ 2-dimensional ist.
\end{problem}

\begin{problem}[title={Der Satz von Maschke ist falsch für nichtkompakte Gruppen}, difficulty={leichter als es aussieht}]
Zeige, dass nicht jede endlich-dimensionale Darstellung von $G=\IZ$ in irreduzible Unterdarstellungen zerlegt werden kann.

Hinweis: Das ist eine leicht getarnte Aussage, die bereits aus der LA-I-Vorlesung bekannt ist.
\end{problem}


\begin{problem}
Es sei $V = V_1^{\oplus m_1} \oplus V_2^{m_2} \oplus \cdots \oplus V_k^{m_k}$ eine Darstellung von $G$, wobei die $V_i$ paarweise nicht-isomorphe, irreduzible Unterdarstellungen seien. Zeige, dass
\[\dim \End_{\IK G}(V) = m_1^2 + m_2^2 + \ldots + m_k^2\]
gilt.
\end{problem}

\begin{problem}\label{ex:keine_2D_darstellung_von_so3}
Zeige, dass die einzigen reellen oder komplexen Darstellungen von $G=SO_3$ von Dimension $<3$ trivial sind.

Hinweis: Verwende, dass laut Satz von Maschke die Darstellung $\Delta: G\to SO(V)$ gewählt werden kann.
\end{problem}

\end{sheet}
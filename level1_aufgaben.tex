% !TeX root = spherical_harmonics.tex
% !TeX spellcheck = de_DE
\begin{sheet}

\begin{problem}[title={Satz von Schur über Körpern $\neq\IC$}]
In Aufgabe~\ref{ex:fundamentaldarstellung_von_so2} wurde gezeigt, dass die kanonische Darstellung $V=\IR^2$ von $G=SO_2$ irreduzibel ist. Zeige, dass $\End_{\IR G}(V)$ 2-dimensional ist.
\end{problem}

\begin{problem}[title={Der Satz von Maschke ist falsch für nichtkompakte Gruppen}, difficulty={leichter als es aussieht}]
Zeige, dass nicht jede endlich-dimensionale Darstellung von $G=\IZ$ in irreduzible Unterdarstellungen zerlegt werden kann.

Hinweis: Das ist eine leicht getarnte Aussage, die bereits aus der LA-I-Vorlesung bekannt ist.
\end{problem}


\begin{problem}[title={Dimension von Endomorphismenräumen}]
Es sei $V = V_1^{\oplus m_1} \oplus V_2^{m_2} \oplus \cdots \oplus V_k^{m_k}$ eine Darstellung von $G$, wobei die $V_i$ paarweise nicht-isomorphe, irreduzible Unterdarstellungen seien. Zeige, dass
\[\dim \End_{\IK G}(V) = m_1^2 + m_2^2 + \ldots + m_k^2\]
gilt.
\end{problem}

\begin{remark}
Diese Aussage ist bemerkenswert nützlich darin, hübsche zahlentheoretische Aussagen über Summen von Quadratzahlen zu beweisen.
\end{remark}

\begin{problem}[title={2D-Darstellungen von $SO_3$}]
	\label{ex:keine_2D_darstellung_von_so3}
Zeige, dass die einzigen reellen oder komplexen Darstellungen von $G=SO_3$ von Dimension $<3$ trivial sind.

Hinweis: Verwende, dass laut Satz von Maschke die Darstellung $\mathcal{D}: G\to SO(V)$ gewählt werden kann.
\end{problem}

\begin{problem}[title={Isomorphismus zwischen irreduziblen Darstellungen von Polynomen und Tensoren}]
Wir wissen, dass sich der Raum der symmetrischen Tensoren in irreduzible zerlegt wie folgt:
\[\Sym^m(V) \isomorphic \STF^m(V) \oplus \STF^{m-2}(V) \oplus \STF^{m-4}(V) \oplus \cdots\]
Und wir wissen, dass $\Sym^m(V)$ zum Raum der homogenen Polynome vom Grad $m$ isomorph ist. Also zerlegt sich diese Darstellung ganz genauso. Wir wollen den Isomorphismus explizit beschreiben.
\begin{subproblem}
Erinnerung: Der Polynom-Tensor-Isomorphismus identifiziert $\STF^m(V)$ mit dem Unterraum der \emph{harmonischen}, homogenen Polynome vom Grad $m$.

Finde Unterdarstellungen vom Polynomraum vom Grad $m$, die zum Raum der harmonischen Polynome vom Grad $m-2k$ isomorph sind.
\end{subproblem}
\begin{subproblem}
Benutze den Polynom-Tensor-Isomorphismus in der anderen Richtung um die entsprechenden Unterdarstellung von $\Sym^m(V)$ zu finden, die zu $\STF^{m-2k}(V)$ isomorph sind.
\end{subproblem}
\end{problem}

\end{sheet}
Wir kennen uns bereits mit Vektorräumen aus. Typische Beispiele sind $\IR^3$ oder der Raum der Polynome von einem bestimmten Grad $n$. Da uns in diesem Kurs noch weitere Vektorräume begegnen werden und zum Vergleich mit anderen Mathematischen Objekten (z.B. Algebren), ist hier ihre Definition zusammengefasst.


\begin{definition}[Vektorräume]\label{vektorraeume:def}
	Sei $K$ ein Körper. Ein \udot{$K$-Vektorraum} $(V,+,\cdot)$ besteht aus
	\begin{itemize}
		\item einer Menge $V$,
		\item einer Abbildung $+: V \times V \to V, (v,w) \mapsto v+w$, genannt \udot{(Vektor)addition} und
		\item einer Abbildung $\cdot: K \times V \to V, (\lambda,v) \mapsto \lambda v$, genannt \udot{Skalarmultiplikation}.
	\end{itemize}
	die mit $\vc{u},\vc{v},\vc{w}\in V$ und $a,b \in K$ die folgenden Bedingungen (Axiome) erfüllen:

		
		\begin{table}[!ht]
			\setlength\extrarowheight{10pt} % for a bit of visual "breathing space"
			\begin{tabularx}{\textwidth}{C X}
				
				\toprule
				\textbf{Axiom}                                                          & \textbf{Bedeutung}                                                                                                                                     \\ \midrule
				Assoziativität der Addition                                               & $\vc{v}+(\vc{w}+\vc{u}) = (\vc{v} + \vc{w}) + \vc{u}$                                                                                                \\
				Kommutativität der Addition                                               & $\vc{u} + \vc{v} = \vc{v} + \vc{u}$                                                                                                                  \\
				Neutrales Element der Addition                                            & $\exists\vc{0} \in V$, gennannt der Nullvektor, sodass $\forall\vc{v} \in V: \vc{v} + \vc{0} = \vc{v}$.                         \\
				Inverses Elements der Addition                                            & $\forall\vc{v} \in V\exists \vc{-v} \in V$, gennant das additive Inverse zu $\vc{v}$, sodass $\vc{v} + (\vc{-v}) = \vc{0}$. \\
				Kompatibilität der skalaren Multiplikation mit der Multiplikation vom Körper        & $a\cdot(b\cdot\vc{v}) = (a\cdot b)\cdot\vc{v}$                                                                                                      \\
				Neutrales Element der skalaren Multiplikation                               & $\forall \vc{v} \in V \exists 1\in K: 1 \cdot \vc{v} = \vc{v}$                           \\
				Distributivität der skalaren Multiplikation bezüglich der Vektoraddition & $a\cdot(\vc{u} + \vc{v}) = a\cdot\vc{u} + a\cdot\vc{v}$                                                                                              \\
				Distributivität der skalaren Multiplikation bezüglich der Addition im Körper  & $(a + b)\cdot\vc{v} = a\cdot\vc{v} + b\cdot\vc{v}$                                                                                                   \\ \bottomrule
			\end{tabularx}
			\caption{Eigenschaften eines Vektorraums}
		\end{table}
\end{definition}
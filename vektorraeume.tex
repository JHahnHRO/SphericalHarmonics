Wir kennen uns bereits mit Vektorräumen aus. Typische Beispiele sind $\IR^3$ oder der Raum der Polynome von einem bestimmten Grad $n$. Da uns in diesem Kurs noch weitere Vektorräume begegnen werden und zum Vergleich mit anderen Mathematischen Objekten (z.B. Algebren), ist hier ihre Definition zusammengefasst.


\begin{definition}[Vektorräume]\label{vektorraeume:def}
	Sei $K$ ein Körper. Ein \udot{$K$-Vektorraum} $(V,+,\cdot)$ besteht aus
	\begin{itemize}
		\item einer Menge $V$,
		\item einer Abbildung $+: V \times V \to V, (v,w) \mapsto v+w$, genannt \udot{(Vektor)addition} und
		\item einer Abbildung $\cdot: K \times V \to V, (\lambda,v) \mapsto \lambda v$, genannt \udot{Skalarmultiplikation}.
	\end{itemize}
	die die folgenden Bedingungen (Axiome) erfüllen:
		
		\begin{table}[!ht]
			\setlength\extrarowheight{10pt} % for a bit of visual "breathing space"
			\begin{tabularx}{\textwidth}{p{7cm} X}
				
				\toprule
				\textbf{Axiom}                                                              & \textbf{Bedeutung} \\
				\midrule
				Assoziativität der Addition                                                 & $\forall u,v,w\in V: u+(v+w) = (u+v)+w$  \\
				Kommutativität der Addition                                                 & $\forall u,v\in V: u+v=v+u$ \\
				Neutrales Element der Addition                                              & $\exists 0\in V\forall v\in V: v+0=0+v$  \\
				Inverses Elements der Addition                                              & $\forall v \in V\exists w \in V: v+w=0$ \\
				Assoziativität der Skalarmultiplikation                                     & $\forall a,b\in K, v\in V: a\cdot(b\cdot v) = (a\cdot b)\cdot v$ \\
				Normierung                                                                  & $\forall v \in V: 1 \cdot v = v$, wobei $1$ das Einselement des Körpers bezeichnet \\
				Distributivität der Skalarmultiplikation bezüglich der Vektoraddition       & $\forall a\in K, u,v\in V: a\cdot(u + v) = a\cdot u + a\cdot v$ \\
				Distributivität der Skalarmultiplikation bezüglich der Addition des Körpers & $\forall a,b\in K, v\in V: (a + b)\cdot v = a\cdot v + b\cdot v$ \\
				\bottomrule
			\end{tabularx}
			\caption{Eigenschaften eines Vektorraums}
		\end{table}
\end{definition}
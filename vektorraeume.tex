% !TeX spellcheck = de_DE
% !TeX root = spherical_harmonics.tex
\subsection{Vektorräume}
Wir kennen uns bereits mit Vektorräumen aus. Typische Beispiele sind $\IR^3$ oder der Raum der Polynome von einem bestimmten Grad $n$. Da uns in diesem Kurs noch weitere Vektorräume begegnen werden und zum Vergleich mit anderen mathematischen Objekten (z.B. Darstellungen), ist hier ihre Definition zusammengefasst.

\begin{definition}[Vektorräume und lineare Abbildungen]\label{vektorraeume:def}
Sei $K$ ein Körper. Ein \udot{$K$-Vektorraum} $(V,+,\cdot)$ besteht aus
\begin{itemize}
	\item einer Menge $V$,
	\item einer Abbildung $+: V \times V \to V, (v,w) \mapsto v+w$, genannt \emph{(Vektor)addition} und
	\item einer Abbildung $\cdot: K \times V \to V, (\lambda,v) \mapsto \lambda\cdot v$, genannt \emph{Skalarmultiplikation}.
\end{itemize}
die die Vektorraum-Axiome in Tabelle \ref{vektorraeume:def_table} erfüllen. Alles, was Element eines Vektorraums ist, kann \emph{Vektor} genannt werden.

\begin{table}[!ht]
	\setlength\extrarowheight{10pt} % for a bit of visual "breathing space"
	\begin{tabularx}{\textwidth}{p{7cm} X}
		
		\toprule
		\textbf{Vektorraum-Axiom}                            & \textbf{Bedeutung} \\
		\midrule
        $(V,+)$ ist eine abelsche Gruppe                     & Addition verhält sich wie erwartet \\
		\hspace{1cm}Assoziativität                           & $\forall u,v,w\in V: u+(v+w) = (u+v)+w$  \\
		\hspace{1cm}Neutrales Element bzw. \enquote{Null}    & $\exists 0\in V\forall v\in V: v+0=0+v$  \\
		\hspace{1cm}Inverse bzw. \enquote{negative} Elemente & $\forall v \in V\exists w \in V: v+w=0$ \\
        \hspace{1cm}Kommutativität                           & $\forall u,v\in V: u+v=v+u$ \\
        Eigenschaften der Skalarmultiplikation \\
		\hspace{1cm}Assoziativität                           & $\forall a,b\in K, v\in V: a\cdot(b\cdot v) = (a\cdot b)\cdot v$ \\
		\hspace{1cm}Normierung bzw. Nichttrivialität         & $\forall v \in V: 1 \cdot v = v$, wobei $1$ das Einselement des Körpers bezeichnet \\
        Verträglichkeit von Addition und Skalarmultiplikation \\
		\hspace{1cm}Distributivität bzgl. $(V,+)$            & $\forall a\in K, u,v\in V: a\cdot(u + v) = a\cdot u + a\cdot v$ \\
		\hspace{1cm}Distributivität bzgl. $(K,+)$            & $\forall a,b\in K, v\in V: (a + b)\cdot v = a\cdot v + b\cdot v$ \\
        \midrule
        \textbf{Axiome linearer Abbildungen}                  & \textbf{Bedeutung} \\
        \midrule
        Additivität & $\forall v_1,v_2\in V: f(v_1+v_2) = f(v_1)+f(v_2)$ \\
        Homogenität & $\forall \lambda\in K, v\in V: f(\lambda\cdot v) = \lambda \cdot f(v)$ \\
        \bottomrule
	\end{tabularx}
	\caption{Definierende Eigenschaften von Vektorräumen und linearen Abbildungen}
    \label{vektorraeume:def_table}
\end{table}

Sind $V,W$ zwei $K$-Vektorräume und $f: V\to W$ eine Abbildung, so heißt $f$ \emph{($K$-)lineare Abbildung} oder \emph{(Vektorraum-)Homomorphismus}, falls die beiden Axiome in Tabelle \ref{vektorraeume:def_table} erfüllt sind. Den Raum aller $K$-linearen Abbildungen von $V$ nach $W$ bezeichnen wir mit $\Hom_K(V,W)$. Eine $K$-linearen Abbildung von $V$ nach $V$ (also gleicher Definitions- und Zielraum) heißt \emph{(Vektorraum-)Endomorphismus}, der Raum aller solcher Abbildungen wird mit $\End_K(V)$ notiert.

Existiert ein Homomorphismus $f': W\to V$ mit $f\circ f'=f'\circ f=\id$, so nennt man $f$ \emph{Isomorphismus} der Vektorräume.
\end{definition}

\begin{remark}
Wir sind praktisch ausschließlich an $\IR$- und $\IC$-Vektorräumen interessiert in diesem Kurs.
\end{remark}

\begin{definition}[Kern \& Bild]
Ist $f: V\to W$ eine $K$-lineare Abbildung, so ist
\[\ker(f) := \Set{v\in V | f(v)=0}\]
der \emph{Kern von $f$} und
\[\im(f) := \Set{w\in W | \exists v\in V: f(v)=w}\]
das \emph{Bild von $f$}.
\end{definition}

\subsection{Bilineare Abbildungen}

\begin{remark}
	Es gibt i.A. keine Multiplikation zweier Vektoren miteinander in irgendeinem Sinne. Wir können immer Skalare mit Vektoren multiplizieren, aber nicht Vektoren mit Vektoren. Nichts desto trotz ist es \emph{manchmal} doch so, dass zusätzlich zu Addition und Skalarmultiplikation eine weitere Operation existiert, die ein sinnvolles Konzept von Multiplikation liefert, z.B.
	\begin{enumerate}
		\item Der Vektorraum der Polynome $\IK[X]$ hat die Polynommultiplikation, d.h.
		\[\left(\sum_{i=0}^n a_i X^i\right) \cdot \left(\sum_{j=0}^m b_j X^j\right) := \sum_{k=0}^{n+m} (\sum_{\substack{i,j \\ i+j=k}} a_i b_j) X^k\]
		\item Der Vektorraum der Funktionen $X\to\IK$ für einen festen Definitionsbereich $X$ hat die punktweise Multiplikation, d.h.
		\[f\cdot g := x\mapsto f(x)g(x)\]
		\item Die Hintereinanderausführung von linearen Abbildungen $\Hom_\IK(V,W) \times \Hom_\IK(U,V) \to \Hom_\IK(U,W), (f,g) \mapsto f\circ g$ ist eine Abbildung, die sich in vielerlei Hinsicht auch wie eine Multiplikation verhält. Für $U=V=W$ erhält man insbesondere eine Multiplikation $\End_\IK(V) \times \End_\IK(V) \to \End_\IK(V)$.
		
		Nach Wahl von je einer Basis können wir $\Hom_\IK(V,W)$, $\Hom_\IK(U,V)$ und $\Hom_\IK(U,W)$ mit $\IK^{n\times m}$, $\IK^{m\times p}$ und $\IK^{n\times p}$ identifizieren. Die Hintereinanderausführung von linearen Abbildung entspricht dann der Matrixmultiplikation.
		\item ...
	\end{enumerate}
	
\end{remark}

\begin{definition}[Bilineare \& multilineare Abbildungen]\label{bilineare_abb:def}
	Sind $V,W,X$ drei $\IK$-Vektorräume, so heißt eine Abbildung $\phi: V\times W\to X$ \emph{bilinear}, falls sie beiden Linearitätsbedingungen in Tabelle \ref{bilineare_abb:def_table} erfüllt, d.h. die Funktion ist separat linear, wenn man nur den ersten oder nur den zweiten Input variiert und den anderen festhält.
	
	\begin{table}[!ht]
		\setlength\extrarowheight{10pt} % for a bit of visual "breathing space"
		\hspace{-0.1\textwidth}
		\begin{tabularx}{1.2\textwidth}{p{6.5cm} p{9.5cm}}
			\toprule
			\textbf{Axiom}                  & \textbf{Bedeutung} \hspace{0.5cm} \\ 
			\midrule
			Linearität: \\
			\hspace{1cm}im ersten Argument  & $\forall v_1,v_2\in V, w\in W: \phi(v_1+v_2,w) = \phi(v_1,w) + \phi(v_2,w)$ \\
			& $\forall \lambda\in\IK, v\in W, w\in W: \phi(\lambda v, w) = \lambda \phi(v,w)$ \\
			\hspace{1cm}im zweiten Argument & $\forall v\in V, w_1, w_2\in W: \phi(v,w_1+w_2) = \phi(v,w_1) + \phi(v,w_2)$ \\
			& $\forall \lambda\in\IK, v\in W, w\in W: \phi(v, \lambda w) = \lambda \phi(v,w)$ \\
			Falls $\phi: V\times V \to X$, kann optional gelten: \\
			\hspace{1cm}Symmetrie bzw. Kommutativität & $\forall u,v\in V: \phi(u,v) = \phi(v,u)$ \\
			\hspace{1cm}Antisymmetrie & $\forall u,v\in V: \phi(u,v) = -\phi(v,u)$ \\
			Falls $\phi: V\times V\to V$, kann optional gelten: \\
			\hspace{1cm}Assoziativität & $\forall u,v,w\in V: \phi(u,\phi(v,w)) = \phi(\phi(u,v),w)$ \\
			\bottomrule
		\end{tabularx}
		\label{bilineare_abb:def_table}
		\caption{Eigenschaften von bilinearen Abbildungen}
	\end{table}
	
	Für Abbildungen $V_1\times V_2\times V_3\to X$, die von drei Inputvektoren abhängig sind, kann man entsprechend definieren, dass eine Abbildung \emph{trilinear} heißt, wenn sie die drei Distributivgesetze erfüllt. Für vier, fünf, ... $k$ Input-Vektoren spricht man entsprechend von \emph{$k$-fach linearen} Abbildungen.
\end{definition}

\begin{example}
	Die drei genannten \enquote{Multiplikationen} von oben sind bilinear. Polynommultiplikation und punktweise Multiplikation von Funktionen sind kommutativ und assoziativ. Die Komposition $\End_\IK(V)\times\End_\IK(V)\to\End_\IK(V)$ ist assoziativ, aber nicht kommutativ, falls $\dim(V) > 1$.
	
	Es gibt viele weitere, äußerst nützliche Beispiele.
	\begin{enumerate}[start=4]
		\item Richtungsableitungen sind bilinear in der Richtung und der Funktion:
		
		Sei $X\subseteq\IR^n$ ein geeigneter Definitionsbereich (z.B. eine offene Menge). Sei außerdem $f:X \to\IC$ eine differenzierbare Funktion. Dann existieren insbesondere alle Richtungsableitungen $(\partial_vf)(x_0) = \lim_{t\to 0} \frac{f(x_0)-f(x_0+tv)}{t}$ in allen Punkten $x_0\in X$. Die Abbildungsvorschrift $(v,f) \mapsto \partial_v f$ liefert eine bilineare Abbildung für mehrere Kombinationen von Vektorräumen, z.B. $\IR^n \times C^1(\IR^n) \to C^0(\IR^n)$ oder $\IR^n \times C^\infty(\IR^n) \to C^\infty(\IR^n)$.
		
		\item Allgemeiner: Lineare Differentialoperatoren:
		
		Für mehrfach differenzierbare Funktionen kann man natürlich auch mehrere Ableitungsschritte hintereinander ausführen. Auf diese Weise erhält man multilineare Abbildungen: $k$-faches Ableiten in $k$ Richtungen, also der Differentialoperator $\partial_{v_1} \partial_{v_2} \cdots \partial_{v_k}$ ist $k$-fach linear in den Vektoren $v_1, ..., v_k$ als Inputs. Die Anwendung auf eine Funktion ist entsprechend $(k+1)$-fach linear in den $k$ Richtungsvektoren und der Funktion als Inputs.
		
		Als $k$-lineare Abbildung ist dies eine symmetrische Abbildung, d.h. $\partial_v \partial_w f = \partial_w\partial_v f$, sofern $f$ zweimal stetig differenzierbar ist. Das ist der Satz von Schwarz\footnote{Hermann Amandus Schwarz (1843 -- 1921), dt. Mathematiker.}.
		
		\item Kreuzprodukt:
		
		Es gibt eine besondere bilineare Abbildung $\textsc{x} : \IR^3 \times \IR^3 \to \IR^3$, genannt \udot{Kreuzprodukt}, die in anderen Dimensionen keine gute Analogie hat, nämlich
		
		\[\begin{pmatrix}x_1\\y_1\\z_1\end{pmatrix} \textsc{x} \begin{pmatrix}x_2\\y_2\\z_2\end{pmatrix} := \begin{pmatrix}
			y_1 z_2 - y_2 z_1 \\
			x_1 z_2 - x_2 z_1 \\
			x_1 y_2 - x_2 y_1
		\end{pmatrix}\]
		Es ist bilinear, aber weder kommutativ noch assoziativ. Stattdessen ist es antisymmetrisch.
	\end{enumerate}
\end{example}

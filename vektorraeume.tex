% !TeX spellcheck = de_DE
% !TeX root = spherical_harmonics.tex
Wir kennen uns bereits mit Vektorräumen aus. Typische Beispiele sind $\IR^3$ oder der Raum der Polynome von einem bestimmten Grad $n$. Da uns in diesem Kurs noch weitere Vektorräume begegnen werden und zum Vergleich mit anderen mathematischen Objekten (z.B. Darstellungen), ist hier ihre Definition zusammengefasst.


\begin{definition}[Vektorräume und lineare Abbildungen]\label{vektorraeume:def}
Sei $K$ ein Körper. Ein \udot{$K$-Vektorraum} $(V,+,\cdot)$ besteht aus
\begin{itemize}
	\item einer Menge $V$,
	\item einer Abbildung $+: V \times V \to V, (v,w) \mapsto v+w$, genannt \emph{(Vektor)addition} und
	\item einer Abbildung $\cdot: K \times V \to V, (\lambda,v) \mapsto \lambda\cdot v$, genannt \emph{Skalarmultiplikation}.
\end{itemize}
die die Vektorraum-Axiome in Tabelle \ref{vektorraeume:def_table} erfüllen. Alles, was Element eines Vektorraums ist, kann \emph{Vektor} genannt werden.

\begin{table}[!ht]
	\setlength\extrarowheight{10pt} % for a bit of visual "breathing space"
	\begin{tabularx}{\textwidth}{p{7cm} X}
		
		\toprule
		\textbf{Vektorraum-Axiom}                            & \textbf{Bedeutung} \\
		\midrule
        $(V,+)$ ist eine abelsche Gruppe                     & Addition verhält sich wie erwartet \\
		\hspace{1cm}Assoziativität                           & $\forall u,v,w\in V: u+(v+w) = (u+v)+w$  \\
		\hspace{1cm}Neutrales Element bzw. \enquote{Null}    & $\exists 0\in V\forall v\in V: v+0=0+v$  \\
		\hspace{1cm}Inverse bzw. \enquote{negative} Elemente & $\forall v \in V\exists w \in V: v+w=0$ \\
        \hspace{1cm}Kommutativität                           & $\forall u,v\in V: u+v=v+u$ \\
        Eigenschaften der Skalarmultiplikation \\
		\hspace{1cm}Assoziativität                           & $\forall a,b\in K, v\in V: a\cdot(b\cdot v) = (a\cdot b)\cdot v$ \\
		\hspace{1cm}Normierung bzw. Nichttrivialität         & $\forall v \in V: 1 \cdot v = v$, wobei $1$ das Einselement des Körpers bezeichnet \\
        Verträglichkeit von Addition und Skalarmultiplikation \\
		\hspace{1cm}Distributivität bzgl. $(V,+)$            & $\forall a\in K, u,v\in V: a\cdot(u + v) = a\cdot u + a\cdot v$ \\
		\hspace{1cm}Distributivität bzgl. $(K,+)$            & $\forall a,b\in K, v\in V: (a + b)\cdot v = a\cdot v + b\cdot v$ \\
        \midrule
        \textbf{Axiome linearer Abbildungen}                  & \textbf{Bedeutung} \\
        \midrule
        Additivität & $\forall v_1,v_2\in V: f(v_1+v_2) = f(v_1)+f(v_2)$ \\
        Homogenität & $\forall \lambda\in K, v\in V: f(\lambda\cdot v) = \lambda \cdot f(v)$ \\
        \bottomrule
	\end{tabularx}
	\caption{Definierende Eigenschaften von Vektorräumen und linearen Abbildungen}
    \label{vektorraeume:def_table}
\end{table}

Sind $V,W$ zwei $K$-Vektorräume und $f: V\to W$ eine Abbildung, so heißt $f$ \emph{($K$-)lineare Abbildung} oder \emph{(Vektorraum-)Homomorphismus}, falls die beiden Axiome in Tabelle \ref{vektorraeume:def_table} erfüllt sind. Den Raum aller $K$-linearen Abbildungen von $V$ nach $W$ bezeichnen wir mit $\Hom_K(V,W)$. Eine $K$-linearen Abbildung von $V$ nach $V$ (also gleicher Definitions- und Zielraum) heißt \emph{(Vektorraum-)Endomorphismus}, der Raum aller solcher Abbildungen wird mit $\End_K(V)$ notiert.

Existiert ein Homomorphismus $f': W\to V$ mit $f\circ f'=f'\circ f=\id$, so nennt man $f$ \emph{Isomorphismus} der Vektorräume.
\end{definition}

\begin{remark}
Wir sind praktisch ausschließlich an $\IR$- und $\IC$-Vektorräumen interessiert in diesem Kurs.
\end{remark}

\begin{definition}[Kern \& Bild]
Ist $f: V\to W$ eine $K$-lineare Abbildung, so ist
\[\ker(f) := \Set{v\in V | f(v)=0}\]
der \emph{Kern von $f$} und
\[\im(f) := \Set{w\in W | \exists v\in V: f(v)=w}\]
das \emph{Bild von $f$}.
\end{definition}
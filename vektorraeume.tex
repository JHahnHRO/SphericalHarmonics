

You should be familiar with vector spaces. Typical expamples include $\IR^3$ or the space of polynomials up to a maximal degree $n$. Let us recall the definition of a vector space:

\begin{definition}[Vektorräume]\label{vektorraeume:def}
	Sei $K$ ein Körper. Ein \udot{$K$-Vektorraum} $(V,+,\cdot)$ besteht aus
	\begin{itemize}
		\item einer Menge $V$,
		\item einer Abbildung $+: V \times V \to V, (v,w) \mapsto v+w$, genannt \udot{(Vektor)addition} und
		\item einer Abbildung $\cdot: K \times V \to V, (\lambda,v) \mapsto \lambda v$, genannt \udot{Skalarmultiplikation}.
	\end{itemize}
	die die folgenden Bedingungen erfüllen:
	\begin{description}
		\item{(V1)} $(V,+)$ ist eine abelsche Gruppe. Das neutrale Element nennt man den \udot{Nullvektor} und schreibt man fast immer als $0$.
		\item{(V2)} Assoziativität der Skalarmultiplikation:
		\[\forall\lambda,\mu\in K \forall v\in V: (\lambda \mu)v = \lambda(\mu v)\]
		\item{(V3)} Distributivität in $v$:
		\[\forall\lambda\in K \forall v,w\in V: \lambda (v+w) = \lambda v + \lambda w\]
		\item{(V4)} Distributivität in $\lambda$:
		\[\forall\lambda,\mu\in K \forall v\in V: (\lambda+\mu) v = \lambda v + \mu v\]
		\item{(V5)} Normierung: $\forall v\in V: 1_K \cdot v=v$
	\end{description}
\end{definition}

\begin{definition}[Vektorräume]
	Ein \emph{Vektorraum über einem Körper $K$}, kurz \emph{$K$-Vektorraum} oder gar -- wenn der Körper klar ist oder keine Rolle spiel -- nur \emph{Vektorraum} ist eine Menge $V$ zusammen mit zwei Abbildungen, genannt $+$ und $\cdot$, die die unten gelisteten Axiome erfüllen.
	
	The first operation, called vector addition or simply addition $+ : V \times V \to V$, takes any two vectors $\vc{v}$ and $\vc{w}$ and assigns to them a third vector which is commonly written as $\vc{v} + \vc{w}$, and called the sum of these two vectors. (The resultant vector is also an element of the set $V$.)
	The second operation, called scalar multiplication $\cdot : F \times V \to V$, takes any scalar $a$ and any vector $\vc{v}$ and gives another vector $a\vc{v}$. (Similarly, the vector $a\vc{v}$ is an element of the set $V$. Scalar multiplication is not to be confused with the scalar product, also called inner product or dot product, which is an additional structure present on some specific, but not all vector spaces. Scalar multiplication is a multiplication of a vector by a scalar; the other is a multiplication of two vectors producing a scalar.)
	
	Elements of $V$ are commonly called vectors. Elements of $F$ are commonly called scalars.
	
	
	To qualify as a vector space, the set V and the operations of addition and multiplication must adhere to a number of requirements called axioms. These are listed in the table below, where $\vc{u}$, $\vc{v}$ and $\vc{w}$ denote arbitrary vectors in $V$, and $a$ and $b$ denote scalars in $F$.
	%	
	%	\begin{table}[!ht]
	%		\setlength\extrarowheight{10pt} % for a bit of visual "breathing space"
	%		\begin{tabularx}{\textwidth}{C X}
	%			
	%			\toprule
	%			\textbf{Axiom}                                                          & \textbf{Meaning}                                                                                                                                     \\ \midrule
	%			Associativity of addition                                               & $\vc{v}+(\vc{w}+\vc{u}) = (\vc{v} + \vc{w}) + \vc{u}$                                                                                                \\
	%			Commutativity of addition                                               & $\vc{u} + \vc{v} = \vc{v} + \vc{u}$                                                                                                                  \\
	%			Identity element of addition                                            & There exists an element $\vc{0} \in V$, called the zero vector, such that $\vc{v} + \vc{0} = \vc{v}$ for all $\vc{v} \in V$.                         \\
	%			Inverse elements of addition                                            & For every $\vc{v} \in V$, there exists an element $\vc{-v} \in V$, called the additive inverse of $\vc{v}$, such that $\vc{v} + (\vc{-v}) = \vc{0}$. \\
	%			Compatibility of scalar multiplication with field multiplication        & $a\cdot(b\cdot\vc{v}) = (a\cdot b)\cdot\vc{v}$                                                                                                      \\
	%			Identity element of scalar multiplication                               & $1\vc{v} = \vc{v}$, where $1$ denotes the multiplicative identity in $F$.                                                                            \\
	%			Distributivity of scalar multiplication with respect to vector addition & $a\cdot(\vc{u} + \vc{v}) = a\cdot\vc{u} + a\cdot\vc{v}$                                                                                              \\
	%			Distributivity of scalar multiplication with respect to field addition  & $(a + b)\cdot\vc{v} = a\cdot\vc{v} + b\cdot\vc{v}$                                                                                                   \\ \bottomrule
	%		\end{tabularx}
	%		\caption{Properties of a Vector space}
	%	\end{table}
\end{definition}
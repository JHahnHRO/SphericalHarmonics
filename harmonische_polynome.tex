% !TeX root = spherical_harmonics.tex
% !TeX spellcheck = de_DE
\subsection{Einige irreduzible Darstellungen von Isometriegruppen}

\begin{remark}
Wir sind immer noch daran interessiert, die Darstellungen von $O_2$ und $O_3$ zu verstehen, und, wie wir eben gelernt haben, ist ein Wesentlicher Teil dieses Vorhabens die Aufgabe, die irreduziblen Darstellungen dieser Gruppen zu finden.

In Aufgabe \ref{ex:fundamentaldarstellung_von_so3} haben wir gelernt, dass die kanonische Darstellung $V=\IK^3$ von $O_3$ irreduzibel ist. Wir wissen, dass jede Gruppe eine triviale Darstellung hat, die auch immer irreduzibel ist. Aber welche gibt es noch?
\end{remark}

\begin{definition}[Spurfreie Tensoren]
Die Tensoren im Kern der Spur $\tr_{i,j} : \Sym^m(V) \to \Sym^{m-2}(V)$ werden \emph{spurfreie} Tensoren genannt. Der Raum der symmetrischen, spurfreien Tensoren bezeichnen wir mit $\STF^m$ (\enquote{symmetric trace-free}).
\end{definition}

\begin{lemma}
Die spurfreien symmetrischen Tensoren sind eine $O_n$-Unterdarstellung von $V^{\otimes m}$.
\end{lemma}

\begin{lemma}
Der Isomorphismus zwischen symmetrischen Tensoren und polynomiellen Abbildungen übersetzt...
\begin{enumerate}
\item ...die Spur in ein Vielfaches des Laplace-Operators.
\item ...die spurfreien symmetrischen Tensoren in harmonische Polynome
\[{^n}\mathscr{H}_\IK^k := \Set{p\in\IK[x_1,\ldots,x_n] | p\,\text{homogen vom Grad}\,k, \Delta p=0}\]
\end{enumerate}
\end{lemma}

\begin{theorem}[Zerlegung der symmetrischen Tensoren in irreduzible Unterdarstellungen]
\begin{enumerate}
\item Die Darstellungen $\STF^m(V)$ sind paarweise nicht isomorph und irreduzibel.
\item Die Zerlegung von $\Sym^m(V)$ in irreduzible Bestandteile ist gegeben durch:
\[\Sym^m(V) \isomorphic \STF^m(V) \oplus \STF^{m-2}(V) \oplus \STF^{m-4}(V) \oplus \cdots\]
\end{enumerate}
\end{theorem}
\begin{proof}
a. Wir betrachten die Möglichkeiten für $O_n$-lineare Abbildungen $f: \STF^m(V) \to \STF^k(V)$. Indem wir vorher die orthogonale Projektion auf $\STF^m(V)$ anwenden und hinterher die Einbettung $\STF^k(V) \to V^{\otimes k}$ können wir $f$ als Einschränkung einer $O_n$-lineare Abbildung $V^{\otimes m} \to V^{\otimes k}$ auffassen. Und wir wissen, wie solche Abbildungen aussehen. Was tun also die Brauer-Diagramme eingeschränkt auf $\STF^m(V)$ ? Wenn auch nur eine Spur vorkommt, ist die Abbildung null, weil $\STF^m$ ja der Kern der Spur(en) ist.

Insbesondere muss also $m\leq k$ sein, damit es überhaupt Brauer-Diagramme geben kann, die auf $\STF^m(V)$ nicht Null sind. Gäbe es eine von Null verschiedene $O_n$-lineare Abbildung $\STF^m\to \STF^k$, so müssten die beiden Darstellungen mindestens einen isomorphen irreduziblen Summanden haben (Satz von Maschke+Lemma von Schur). Die Umkehrabbildung auf diesem liefert eine von Null verschiedene $O_n$-lineare Abbildung in die andere Richtung; es müsste also mit dem gleichen Argument auch $m\geq k$ sein.

Die Darstellungen sind also insbesondere paarweise nicht-isomorph. Wie viele Freiheitsgrade haben wir, um eine $O_n$-lineare Abbildung $\STF^m\to \STF^m$ festzulegen? Wir wissen, dass in Brauer-Diagrammen, die nicht die Null-Abbildung sind, keine Spuren vorkommen können. Weil $m=k$ ist, können also auch keine Casimir-Element-Einfügungen vorkommen; es muss sich um Permutationen handeln. Aber $\STF^m \subseteq\Sym^m$, d.h. alle Permutationsabbildungen sind dieselbe Abbildung auch $\STF^m$. Das heißt, es gibt höchstens einen Freiheitsgrad. Aus der Umkehrung des Lemmas von Schur folgern wir, dass $\STF^m$ irreduzibel sein muss.

\medbreak
b. Die Spur $\Sym^m(V) \to \Sym^{m-2}(V)$ ist eine surjektive, $O_n$-lineare Abbildung mit Kern $\STF^m$. Das liefert uns einen Isomorphismus zwischen dem orthogonalen Komplement von $\STF^m$ und $\Sym^{m-2}(V)$, und somit $\Sym^m(V) \isomorphic \STF^m(V) \oplus \Sym^{m-2}(V)$, woraus induktiv die Behauptung folgt.
\end{proof}
% !TeX root = spherical_harmonics.tex
% !TeX spellcheck = de_DE

\begin{remark}
Jedes Element von $SO_3$ ist bekanntlich eine Drehung $\rho_{\alpha,v}$ um eine Drehachse $v\in S^2$ und einen Drehwinkel $\alpha$. Wir erinnern noch einmal, dass $\rho_{\alpha,v} =\rho_{-\alpha,-v}$ ist.

\smallbreak
Besonders wichtig für uns ist, dass alle Drehungen um den gleichen Drehwinkel zueinander konjugiert sind. Insbesondere sind also alle Charaktere aller Darstellungen von $SO_3$ nur von $\alpha$, nicht von $v$ abhängige, (gerade) Funktionen.
\end{remark}

\begin{theorem}[Darstellungstheorie von $SO_3$]
Jede stetige, endlich-dimensionale, irreduzible Darstellung von $G=SO_3$ über $\IK\in\set{\IR,\IC}$ ist zu $STF^k$ für ein $k\in\IN$ isomorph. Der Charakter von $STF^k$ ist genau $\chi_k(\rho_{\alpha,v}) = \sum_{j=-k}^k e^{ij\alpha} = 1+\sum_{j=1}^k 2\cos(j\alpha)$. 
\end{theorem}
\begin{proof}
Wir zeigen zunächst, dass der Charakter der Darstellung $STF^k$ genau die angegebene Form hat. Wie wir eben festgestellt haben, ist es ausreichend, eine feste Drehachse zu betrachten und für jeden Drehwinkel um diese Achse den Charakterwert auszurechnen.

Wir benutzen außerdem, dass $STF^k$ zum Raum der harmonischen, homogenen Polynome vom Grad $k$ isomorph ist. Dieser Isomorphismus ist basis-abhängig und wir wählen die Basis so, dass die Unbekannte $z$ unsere feste Drehachse ist. Wir schreiben ein beliebiges homogenes Polynom dann als $p = \sum_{a=0}^k p_a(x,y)z^a$, wobei $p_a$ ein homogenes Polynom vom Grad $k-a$ ist. Dass $p$ harmonisch ist, ist äquivalent zu
\begin{align*}
0 &= \Delta p \\
\iff \partial_z^2 p &= -(\partial_x^2+\partial_y^2)p \\
\iff \sum_{a\geq 2} p_a(x,y) a(a-1)z^{a-2} &= -\sum_a (\partial_x^2+\partial_y^2)p_a z^a \\
\iff \forall a: p_{a+2}(x,y) &= \frac{-1}{(a+1)(a+2)}(\partial_x^2+\partial_y^2)p_a(x,y)
\end{align*}
Mit anderen Worten können wir $p$ eindeutig festlegen, indem wir $p_0$ und $p_1$ beliebig festlegen und diese Rekursion anwenden. Die Rekursion liefert uns einen expliziten Isomorphismus der Vektorräume $P^{k}\oplus P^{k-1}$ mit $P_h^k$. Außerdem ist die Rekursion offenbar $SO_2$-linear, weil der 2D-Laplace-Operator $SO_2$-linear ist. Das zeigt uns, dass das sogar ein Isomorphismus von Darstellungen ist. Wir erhalten also die Aussage, dass die Restriktion von $STF^k$ auf $SO_2$ isomorph ist zu $Sym^k(\IR^2)\oplus Sym^{k-1}(\IR^2)$ und wie sich die symmetrischen Tensoren in irreduzible zerlegen, wissen wir bereits.

Insbesondere können wir also die Charaktere explizit ausrechnen: Es ist
\begin{align*}
    \chi_{STF^k}(\rho_{\alpha,v}) &= \chi_{Sym^k(\IR^2)}(\rho_\alpha) + \chi_{Sym^{k-1}(\IR^2)}(\rho_\alpha) \\
    &= (2\cos(k\alpha)+2\cos((k-2)\alpha)+\ldots)+(2\cos((k-1)\alpha)+2\cos((k-3)\alpha) +\ldots) \\
    &= 1+\sum_{j=1}^k \cos(j\alpha)
\end{align*}
wie behauptet.

\medbreak
Wieso sind das nun alle irreduziblen Charaktere? Wir haben bereits festgestellt, dass jeder Charakter $\chi: SO_3\to\IK$ nur vom Drehwinkel $\alpha$ abhängig ist und eine gerade Funktion von $\alpha$ ist. Als Darstellung von $SO_2$ betrachtet, muss es also die Form $a_0 + \sum_{k>0} a_k\cdot 2\cos(k\alpha)$ mit $a_k\in\IN$ haben. Die Funktionen $2\cos(k\alpha)$ bilden eine Basis des Vektorraum solcher Funktionen. Die Funktionen $\sum_{j=0}^k \cos(j\alpha)$ also auch. Es kann also keinen weiteren irreduziblen Charakter geben, denn der müsste im selben Vektorraum liegen, aber orthogonal zu allen diesen Basiselementen sein.
\end{proof}

\begin{remark}
Man beachte, dass beide Basen im letzten Abschnitt zwar Orthonormalbasen sind, aber bzgl. verschiedener Skalarprodukte (Haar-Integral von $SO_2$ vs. $SO_3$).
\end{remark}

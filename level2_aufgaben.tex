\begin{sheet}

\begin{problem}[title={Konjugation in orthogonalen Gruppen}]\label{gruppen:ex:konjugation}
Wir haben festgestellt, dass Charaktere von Darstellungen nicht zwischen konjugierten Elementen unterscheiden, d.h.
\[ g=hg'h^{-1} \implies \chi(g) = \chi(g') \]
Es scheint also nützlich zu sein, genau zu verstehen, welche Gruppenelemente zueinander konjugiert sind.

Wir betrachten zunächst $G=O_2$.
\begin{subproblem}
Zeige, dass die Rotation um $\alpha$ und $-\alpha$ zueinander konjugiert sind.
\end{subproblem}

Machen wir nun mit $G=SO_3$ und $O_3$ weiter.
\begin{subproblem}
Zeige, dass alle Rotation um $\alpha$ zueinander konjugiert. Präziser: Zeige, dass
\[\forall g\in O_3: g\rho_{\alpha,v}g^{-1} = \rho_{\alpha,g(v)} \]
gilt, wobei $\rho_{\alpha,v}$ die Rotation um den Winkel $\alpha$ und die Drehachse $\IR v$ ist.
\end{subproblem}
\begin{subproblem}
Zeige, dass $g\iota g^{-1}=\iota$ für alle $g\in O_3$ gilt, wobei $\iota$ die Inversion $v\mapsto -v$ ist.
\end{subproblem}

Und ganz allgemein für $G=O_n$:
\begin{subproblem}
Zeige, dass alle Spiegelungen zueinander konjugiert sind. Präziser: Zeige, dass
\[\forall g\in O_2: gs_vg^{-1} = s_{g(v)}\]
gilt, wobei $s_v$ die Spiegelung an der Geraden $\IR v$ ist.
\end{subproblem}
\end{problem}

\end{sheet}
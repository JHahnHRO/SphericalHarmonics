% !TeX root = spherical_harmonics.tex
% !TeX spellcheck = de_DE
\begin{sheet}

\begin{problem}
Wir haben einen Isomorphismus zwischen Polynomen und symmetrischen Tensoren. Welches Polynom entspricht dem Casimir-Element $\Omega$ auf Polynomseite?
\end{problem}

\begin{problem}[title={Die 4 Klassen von linearen Tensor-Abbildungen sind natürlich}]
	\label{aufg:natAbbKlass}
	Zeige, dass die vier Klassen von linearen Abbildungen zwischen Tensoren tatsächlich natürliche Abbildungen sind:
	\begin{subproblem}
		Skalarmultiplikation (\ref{def:skalarmultTensor})
	\end{subproblem}
	\begin{subproblem}
		Permutation (\ref{def:permutation})
	\end{subproblem}
	\begin{subproblem}
		Spur nehmen (\ref{def:spur})
	\end{subproblem}
	\begin{subproblem}
		Casimir-Element einfügen (\ref{def:casimireinfuegen})
	\end{subproblem}
\end{problem}

\begin{problem}[title={Bunte Casimir-Elemente}]
	In der grafischen Darstellung der Überführung einer linearen Abbildung $V^{\otimes m} \to V^{\otimes n}$ zu einem Element des Tensorraums $\left(V^{\otimes m}\right)^\ast \otimes V^{\otimes n}$ haben wir Casimir-Elemente mit drei verschiedenen Farben gezeichnet. Überzeuge dich, dass diese drei Farben nicht nur der Übersichtlichkeit dienen, sondern auch die verschiedenen Arten von Casimir-Elementen deutlich machen. Wie ist es gerechtfertigt, dass die Permutation oder Identität ebenfalls auf ein Casimir-Element abgebildet wird? Wo versteckt sich die Permutation?
	
	Wir haben \emph{farbige Brauer-Diagramme} ohne Definition eingeführt, hole dies nach und finde eine sinnvolle Definition.
\end{problem}

\begin{problem}[title={Tensoren auf $O_3$-Fixpunkte projizieren}]
	\label{aufg:TensorfixProj}
	Projiziere die folgenden Tensoren auf ihre $O_3$ Fixpunkte, indem du das entsprechende Integral über $O_3$ löst:
	\begin{subproblem}
		1-Tensor $v\in\IR^3$.
	\end{subproblem}
	\begin{subproblem}
		$u\otimes u$ mit $u \in \IR^3$.
	\end{subproblem}
	\begin{subproblem}
		$u\otimes v$ mit $u\perp v \in \IR^3$.
	\end{subproblem}
	\begin{subproblem}
		$u\otimes (u+v)$ mit $u\perp v \in \IR^3$.
	\end{subproblem}
	\begin{subproblem}
		$u\otimes v \otimes w$ mit $u,v,w \in \IR^3$.
	\end{subproblem}
	\begin{subproblem}
		\label{aufg:TensorfixProjgeradeu}
		$u^{\otimes n}$ mit $u \in \IR^3$ und $n$ gerade. Hinweis: Polynome und symmetrische Tensoren.
	\end{subproblem}
	\begin{subproblem}
		\label{aufg:TensorfixProjungeradeu}
		$u^{\otimes n}$ mit $u \in \IR^3$ und $n$ ungerade.
	\end{subproblem}
\end{problem}
\end{sheet}

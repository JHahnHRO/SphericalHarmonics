% !TeX root = spherical_harmonics.tex
% !TeX spellcheck = de_DE
\subsection{Wiederholung: Gruppen}
% !TeX root = spherical_harmonics.tex
% !TeX spellcheck = de_DE
\subsection{Gruppen}

\begin{definition}[Gruppen und Gruppenhomomorphismen]\label{gruppen:def}
Eine \udot{Gruppe} $(G,\cdot)$ besteht aus
\begin{itemize}
	\item einer Menge $G$,
	\item einer Abbildung $\cdot: G\times G, (g,h) \mapsto g\cdot h$, genannt \emph{Multiplikation},
\end{itemize}
die die Gruppen-Axiome in Tabelle \ref{gruppen:def_table} erfüllen.

\begin{table}[!ht]
	\setlength\extrarowheight{10pt} % for a bit of visual "breathing space"
	\begin{tabularx}{\textwidth}{p{7cm} X}
		
		\toprule
		\textbf{Gruppen-Axiome}                               & \textbf{Bedeutung} \\
		\midrule
        \hspace{1cm}Assoziativität                           & $\forall g_1,g_2,g_3\in G: g_1\cdot(g_2\cdot g_3) = (g_1\cdot g_2)\cdot g_3$ \\
		\hspace{1cm}Neutrales Element bzw. \enquote{Eins}    & $\exists 1\in V\forall g\in G: g\cdot 1 = g = 1\cdot g$  \\
		\hspace{1cm}Inverse Elemente                         & $\forall g \in G\exists h \in G: g\cdot h = 1 = h\cdot g$ \\
		Optional kann gelten & \\
        \hspace{1cm}Kommutativität (\emph{abelsche Gruppe})  & $\forall g,h\in G: g\cdot h=h\cdot g$ \\
        \textbf{Axiom von Gruppenhomomorphismen}             & \textbf{Bedeutung} \\
        \midrule
        Verträglichkeit mit Multiplikation & $\forall g,h\in G: f(g\cdot h) = f(g)\cdot f(h)$ \\
        \bottomrule
	\end{tabularx}
	\caption{Definierende Eigenschaften von Gruppen und Gruppenhomomorphismen}
    \label{gruppen:def_table}
\end{table}

Sind $G,H$ zwei Gruppen und $f: G\to H$ eine Abbildung, so heißt $f$ \emph{Gruppenhomomorphismus}, falls sie mit den beiden Multiplikationen verträglich ist.

Existiert ein Homomorphismus $f': W\to V$ mit $f\circ f'=f'\circ f=\id$, so nennt man $f$ \emph{Isomorphismus} der Gruppen.
\end{definition}




\subsection{Elementare Darstellungstheorie}

\begin{definition}[Darstellungen]\label{darstellungen:def}
Sei $G$ eine Gruppe und $K$ ein Körper. Eine \emph{Darstellung von $G$ über $K$} besteht aus
\begin{itemize}
	\item einem $K$-Vektorraum $V$,
	\item einer Abbildung $\cdot: G \times V \to V, (g,v) \mapsto g\cdot v$
\end{itemize}
die die Axiome in Tabelle \ref{darstellungen:def_table} erfüllen. Man sagt auch \enquote{$G$ operiert auf dem Vektorraum $V$} statt \enquote{$G$ ist eine Darstellung von $G$}.

\begin{table}[!ht]
	\setlength\extrarowheight{10pt} % for a bit of visual "breathing space"
	\begin{tabularx}{\textwidth}{p{7cm} X}
		
		\toprule
		\textbf{Axiome von Darstellungen}                    & \textbf{Bedeutung} \\
		\midrule
        \hspace{1cm}Linearität                               & $\forall g\in G: (v\mapsto g\cdot v)$ ist eine lineare Abbildung $V\to V$ \\
		\hspace{1cm}Assoziativität                           & $\forall g,h\in G\forall v\in V: g\cdot (h\cdot v)=(g\cdot h)\cdot v$  \\
		\hspace{1cm}Nichtrivialität/Normierung               & $\forall v\in V: 1_G\cdot v=v$  \\
		\textbf{Axiome von Homomorphismen}                   & \textbf{Bedeutung} \\
        \midrule
        \hspace{1cm}$K$-Linearität & $f$ ist eine lineare Abbildung $V\to W$ \\
        \hspace{1cm}$G$-Linearität & $\forall g\in G, v\in V: f(g\cdot v) = g\cdot f(v)$ \\
        \bottomrule
	\end{tabularx}
	\caption{Definierende Eigenschaften von Darstellungen und Homomorphismen zwischen Darstellungen}
    \label{darstellungen:def_table}
\end{table}

Sind $V,W$ zwei Darstellungen von $G$ und $f: V\to W$ eine Abbildung, so heißt $f$ \emph{Homomorphismus} oder \emph{$KG$-lineare Abbildung}, falls die beiden Axiome in Tabelle \ref{darstellungen:def_table} erfüllt sind. Den Raum aller $K$-linearen Abbildungen von $V$ nach $W$ bezeichnen wir mit $\Hom_{KG}(V,W)$. Eine $KG$-linearen Abbildung von $V$ nach $V$ (also gleicher Definitions- und Zielraum) heißt \emph{Endomorphismus}, der Raum aller solcher Abbildungen wird mit $\End_{KG}(V)$ notiert.

Existiert ein Homomorphismus $f': W\to V$ mit $f\circ f'=f'\circ f=\id$, so nennt man $f$ \emph{Isomorphismus} der Darstellungen.
\end{definition}

\begin{remark}
Eine Darstellung kann alternativ aufgefasst werden als Gruppenhomomorphismus $G\to GL(V)$: Jedem Gruppenelement $g\in G$ wird die (invertierbare!) lineare Abbildung $v\mapsto gv$ zugeordnet. Ist umgekehrt ein Gruppenhomomorphismus $\phi: G\to GL(V)$ gegeben, so kann man $V$ als $G$-Darstellung auffassen, indem man $g\cdot v:=\phi(g)(v)$ definiert.
\end{remark}
% !TeX spellcheck = de_DE
% !TeX root = spherical_harmonics.tex
\begin{sheet}

\begin{problem}[title={Komische Minus Einsen: Invertieren für die Assoziativitätsbedingung}, difficulty=leicht]\label{ex:invertieren_fuer_linksmoduln}
\begin{subproblem}
In den Beispielen wurde definiert, dass die symmetrische Gruppe $S_m$ durch
\[\sigma \cdot (x_1,\ldots,x_m) := (x_{\sigma^{-1}(1)}, \ldots, x_{\sigma^{-1}(m)})\]
auf $\IK^m$ auf der Tensorpotenz $V^{\otimes m}$ durch
\[\sigma \cdot v_1\otimes\cdots\otimes v_m := v_{\sigma^{-1}(1)}\otimes \cdots \otimes v_{\sigma^{-1}(m)} \]
operiert.
\end{subproblem}
\begin{subproblem}
Ebenfalls definiert wurde, dass bei einer gegebenen Darstellung $T$ der Dualraum $T^\ast := \Hom_K(T,K)$ zu einer Darstellung derselben Gruppe wird, indem man sie durch
\[g\cdot \phi := t\mapsto \phi(g^{-1}t)\]
operieren lässt.
\end{subproblem}
\begin{subproblem}
Ebenfalls definiert wurde, dass $G=GL_n(\IK)$ auf $T:=\IK[X_1, ..., X_n]$ dem Raum der polynomiellen Abbildungen durch
\[g\cdot p := X\mapsto p(g^{-1}X)\]
operieren lässt.
\end{subproblem}

Beweise, dass das Invertieren in diesen Definitionen nötig ist, damit die Assoziativitätsbedingung gilt.
\end{problem}

\begin{problem}[title={Darstellung der symmetrischen Gruppe mit Matrizen}]
Wie lautet die Matrixdarstellung zu der obigen Darstellung von $S_m$ auf $\IK^m$ ?
\end{problem}

\begin{problem}[title={Dualisiere eine Matrixdarstellung}]
Wenn $\mathcal{D}: G\to\IK^{n\times n}$ eine Matrixdarstellung von $G$ auf $T$ bzgl. irgendeiner fest gewählten Basis $B\subseteq T$ ist, wie lautet die Matrixdarstellung $\mathcal{D}^\ast$ der dualen Darstellung von $G$ auf dem Dualraum $T^\ast$ bzgl. der dualen Basis $B^\ast$ ?
\end{problem}

\begin{problem}[title={Basiswechsel einer Darstellung}]
Wenn $\mathcal{D}, \mathcal{D}': G\to GL_n(\IK)$ zwei Matrixdarstellungen derselben Darstellung von $G$ auf $T$ sind, d.h. $\mathcal{D}$ bzgl. einer Basis $B$ und $\mathcal{D}'$ bzgl. einer zweiten Basis $B'$ gedacht ist, und wenn $Sin GL_n(\IK)$ die Basiswechselmatrix von $B$ nach $B'$ ist, was ist dann der Zusammenhang zwischen $\mathcal{D}$, $S$und $\mathcal{D}'$ ?
\end{problem}

\begin{problem}[title={Isomorphismus zwischen Tensorprodukt und linearen Abbildungen}]\label{ex:hom_tensor_isomorphismus}
\begin{subproblem}
Es seien $U$ und $T$ zwei endlich-dimensionale Darstellungen von $G$. Man beweise, dass die beiden Darstellungen $\Hom_\IK(U,T)$ und $U^\ast \otimes T$ via
\[\alpha: U^\ast \otimes T \to \Hom_\IK(U,T), \phi \otimes t \mapsto (u\mapsto \phi(u)t)\]
isomorph sind.
\end{subproblem}

Insbesondere erhalten wir im Spezialfall $U=T$ einen Isomorphismus zwischen $U^\ast\otimes U$ und $\End_\IK(U)$.

\begin{subproblem}
Für $U=T$, wie lautet der Tensor, der durch $\alpha$ auf $\id_U$ abgebildet wird?
\end{subproblem}

\begin{subproblem}
Zeige: Die Auswertungsabbildung $\varepsilon: U^\ast \otimes U \to \IK, \phi\otimes u\mapsto \phi(u)$ ist $G$-linear.
\end{subproblem}

\begin{subproblem}
$\varepsilon\circ\alpha^{-1}$ ist eine $G$-lineare Abbildung $\beta: \End_\IK(U) \to \IK$. Welche?
\end{subproblem}

\end{problem}

\begin{problem}[title={\enquote{Matrizen als 2-Tensoren}}]\label{tensoren:ex:matrizen}
Wir benutzen die Bezeichnungen der vorherigen Aufgabe.
\begin{subproblem}
Wenn je eine Basis $B\subseteq U, C\subseteq T$ gewählt ist, dann sei $B^\ast$ die dazu gehörige duale Basis von $U^\ast$. Was tut $\alpha$ dann mit den Elementen der Basis $\set{b^\ast\otimes c | b\in B, c\in C}$ ? Wie sehen die Darstellungsmatrizen dieser linearen Abbildungen aus?
\end{subproblem}

\begin{subproblem}\label{tensoren:ex:matrizen:spur}
Wenn wir mittels $\alpha$ und Basiswahl also $U^\ast\otimes U$ mit $\IK^{n\times n}$ identifizieren, welcher Abbildung $U^\ast \otimes U \to \IK$ entspricht dann die Spurabbildung $\tr: \IK^{n\times n}\to\IK$ ?
\end{subproblem}
\end{problem}

\begin{problem}[title={Reduzibilität von $T\otimes T$}]
Zeige, dass $T\otimes T$ (und entsprechend auch keine höhere Tensorpotenz) niemals irreduzibel ist, wenn $\dim(T) \neq 1$ ist.
\end{problem}

\begin{problem}[title={Irreduzibilität von $T^\ast\implies$ Irreduzibilität von $T$?}]
Zeige, dass $T$ irreduzibel ist, wenn $T^\ast$ es ist.
\end{problem}

\begin{problem}[title={Kanonische Darstellung von $SO_2$}]\label{ex:fundamentaldarstellung_von_so2}
\begin{subproblem}
Zeige, dass die kanonische Darstellung $V=\IR^2$ von $G=SO_2$ irreduzibel ist.
\end{subproblem}
\begin{subproblem}
Zeige, dass die komplexifizierte Darstellung $V_\IC=\IC^2$ von $G$ reduzibel ist.
\end{subproblem}
\begin{subproblem}
Finde die irreduziblen Unterdarstellungen von $V_\IC$.

Hinweis: Eigenräume.
\end{subproblem}
\end{problem}

\begin{problem}[title={Die kanonische Darstellung von $SO_3$ ist irreduzibel}]\label{ex:fundamentaldarstellung_von_so3}
Zeige, dass $\IK^3$ eine irreduzible Darstellung von $G=SO_3$ ist sowohl für $\IK=\IR$ als auch für $\IK=\IC$.
\end{problem}

\begin{problem}[title={(Nicht-)Isomorphie von $V$ und $V^\ast$}]
Sei $V$ ein endlich-dimensionaler $\IR$-Vektorraum. Zweck dieser Aufgabe ist es, genauer zu beleuchten, wieso $V$ und $V^\ast$ grundverschiedene Vektorräume sind, obwohl sie als Vektorräume ja isomorph sind (d.h. die gleiche Dimension haben)
\begin{subproblem}
\textbf{Manchmal sind sie isomorph -- Satz von Riesz}.
Es sei ein Skalarprodukt $\braket{,}$ auf $V$ gegeben. Zeige, dass $V$ und $V^\ast$ als $O(V)$-Darstellungen via $V\to V^\ast, v\mapsto \braket{v,\cdot}$ isomorph sind.
\end{subproblem}
\begin{subproblem}[difficulty={leicht bis mittel}]
\textbf{Manchmal aber auch nicht}.
Zeige, dass $V$ und $V^\ast$ \emph{nicht} isomorph sind als $GL(V)$-Darstellungen.
\end{subproblem}
\end{problem}

\begin{remark}
In allen Kontexten, in denen auch mit schiefen Koordinatensystemen gerechnet werden muss, \emph{muss} deshalb zwischen $V$ und $V^\ast$ sowie zwischen ko- und kontravarianten Tensoren unterschieden werden, da es keinen in diesem Kontext natürlichen Isomorphismus $V\isomorphic V^\ast$ gibt.

In allen Kontexten, in denen ausschließlich mit orthonormierten Koordinatensystemen gerechnet wird, existiert hingegen ein für diesen Kontext natürlichen Isomorphismus, sodass es sinnvoll ist, alle Tensoren gleich zu behandeln. Die Unnatürlichkeit des allgemeinen Falls wird so in der Wahl des Skalarprodukts versteckt.
\end{remark}

\begin{problem}[title={Quotienten}]
Das duale Konzept zu Unterdarstellungen sind Quotientendarstellungen. Zeige: Ist $V$ eine Darstellung von $G$ und $U\leq V$ ein $G$-invarianter Unterraum, dann ist auch der Quotient $V/U$ eine Darstellung von $G$ via $g\cdot\overline{v} := \overline{g\cdot v}$.
\end{problem}
	
\end{sheet}
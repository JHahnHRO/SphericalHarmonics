% !TeX spellcheck = de_DE
% !TeX root = spherical_harmonics.tex
\begin{sheet}

\begin{problem}[title={Komische Minus Einsen}, difficulty=leicht]\label{ex:invertieren_fuer_linksmoduln}
\begin{subproblem}
In den Beispielen wurde definiert, dass die symmetrische Gruppe $Sym(m)$ durch
\[\sigma \cdot (v_1,\ldots,v_m) := (v_{\sigma^{-1}(1)}, \ldots, v_{\sigma^{-1}(m)})\]
auf $K^m$ auf der Tensorpotenz $V^{\otimes m}$ durch
\[\sigma \cdot v_1\otimes\cdots\otimes v_m := v_{\sigma^{-1}(1)}\otimes \cdots \otimes v_{\sigma^{-1}(m)} \]
operiert.
\end{subproblem}
\begin{subproblem}
Ebenfalls definiert wurde, dass bei einer gegebenen Darstellung $V$ der Dualraum $V^\ast := \Hom_K(V,K)$ zu einer Darstellung derselben Gruppe wird, indem man sie durch
\[g\cdot \phi := v\mapsto \phi(g^{-1}v)\]
operieren lässt.
\end{subproblem}

Beweise, dass das Invertieren in beiden Definitionen nötig ist, damit die Assoziativitätsbedingung gilt.
\end{problem}

\begin{problem}
Wie lautet die Matrixdarstellung zu der obigen Darstellung von $Sym(m)$ auf $K^m$ ?
\end{problem}

\begin{problem}
Wenn $\Delta: G\to K^{n\times n}$ eine Matrixdarstellung von $G$ auf $V$ bzgl. irgendeiner fest gewählten Basis $B\subseteq V$ ist, wie lautet die Matrixdarstellung $\Delta^\ast$ der dualen Darstellung von $G$ auf dem Dualraum $V^\ast$ bzgl. der dualen Basis $B^\ast$ ?
\end{problem}

\begin{problem}[title={Basiswechsel}]
Wenn $\Delta$ und $\Delta'$ zwei Matrixdarstellungen derselben Darstellung von $G$ auf $V$ sind, wenn d.h. $\Delta$ bzgl. einer Basis $B$ und $\Delta'$ bzgl. einer zweiten Basis $B'$ gedacht ist, und wenn $S$ die Basiswechselmatrix von $B$ nach $B'$ ist, was ist dann der Zusammenhang zwischen $\Delta$, $S$und $\Delta'$ ?
\end{problem}

\begin{problem}\label{ex:hom_tensor_isomorphismus}
\begin{subproblem}
Es seien $V$ und $W$ zwei endlich-dimensionale Darstellungen von $G$. Man beweise, dass die beiden Darstellungen $\Hom_K(V,W)$ und $V^\ast \otimes W$ via
\[\alpha: V^\ast \otimes W \to \Hom_K(V,W), \phi \otimes w \mapsto (v\mapsto \phi(v)w)\]
isomorph sind.
\end{subproblem}

Insbesondere erhalten wir im Spezialfall $V=W$ einen Isomorphismus zwischen $V^\ast\otimes V$ und $\Hom_K(V,V)$.

\begin{subproblem}
Für $V=W$, wie lautet der Tensor, der durch den Isomorphismus auf $\id_V$ abgebildet wird?
\end{subproblem}

\begin{subproblem}
Zeige: Die Auswertungsabbildung $\varepsilon: V^\ast \otimes V \to K, \phi\otimes v\mapsto \phi(v)$ ist $G$-linear.
\end{subproblem}

\begin{subproblem}
$\varepsilon\circ\alpha^{-1}$ ist eine $G$-lineare Abbildung $\beta: \Hom_K(V,V) \to K$. Welche?
\end{subproblem}
\end{problem}


\begin{problem}
Zeige, dass $V\otimes V$ niemals irreduzibel ist, wenn $\dim(V) \neq 1$ ist.
\end{problem}

\begin{problem}
Zeige, dass $V$ irreduzibel ist, wenn $V^\ast$ es ist.
\end{problem}


\begin{problem}[title={Kanonische Darstellung von $SO_2$}]\label{ex:fundamentaldarstellung_von_so2}
\begin{subproblem}
Zeige, dass die kanonische Darstellung $V=\IR^2$ von $G=SO_2$ irreduzibel ist.
\end{subproblem}
\begin{subproblem}
Zeige, dass die komplexifizierte Darstellung $V_\IC=\IC^2$ von $G$ reduzibel ist.
\end{subproblem}
\begin{subproblem}
Finde die irreduziblen Unterdarstellungen von $V_\IC$.

Hinweis: Eigenräume.
\end{subproblem}
\end{problem}

\begin{problem}[title={Die kanonische Darstellung von $SO_3$ ist irreduzibel}]\label{ex:fundamentaldarstellung_von_so3}
Zeige, dass $\IK^3$ eine irreduzible Darstellung von $G=SO_3$ ist sowohl für $\IK=\IR$ als auch für $\IK=\IC$.
\end{problem}

\begin{problem}[title={Was geht bei $SO_2$ schief?}]
Spoiler: $SO_2$ ist kommutativ.

\begin{subproblem}
Zeige (oder erinnere dich) zunächst: Ist $\lambda\in\IC$ beliebig und sind $\alpha,\beta\in\End_\IC(V)$ zwei kommutierende Endomorphismen (d.h. $\alpha\circ\beta=\beta\circ\alpha$), dann ist der Eigenraum $\operatorname{Eig}_\lambda(\alpha)$ ein $\beta$-invarianter Untervektorraum.
\end{subproblem}
\begin{subproblem}
Zeige, dass alle endlich-dimensionalen, irreduziblen, komplexen Darstellungen einer kommutativen Gruppe $G$ eindimensional sind.
\end{subproblem}
\end{problem}

\begin{problem}[title={Isomorphie von $V$ und $V^\ast$}]
Sei $V$ ein endlichdimensionaler $\IR$-Vektorraum. Zweck dieser Aufgabe ist es, genauer zu beleuchten, wieso $V$ und $V^\ast$ Grund verschiedene Vektorräume sind, obwohl sie als Vektorräume ja isomorph sind (d.h. die gleiche Dimension haben)
\begin{subproblem}
\textbf{Manchmal sind sie isomorph -- Satz von Riesz}.
Es sei ein Skalarprodukt $\braket{,}$ auf $V$ gegeben. Zeige, dass $V$ und $V^\ast$ als $SO(V)$-Darstellungen via $V\to V^\ast, v\mapsto \braket{v,\cdot}$ isomorph sind.
\end{subproblem}
\begin{subproblem}[difficulty={schwerer als man denkt}]
\textbf{Manchmal aber auch nicht}.
Zeige, dass $V$ und $V^\ast$ \emph{nicht} isomorph sind als $GL(V)$-Darstellungen.
\end{subproblem}
\end{problem}
	
\end{sheet}
\begin{sheet}

\begin{problem}[title={Casimir-Elemente von euklidischen Räumen}]
Es sei $V$ ein endlich-dimensionaler $\IR$-Vektorraum mit Skalarprodukt und $e_1, ..., e_n$ eine Orthonormalbasis.

\begin{subproblem}
Zeige, dass der Casimir-Tensor
\[\Omega_V := \sum_{i=1}^n e_i\otimes e_i\]
unabhängig von der Basiswahl ist, d.h. wenn $e_1', ..., e_n'$ eine weitere Orthonormalbasis von $V$ ist, dann gilt:
\[\sum_{i=1}^n e_i\otimes e_i = \sum_{i=1}^n e_i'\otimes e_i'\]

Hinweis: Orthogonale Matrizen.
\end{subproblem}

\begin{subproblem}
Zeige, dass $\Omega_V$ \enquote{isotrop} ist, d.h. für alle Isometrien $Q: V\to V$ gilt:
\[(Q\otimes Q)(\Omega) = \Omega\]

Hinweis: Benutze a.
\end{subproblem}
\end{problem}

\begin{problem}[title={Casimir-Elemente allgemein}]
Es sei $V$ ein endlich-dimensionaler $K$-Vektorraum, $b_1, ..., b_n$ eine beliebige Basis von $V$ und $b_1^\ast, ..., b_n^\ast$ die dazu duale Basis von $V^\ast$.

\begin{subproblem}
Zeige, dass
\[\Omega := \sum_{i=1}^n b_i \otimes b_i^\ast \in V\otimes V^\ast\]
unabhängig von der Basiswahl ist.

Hinweis: Wenn $A$ eine Basiswechselmatrix zwischen zwei Basen von $V$ ist, wie sieht dann die Basiswechselmatrix der beiden dazugehörigen dualen Basen von $V^\ast$ aus?
\end{subproblem}
\begin{subproblem}
Wie entspricht das dem Casimir-Element euklidischer Räume?
\begin{enumerate}[label=\roman*.)]
\item Zeige zunächst, dass die Abbildung $V\to V^\ast, v\mapsto \braket{v,-}$ ein Isomorphismus $V\to V^\ast$ ist.
\item Was tut diese Abbildung mit einer Orthonormalbasis?
\end{enumerate}
\end{subproblem}
\end{problem}

\end{sheet}
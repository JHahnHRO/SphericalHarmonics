% !TeX root = spherical_harmonics.tex
% !TeX spellcheck = de_DE

\begin{sheet}

\begin{problem}[title={Casimir-Elemente von euklidischen Räumen}]
Es sei $V$ ein endlich-dimensionaler $\IR$-Vektorraum mit Skalarprodukt und $e_1, ..., e_n$ eine Orthonormalbasis.

\begin{subproblem}
Zeige, dass der Casimir-Tensor
\[\Omega_V := \sum_{i=1}^n e_i\otimes e_i\]
unabhängig von der Basiswahl ist, d.h. wenn $e_1', ..., e_n'$ eine weitere Orthonormalbasis von $V$ ist, dann gilt:
\[\sum_{i=1}^n e_i\otimes e_i = \sum_{i=1}^n e_i'\otimes e_i'\]

Hinweis: Orthogonale Matrizen.
\end{subproblem}

\begin{subproblem}
Zeige, dass $\Omega_V$ \enquote{isotrop} ist, d.h. für alle Isometrien $Q: V\to V$ gilt:
\[(Q\otimes Q)(\Omega) = \Omega\]

Hinweis: Benutze a.
\end{subproblem}
\end{problem}

\begin{problem}[title={Casimir-Elemente allgemein}]
Es sei $V$ ein endlich-dimensionaler $K$-Vektorraum, $b_1, ..., b_n$ eine beliebige Basis von $V$ und $b_1^\ast, ..., b_n^\ast$ die dazu duale Basis von $V^\ast$.

\begin{subproblem}
Zeige, dass
\[\Omega := \sum_{i=1}^n b_i \otimes b_i^\ast \in V\otimes V^\ast\]
unabhängig von der Basiswahl ist.

Hinweis: Wenn $A$ eine Basiswechselmatrix zwischen zwei Basen von $V$ ist, wie sieht dann die Basiswechselmatrix der beiden dazugehörigen dualen Basen von $V^\ast$ aus?
\end{subproblem}
\begin{subproblem}
Wie entspricht das dem Casimir-Element euklidischer Räume?
\begin{enumerate}[label=\roman*.)]
\item Zeige zunächst, dass die Abbildung $V\to V^\ast, v\mapsto \braket{v,-}$ ein Isomorphismus $V\to V^\ast$ ist.
\item Was tut diese Abbildung mit einer Orthonormalbasis?
\end{enumerate}
\end{subproblem}
\end{problem}

\begin{problem}[title={Aber Tensoren sind doch so Buchstaben mit Indizes}]
	\label{ex:TensorenMitIndizes}
	Häufig wird einem von Physikern oder Ingenieuren ein Tensor lediglich als ein Buchstabe mit Indizes untergejubelt - z.B. der Spannungstensor $\sigma_{ij}$. Wir wollen verstehen, wie der Zusammenhang mit unserer Definition ist.

	Es sei $V$ ein endlich-dimensionaler $\IR$-Vektorraum mit Skalarprodukt und $e_1, ..., e_n$ eine Orthonormalbasis. Wir betrachten das $m$-fache Tensorprodukt $V^{\otimes m}$. Wer mag, kann zur Vereinfachung $n=3$ und $K=\IR$ wählen.
	\begin{subproblem}
		Einen beliebigen Tensor aus $V^{\otimes m}$ schreiben wir z.B. als $T\in V^{\otimes m}$, während er andernorts mit $T_{j_1 \cdots j_n}$ bezeichnet wird, was streng genommen nur eine Kollektion besonders nummerierter Zahlen aus $K$ ist. Wie ist der Zusammenhang zwischen $T$ und $T_{j_1 \cdots j_n}$?

		Hinweise: Von $V$ induzierte Basiswahl für $V^{\otimes m}$, Vergleiche mit einem Vektor $v\in V$ und $v_i$.
	\end{subproblem}
	\begin{subproblem}
		Ein sehr häufig verwendeter \enquote{Buchstabe mit Indizes} ist das Kronecker-$\delta$, oder auch der $\delta_{ij}$-Tensor.
		Um welchen Tensor handelt es sich hier?
		Hinweis: Übersetze in die Schreibweise mit dem Tensorprodukt $\otimes$.
	\end{subproblem}
	\begin{subproblem}
		Die Spur eines Tensors zwischen seinem $k$-ten und $l$-ten Faktor wird in Indexschreibweise als Dopplung eines bestimmten Indexes an den entsprechenden Stellen notiert, $T_{j_1\cdots j_{k-1} i j_{k+1} \cdots j_{l-1} i j_{l+1}  \cdots j_m}$, die eine Summe über $i$ von 1 bis $n$ impliziert (a.k.a. \emph{Einstein'sche Summenkonvention}). Überzeuge dich, dass dies unserer Definition von Spur entspricht.

		Üblich zum Spur nehmen ist auch eine Schreibweise mit dem Kronecker-$\delta$: \[T_{j_1\cdots j_{k-1} i j_{k+1} \cdots j_{l-1} r j_{l+1}  \cdots j_m} \delta_{ir},\] ebenfalls mit impliziter Summe über gedoppelte Indizes. Zeige, dass dies die gleiche Operation beschreibt.
	\end{subproblem}
	\begin{subproblem}
		Eine weitere häufiger zu findende Schreibweise mit dem Kronecker-$\delta$ ist die folgende:
		\[T_{j_1\cdots j_{k-1} j_{k} j_{k+1} \cdots j_{l-1} j_l j_{l+1}  \cdots j_m} \delta_{ir},\]
		was üblicherweise gekürzt wird auf $T_{j_1\cdots j_m} \delta_{j_{m+1} j_{m+2}}$. Was ist der Unterschied zu c.)? Schreibe diesen Tensor ohne Indizes auf.
	\end{subproblem}
\end{problem}

\begin{problem}[title={Was denn für Indizes?}]
	Schön, dass dir die Index-Schreibweise noch nicht begegnet ist. Da wir in unserem Kurs so wenig wie möglich mit dieser Schreibweise arbeiten wollen, kann dies zu deinem Vorteil sein.

	Es sei $V$ ein endlich-dimensionaler $\IR$-Vektorraum mit Skalarprodukt und $e_1, ..., e_n$ eine Orthonormalbasis. Wir betrachten das $m$-fache Tensorprodukt $V^{\otimes m}$. Wer mag, kann zur Vereinfachung $n=3$ und $K=\IR$ wählen. Zeige, dass $\tr_{ir}(T)=\tr_{i,m+1}\circ\tr_{r,m+1}\circ\casimir(T)$ gilt.

\end{problem}


\begin{problem}[title={Warum heißt das Spur?}]
	Der Begriff Spur kommt einem üblicherweise als Summe der Diagonaleinträge einer Matrix unter. Überzeuge dich, dass es sich im Falle von 2-Tensoren um genau diese Operation handelt.
\end{problem}

\begin{problem}[title={Brauer-Diagramme}]
	Wir wollen uns im Folgenden anhand ein paar Beispielen überzeugen, dass die Multiplikation zweier Brauer-Diagrammen tatsächlich dem Hintereinanderausführen der dazugehörigen linearen Abbildungen entspricht.
	\begin{subproblem}
		Schreibe die Brauer-Diagramme von den beiden Abbildungen $\Pi_\sigma$ und $\casimir$ aus Gleichung \ref{eq:sigmaOfCasimir} auf und berechne das Brauer-Diagramm der  Abbildung $\Pi_\sigma \circ\casimir$.
	\end{subproblem}
	\begin{subproblem}
		Wir haben für die Multiplikation zweier Brauer-Diagramme definiert, dass für jeden geschlossenen Kreis das Ergebnis-Diagramm mit dem Faktor $\dim{V}$ multipliziert wird. Dieser Faktor muss so festgelet werden, um Brauer-Diagramme als äquivalente Schreibweise für die vorgestellten linearen Abbildungen zwischen Tensoren benutzen zu können. Warum?
		
		Hinweis: Spur vom Casimir-Element
	\end{subproblem}
	\begin{subproblem}
		Gegeben seien die folgenden Brauer-Diagramme:
		\begin{align*}
			B_5&=\begin{tikzpicture}[baseline={([yshift=-.5ex]current bounding box.center)}]
				\foreach\x in {1,2,...,4}{
					\node[v] (s\x) at (\x,1){};
				}
				\foreach\x in {1,2,...,6}{
					\node[v] (h\x) at (\x,0){};
				}
				\foreach\x in {1,2}{
					\draw (s\x) to (h\x);
				}
			\draw (s3) to (h5);
			\draw (s4) to (h6);
			\draw (h3) to [bend left=30] (h4);
			\end{tikzpicture}
			&
			B_7&= 
			\begin{tikzpicture}[baseline={([yshift=-.5ex]current bounding box.center)}]
				\foreach\x in {1,2,...,6}{
					\node[v] (h\x) at (\x,0){};
				}
				\foreach\x in {1,3,4,6}{
					\node[v] (s\x) at (\x,1){};
				}
				\foreach\x in {1,3,4,6}{
					\draw (s\x) to (h\x);
				}
				\draw (h2) to [bend left=30] (h5);
			\end{tikzpicture}
			\\ \\ \\
			B_6&=
			\begin{tikzpicture}[baseline={([yshift=-.5ex]current bounding box.center)}]
				\foreach\x in {1,2,...,4}{
					\node[v] (h\x) at (\x,0){};
				}
				\foreach\x in {1,2,...,6}{
					\node[v] (s\x) at (\x,1){};
				}
				\draw (s1) to (h1);
				\draw (s4) to (h2);
				\draw (s5) to (h3);
				\draw (s6) to (h4);
				\draw (s2) to [bend right=30] (s3);
			\end{tikzpicture}
			\hspace{1cm}
			&
			B_8&=
			\begin{tikzpicture}[baseline={([yshift=-.5ex]current bounding box.center)}]
				\foreach\x in {1,2,...,4}{
					\node[v] (h\x) at (\x,0){};
				}
				\foreach\x in {1,2,...,6}{
					\node[v] (s\x) at (\x,1){};
				}
				\foreach\x in {1,2,...,4}{
					\draw (s\x) to (h\x);
				}
				\draw (s5) to [bend right=30] (s6);
			\end{tikzpicture}
		\end{align*}
		Zeige, dass das Ergebnis der Multiplikation $B_6\circ B_5$ der linearen Abbildung, die durch Hintereinanderausführen von $B_5$ und $B_6$ gegeben ist, entspricht. Zeige dies auch für $B_7 \circ B_8$.
	\end{subproblem}
\end{problem}

\begin{problem}[title={Wiederholung von Isomorphie zwischen Vektorräumen: Matrix vs. 2-Tensor}]
\todo{Ins Darstellungstheoriekapitel}
	Matrizen und 2-Tensoren sind einander sehr ähnlich, so ähnlich sogar, dass sie zueinander isomorph sind. Gegeben ein $K$-Vektorraum $V$ der Dimension $n$. Genauer: Der Raum der Matrizen $K^{n\times n}$ und $V\otimes V^\ast$ sind isomorph zueinander. Wir wollen nun diesen Isomorphismus genauer betrachten.
	
$V\otimes V$, $V\otimes V^\ast$, $V^\ast \otimes V^\ast$
\end{problem}

\begin{problem}[title={Wilde Behauptungen}, difficulty={schwer}]
Im Skript wurde behauptet:

\enquote{alle physikalisch sinnvollen Dinge [sind] von sich aus basisfrei und somit zwangsläufig auch basisfrei berechenbar, wenn sie überhaupt berechenbar sind}

Überzeuge dich davon, dass das nicht nur so daher gesagt ist, sondern ein beweisbarer Fakt ist. Insbesondere ist hier als Teilbehauptung enthalten: Es ist möglich, physikalisch sinnvolle Daten (sowohl die Input- als auch Output-Daten der Berechnung) basisfrei so zu repräsentieren, dass damit immer noch Berechnungen möglich sind.

\medbreak
\textcolor{red}{(Und es sei erneut davor gewarnt, dass \enquote{berechnen} nicht \enquote{effizient berechnen} bedeutet)}
\end{problem}

\end{sheet}